\documentclass[12pt,fleqn]{article}
\setlength{\parindent}{0pt}
\usepackage{graphicx}
\usepackage{cancel}
\usepackage{listings}
\usepackage[latin5]{inputenc}
\usepackage{color}
\setlength{\parskip}{8pt}
\setlength{\parsep}{0pt}
\setlength{\headsep}{0pt}
\setlength{\topskip}{0pt}
\setlength{\topmargin}{0pt}
\setlength{\topsep}{0pt}
\setlength{\partopsep}{0pt}
\setlength{\mathindent}{0cm}
\usepackage{latexsym}
\usepackage{showkeys}
\renewcommand*\showkeyslabelformat[1]{(#1)}

\begin{document}
Ders 1 

Kumeler 

Eger $S$ kumesi ``yukaridan sinirlanmis (bounded from above)'' ise o zaman
$x \in S$ icin oyle bir $y$ var demektir ki her $x$ icin $x \le y$
olsun. Yani $S$ icindeki her deger bu $y$ degerinden kucuk olsun. Bu $x$
degerine $S$'in supremum'u da deniyor, ve $\sup\limits_{x \in S}(x)$ ya da
$sup\{x:x \in S\}$ olarak gosterilebiliyor. Benzer sekilde kumenin en alt
siniri, yani infimum degeri $\inf\limits_{x \in S}(x)$ ya da $inf\{x:x \in
S\}$ olarak gosteriliyor. 

















\end{document}

\documentclass[12pt,fleqn]{article}\usepackage{../common}
\begin{document}
Ders 7

Yansitma Teorisi 

Tanim 

Bir On-Hilbert uzayindaki iki vektor $x,y$'nin ortogonal oldugu soylenir
eger $(x|y) = 0$ ise. Bunu sembolize etmek icin $x \perp y$ kullaniriz. Bir
vektor $x$ kume $S$'ye ortogonaldir ($x \perp S$ olarak gosterilir), eger
her $s \in S$ icin $x \perp s$ ise. 


\end{document}

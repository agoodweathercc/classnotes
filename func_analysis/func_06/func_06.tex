\documentclass[12pt,fleqn]{article}\usepackage{../common}
\begin{document}
Ders 6

Hilbert Uzaylari 

Giris 

Her lise geometri ogrencisi bir noktadan bir cizgiye olan en kisa mesafenin
o cizgiye dik olan ikinci bir cizgiden gectigini bilir. Kabaca da hemen
gorulebilecek akla yatkin bu basit sonuc, noktadan duzleme olan mesafeler
icin de kolayca genellestirilebilir. Daha da ileri gidip n-boyutlu Oklit
uzaylarina genellemek gerekirse, bir noktadan bir altuzaya gidecek en kisa
vektorun o altuzaya ortagonal olacagidir. Hatta bu son sonuc en kuvvetli,
onemli optimizasyon prensiplerinin biri olan Yansitma Teorisi'nin ozel
sartlardindan biridir.

Bu gozlemde kritik puf nokta ortoganalliktir. Ortoganallik kavrami genel
olarak norm edilmis uzaylarda mevcut degildir, ama Hilbert Uzaylarinda
mevcuttur. Hilbert Uzayi norm edilmis uzaylarin ozel bir halidir, norm
edilmis uzaylardaki ozelliklere ek olarak bir de icsel carpim (inner
product) islemi tanimlar, bu islem analitik geometrideki iki vektorun
noktasal carpimina (dot product) esdegerdir, iki vektorun icsel carpimi
sifir ise o vektorlerin ortoganal oldugu soylenecektir. 

Icsel carpim ile kusanmis Hilbert Uzaylari iki ve uc boyutlardaki geometrik
buluslari genellememizi saglayacak yapisal bir cevher saglar bize, sonuc
olarak pek cok analitik cozum Hilbert Uzaylarina uygulanabilir
haldedir. Ortonormal bazlar, Fourier Serileri, en az kareler minimizasyonu
gibi kavramlarinin hepsinin Hilbert Uzayinda karsiliklari vardir. 

On-Hilbert Uzaylari (Pre-Hilbert Spaces)





\end{document}

\documentclass[12pt,fleqn]{article}\usepackage{../common}
\begin{document}
Ders 17

Bugun dikeylik / ortogonallik (orthogonality) ders dizisinin
sonuncusundayiz. Ortogonal vektorleri, iki tanesini, gorduk, ortogonal
altuzaylari gorduk, ki bunlar satir uzayi ve null uzayi idi, bugun
ortogonal baz ve ortogonal matrisi gorecegiz. 

Simdi ortonormal (orthonormal) kelimesinden bahsetmek istiyorum. Bu arada
bu derte $q$ harfini ortogonal vektorleri temsil etmek icin kullanacagim. 

Ortonormal vektorler 

\[ 
q_i^Tq_j = \left\{ \begin{array}{lll}
0 & eger & i \ne j \ ise\\
0 & eger & i = j \ ise
\end{array} \right.
 \]

Yani $q$ vektorleri diger her $q$'ya (kendisi haricinde) ortogonal. 



\end{document}

\documentclass[12pt,fleqn]{article}\usepackage{../common}
\begin{document}
Ders 17

Bugun dikeylik / ortogonallik (orthogonality) ders dizisinin
sonuncusundayiz. Ortogonal vektorleri, iki tanesini, gorduk, ortogonal
altuzaylari gorduk, ki bunlar satir uzayi ve null uzayi idi, bugun
ortogonal baz ve ortogonal matrisi gorecegiz. 

Simdi ortonormal (orthonormal) kelimesinden bahsetmek istiyorum. Bu arada
bu derte $q$ harfini ortogonal vektorleri temsil etmek icin kullanacagim. 

Ortonormal vektorler 

\[ 
q_i^Tq_j = \left\{ \begin{array}{lll}
0 & eger & i \ne j \ ise\\
1 & eger & i = j \ ise
\end{array} \right.
 \]

Yani $q$ vektorleri diger her $q$'ya (kendisi haricinde) ortogonal. Her
biri bir digerine 90 derece dik vektorlerden olusan bir baz olmasi dogal
bir sey. $q$ vektorleri birim, bu sebeple kendisiyle noktasal carpimi
1. Ortonorma kelimesindeki ``normal'' buradan geliyor, normalize edilmis
vektorlerimiz var. 

Ortonormal baza sahip olmak hesaplari basitlestirir, cogu hesabi
iyilestirir, sayisal lineer cebirin cogunlugu ortonormal vektorlerle is
yapmak etrafinda kurulmustur, cunku onlar asiri buyumezler, asiri
kuculmezler, kontrol altinda is yapmak mumkun olur. 

Bu $q$'leri $Q$ icine koyacagiz. Dersin ikinci kisminda eger ortonormal bir
$A$ matrisim var ise, onu nasil ortonormal yaparim, onu gorecegiz. Simdi
ustteki 1 ve 0 iceren formulu matris olarak yazmak istiyorum. 

\[ 
Q = 
\left[\begin{array}{rrr}
\uparrow &  & \uparrow \\
q_1 & ... &  q_n \\
\downarrow &  & \downarrow 
\end{array}\right]
 \]

O zaman 

\[ 
Q^TQ = 
\left[\begin{array}{rrr}
\leftarrow & q_1^T & \rightarrow \\
& &  \\
\leftarrow & q_n^T & \rightarrow 
\end{array}\right]
\left[\begin{array}{rrr}
\uparrow &  & \uparrow \\
q_1 & ... &  q_n \\
\downarrow &  & \downarrow 
\end{array}\right] = 
I
 \]

Eger $Q$ kare ise, $Q^TQ = I$  bize $Q^T = Q^{-1}$ oldugunu soyler. 

Ornek 

Her permutasyon matrisi 

\[ 
Q = 
\left[\begin{array}{rrr}
0 & 0 & 1 \\
1 & 0 & 0 \\
0 & 1 & 0 
\end{array}\right]
 \]

kendi devrigi ile carpilinca, yani 

\[ 
Q^T = 
\left[\begin{array}{rrr}
0 & 1 & 0 \\
0 & 0 & 1 \\
1 & 0 & 0 
\end{array}\right]
 \]

ile, sonuc $I$ olacaktir. Bir diger ornek 

\[ Q = 
\left[\begin{array}{cc}
cos \theta & -sin \theta \\
sin \theta & cos \theta \\
\end{array}\right]
 \]


Ornek

\[ Q = 
\left[\begin{array}{cc}
1 & 1 \\
1 & -1 \\
\end{array}\right]
 \]

ortogonal matris degildir (simdilik). Her kolonun uzunlugu $\sqrt{2}$, o
zaman tum matrisi $\sqrt{2}$ bolerim,

\[ Q = \frac{1}{\sqrt{2}}
\left[\begin{array}{cc}
1 & 1 \\
1 & -1 \\
\end{array}\right]
 \]

Ornek 

\[ Q = \frac{ 1}{2}
\left[\begin{array}{rrrr}
1 & 1& 1& 1 \\
1 & -1& 1& -1 \\
1 & 1& -1& -1 \\
1 & -1& -1& 1 
\end{array}\right]
 \]

Ortogonal matrisler niye iyidir? Onlar hangi hesaplari basitlestirirler? 










\end{document}

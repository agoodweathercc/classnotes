\documentclass[12pt,fleqn]{article}
\setlength{\parindent}{0pt}
\usepackage{graphicx}
\usepackage{cancel}
\usepackage{listings}
\usepackage[latin5]{inputenc}
\usepackage{color}
\setlength{\parskip}{8pt}
\setlength{\parsep}{0pt}
\setlength{\headsep}{0pt}
\setlength{\topskip}{0pt}
\setlength{\topmargin}{0pt}
\setlength{\topsep}{0pt}
\setlength{\partopsep}{0pt}
\setlength{\mathindent}{0cm}
\usepackage{latexsym}
\usepackage{showkeys}
\renewcommand*\showkeyslabelformat[1]{(#1)}

\begin{document}
Ders 19 

Bu dersten baslayarak ve birkac ders boyunca cogu muhendis ve bazi
bilimcinin, onlarin karsilastigi turden tum diferansiyel denklemleri
cozmekte en populer buldugu yontemi gorecegiz. Yontemin ismi Laplace
Transformu. 

Bu yontemi kullanmak icin birkac hafta yeterli, ama o zaman bile metot
etrafinda belli bir gizem bulutu kaliyor, insanlar teknigin nereden
geldigini bir turlu anlayamiyorlar, ve dogal olarak bu onlari rahatsiz
ediyor.  

Laplace Transformunu anlamanin iyi bir yolu onu ustel seri (power series)
olarak gormektir. Bir ustel seri bildigimiz gibi su formdadir

\[ \sum_{0}^{\infty} a_n x^n \]

Bu seriye yapilacak en dogal islem onu toplamaktir. Sonuc genelde $f(x)$
gibi bir genel tanimla gosterilir, biz burada gelenekten biraz kopacagiz,
toplamin $a$ ile iliskisini iyice belli etmek icin $A(x)$ kullanacagiz. 

\[ \sum_{0}^{\infty} a_n x^n = A(x)\]

Bir degisiklik daha: $a_n$ aslinda bir ayriksal dizin icindeki belli $a$
degerleri, bunu da iyice belli etmek icin bilgisayar notasyonu kullanalim,
$a_n$, $a(n)$ olsun. 

\[ \sum_{0}^{\infty} a(n) x^n = A(x)\]

Bu sekilde bakinca, ustel serinin yaptigi bir ayriksal fonksiyonu $a$'yi
(cunku icinde reel sayilar var, ve bir fonksiyon) belli bir toplam ile
ilintilendirmek. 

\[ a(n) \leadsto A(x) \]

Peki eger $a(n) = 1$ ise, yani fonksiyon hep ayni sabit deger 1'i
veriyorsa, o zaman toplam ne olur? 

\[ 1 \leadsto \frac{1}{1-x}, |x|<1 \]

cunku $a(n) = 1$ ise ustel seri 

\[ 1 \cdot x + 1 \cdot x^2 + 1 \cdot x^3 + ... \]

\[ = x + x^2 + x^3 + ... \]

olacaktir, ve bu toplam $1/1-x$'e yaklasir (dikkat: $|x|<1$ oldugu
durumda). 

















\end{document}

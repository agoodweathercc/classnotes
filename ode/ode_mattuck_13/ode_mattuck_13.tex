\documentclass[12pt,fleqn]{article}\usepackage{../common}
\begin{document}
Ders 13

Bugunku dersimizin hedefi ozel cozumler bulmak. Formu yazalim

\[ y'' + Ay' + By = f(x) \]

Ve genel cozum $y = y_p + c_1y_1 + c_2y_2$ formunda olacak. 

Gercek su ki esitligin sag tarafina yazilabilecek her fonksiyon o kadar
ilginc degil. Ilginc olanlardan bir tanesi ustel (exponential)
fonksiyonlar, yani $e^{ax}$ formundaki fonksiyonlar, ki cogunlukla $a<0$
kullanilir. Diger bazi ilginc olanlar 

\[ \sin \omega x \]

\[ \cos \omega x \]

gibi salinim
ornekleri, ki bunlar da elektriksel devrem baglaminda alternatif AC/DC
akimi temsil ediyorlar.

Ya da ``gittikce yokolan salinim'' ilginc, Burada 

\[ e^{ax}sin \omega x \]

\[ e^{ax}cos \omega x \]

gibi ornekler var. Ama aslinda ustteki tum ilginc fonksiyonlar genel tek
bir forma baglanabilir, bu form icin ustel sayinin kompleks olmasina izin
vermek gerekiyor. Form soyle

\[ e^{(a+i\omega)x} \]

$\omega = 0$ ise o zaman $e^{ax}$ elde ederim. $a=0$ ise $sin\omega x$,
$cos\omega x$ elde ederim. Ikisi de sifir degilse o zaman gittikce yokolan
salimi elde ederim. 

Bundan sonra habire $a+i\omega$ yazmamak icin onun yerine $\alpha$
kullanacagiz, $\alpha$'yi gorunce onun bir kompleks sayi oldugunu
anlayin. Yani esitligin sag tarafi $e^{\alpha x}$ olacak. 

Birazdan gorecegimiz uzere, bu tur bir girdi kullanmak aslinda kolaylikla
cozum sagliyor. Yerine gecirme (substitution) kuralini kullanarak cozume
erismek cok kolay. 

\[ y'' + Ay' + By = f(x) \]

Polinom operator kullanirsak

\[ (D^2 + AD + B )y = f(x) \]

Parantez icindekine $p(D)$ diyelim. Ve su formulu ortaya atalim. 

\[ p(D)e^{\alpha x} = p(\alpha)e^{\alpha x}  \]

Bu yerine gecirme kurali (substitution rule). 

Ispat

\[ (D^2 + AD + B )e^{\alpha x} \]

\[ = D^2e^{\alpha x} + ADe^{\alpha x} + Be^{\alpha x} \]

\[ = \alpha^2 e^{\alpha x} + A \alpha e^{\alpha x} + Be^{\alpha x} \]

\[ = e^{\alpha x}(\alpha^2  + A \alpha + B)\]

Parentez icinin $p(\alpha)$ oldugunu goruyoruz. 

\[ = e^{\alpha x}p(\alpha)\]

Simdi bunlari yeni bir teori icin kullanalim

Ustel Girdi Teorisi 

Bu teori Ustel Cevap Formulu (Exponential Response Formulu -ERF- olarak ta
biliniyor).

ODE

\[ y'' + Ay' + By = e^{\alpha x} \]

$e^{\alpha x}$, problemde $e^{(3+i)x}$ olabilir mesela, o zaman $\alpha
= (3+i)$ olacaktir. 

Operator formunda,

\[ p(D)y = e^{\alpha x} \]

icin ozel cozum sudur

\[ y_p = \frac{e^{\alpha x}}{p(\alpha)} \]

Bu teori bu dersin en onemli teorilerinden biri. Bu dersteki pek cok
kavrami yanyana getiriyor. 

Ispat

Ispatlamak icin cozum $y_p$'yi ana denkleme koyalim ve alttaki ifadenin dogru
olup olmayacagina bakalim. ODE'yi tekrar ifade edelim, 

\[ p(D)y =  e^{\alpha x} \]

ama $y$ yerine $y_p$ koyalim, ve bakalim ifadenin sol tarafini
donusturunce, sag taraftaki ayni sonuc cikacak mi?

\[ = p(D)y_p\]

\[ = p(D)\frac{e^{\alpha x}}{p(\alpha)}  \]

Ust taraf icin yerine gecirme kuralini kullanalim

\[ = \frac{p(\alpha)e^{\alpha x}}{p(\alpha)} \]

\[ = e^{\alpha x} \]

Sag tarafta ayni ifadeye eristik, demek ki teori dogru. Peki ya $p(\alpha)
= 0$ olsaydi? 
Bu onemli bir istisnai durum, ama problemde boyle olmadigini farzediyoruz.

Ornek 

\[ y'' - y' + 2y = 10e^{-x}sin(x) \]

Sag taraf gittikce yokolan (decaying) bir salinim. Genel cozumu bul.

Ozel cozumu bulalim ve sag tarafi kompleklestirelim. 

\[ (D^2 - D + 2)y = 10 e ^{(-1 + i)x} \]

Komplekslestirmeyi nasil yaptim? Dikkat edersek, $10e^{-x}sin(x)$ ifadesi
$10 e^{(-1 + i)x}$ ifadesinin kompleks kismini temsil ediyor. 

\[ e ^{(-1 + i)x} = e^{-x} e^{ix} \]

Euler acilimina gore $e^{ix} = \cos(x) + i\sin(x)$, sadece hayali kismi alirsak

\[ e^{-x} \sin(x)  \]

Basindaki 10'u da eklemeyi unutmuyoruz tabii. Bu teknigi daha once de
kullanmistik, ve kullandigimiz zaman orijinal ODE'deki degiskenin uzerine
bir dalga isareti koymustuk, cunku elimizde farkli bir ODE var, farkli
ODE'den aldigimiz cozumun kompleks kismini almak gerekecek. Yeni formul

\[ (D^2 - D + 2)\tilde{y} = 10 e ^{(-1 + i)x} \]

Ozel cozum ne? ERF'ten hareketle

\[ \tilde{y}_p = \frac{10 e^{(-1+i)x}}{(-1+i)^2 - (-1+i) + 2} \]

\[  = \frac{10 e^{(-1+i)x}}{3 - 3i} \]

\[ = \frac{10}{3}\frac{(1+i)}{2} e^{-x} \bigg( \cos(x) + i\sin(x) \bigg)\]

$y_p$, $\tilde{y}_p$'nin hayali kismi olacak

\[ \frac{5}{3}e^{-x}( \cos(x) + \sin(x)) \]

Bu son formdan hoslanmiyorsak, onu hemen cevirebiliriz, dik ucgeni
hatirlayalim, kenarlar 1 ve 1 ise hipotenusu $\sqrt{2}$

\[ = \frac{5}{3}e^{-x} \sqrt{2}\cos(x - \frac{\pi}{4})\]

Ya $p(\alpha) = 0$ Olursa?

Bu durumda ERF yerine yeni bir formul gerekecek. Bu yeni formul icin de
yeni bir Yerine Gecirme Kanunu gerekecek. 

Bu noktada $\alpha$ sembolu yerine $a$ sembolunu kullanacagiz, ama temsil
edilen sey hala kompleks bir sayi (not: kompleks olmasi garanti degil bu
arada, en azindan kompleks olmasina izin veriliyor). 

Ustel Kaydirma Kanunu (Exponential Shift Law)

Bu kanun alttaki ifadeyi basitlestirmekte kullaniliyor. Once bu ifadenin
icerdigi cebirsel zorlugu gorelim.

\[ p(D)e^{ax}u(x) \]

Eger bu formulun turevini alirsak, $u(x)$'in 2. hatta daha fazla
derecelerdeki turevini almamiz gerecek. Bundan kurtulmanin bir yolu yok mu?
Var. Ustel Kaydirma Kanununa gore eger $e^{ax}$ $p(D)$ uzerinden sol tarafa
gecerse, $p(D)$ degiserek $p(D+a)$ haline gelir. 

\[ p(D)e^{ax}u(x) = e^{ax}p(D+a)u(x)\]

Ispat

Ozel bir sarta bakalim, $p(D) = D$ olsun. O zaman 

\[ p(D)e^{ax}u(x) \]

su hale gelir

\[ = De^{ax}u\]

\[ = De^{ax}Du + ae^{ax}u \]

\[ = De^{ax}Du + ae^{ax}u \]

\[ = e^{ax}(D + a)u \]

ki son ifade kaydirma kanununu ile uyumlu. 

Peki $P(D) = D^2$ olsaydi? Bunun ispati icin ustteki islemlerin hepsini
tekrarlamaya gerek yok, mesela $D^2e^{ax}$ hesabi yapmaya gerek yok,
onceki hesabi kullanalim, $De^{ax}u$'u zaten biliyoruz

\[  D^2e^{ax}u = D(De^{ax}u) = D(e^{ax}(D+a)u)  \]

\[ = e^{ax}(D+a)[(D+a)u] = e^{ax}(D+a)^2u \]

Bu sonucun $D^3$, $D^4$, vs. icin genellestirilebilecegini gorebiliyoruz
herhalde. Matematiksel tumevarim (induction) ile $D^NN$ icin bu formul
ispatlanabilir. 

Devam edelim

\[ (D^2+AD+B)y = e^{ax} \]

$a$'nin kompleks olabilecegini unutmayalim, ama problemimiz su: $p(a) =
0$.
Simdi ozel cozumu nasil elde edecegim? 

\[ y_p = \frac{x e^{ax}}{p'(a)} \]

Peki ya $p'(a) = 0$  olursa? Bunun icin $a$'nin $p(D)$'nin basit bir koku
oldugunu farzedecegiz, cunku bu faraziye sayesinde $p'(a)$ hicbir zaman
sifir olamaz. 

Ya iki kokun ikisi de $a$ olsaydi? Elde ikiden fazla kok olamaz tabii,
cunku bir karesel denklemin ancak o kadar koku olabilir. O zaman ozel cozum
soyle olacakti

\[ y_p = \frac{x^2e^{ax}}{p''(a)} \]

Daha ust seviye polinomlar icin bu isin nereye gittigi belli oluyor
herhalde. 

Bunlardan birini ispatlayalim isterseniz. Sunu mesela

\[ y_p = \frac{x e^{ax}}{p'(a)} \]

Bu ispati ustel kaydirma kanunu kullanarak yapacagiz. 

Ispat

Basit kok durumu

\[ p(D) = (D-b)(D-a) \]

ve $b \ne a$ cunku iki tane farkli kok var. O zaman turevi alirsak

\[ p'(D) = (D-a) + (D-b) \]

\[ p'(a) = a-b \]

Onerdigimiz ozel cozum suydu

\[ \frac{P(D) e^{ax} x}{p'(a)} \]

Amacimiz bu cozum icine daha once elde ettigimiz sonuclari koyarsak,
ODE'nin sag tarafini, girdiyi elde etmek, boylece elimizde bir cozum
oldugunu anlayabilmis olacagiz. 

\[ = e^{ax}(D+a-b)\frac{Dx}{p'(a)} = e^{ax} \frac{(a-b) \cdot 1}{a-b} \]

\[ = e^{ax} \]

Ornek

\[ y'' - 3y + 2y = e^x \]

$e^x$ ile $e^{ax}$ formunu karsilastirirsak, $a$'nin 1 oldugunu goruruz,
yani $a$ reel bir sayi. Ayrica $a$, $D^2-3D+2$'nin basit bir koku, yerine
koyarsak $D^2-3D+2$'nin sifir verdigini gorurduk. 

\[ y_p = \frac{xe^x}{-1} \]

alttaki -1 nereden geldi? 

\[ p'(D) = 2D - 3 \]

\[ p'(1) = -1 \]

Yani nihai sonuc

\[ y_p = -xe^x \]


\end{document}

\documentclass[12pt,fleqn]{article}
\setlength{\parindent}{0pt}
\usepackage{graphicx}
\usepackage{listings}
\usepackage[latin5]{inputenc}
\setlength{\parskip}{8pt}
\setlength{\parsep}{0pt}
\setlength{\headsep}{0pt}
\setlength{\topskip}{0pt}
\setlength{\topmargin}{0pt}
\setlength{\topsep}{0pt}
\setlength{\partopsep}{0pt}
\setlength{\mathindent}{0cm}

\begin{document}
MIT OCW ODE - Ders 13

Bugunku dersimizin hedefi ozel cozumler bulmak. Formu yazalim

\[ y'' + Ay' + By = f(x) \]

Ve genel cozum $y = y_p + c_1y_1 + c_2y_2$ formunda olacak. 

Gercek su ki esitligin sag tarafina yazilabilecek her fonksiyon o kadar
ilginc degil. Ilginc olanlardan bir tanesi ustel (exponential)
fonksiyonlar, yani $e^{ax}$ formundaki fonksiyonlar, ki cogunlukla $a<0$
kullanilir. Diger bazi ilginc olanlar $cos(x)$, $sin(x)$ gibi salinim
ornekleri, ki bunlar da elektriksel devrem baglaminda alternatif AC/DC
akimi temsil ediyorlar.

Ya da ``gittikce yokolan salinim'' ilginc, Burada 

\[ e^{ax}sin \omega x \]

\[ e^{ax}cos \omega x \]

gibi ornekler var. 


\end{document}

\documentclass[12pt,fleqn]{article}
\setlength{\parindent}{0pt}
\usepackage{graphicx}
\usepackage{listings}
\usepackage[latin5]{inputenc}
\setlength{\parskip}{8pt}
\setlength{\parsep}{0pt}
\setlength{\headsep}{0pt}
\setlength{\topskip}{0pt}
\setlength{\topmargin}{0pt}
\setlength{\topsep}{0pt}
\setlength{\partopsep}{0pt}
\setlength{\mathindent}{0cm}

\begin{document}
MIT OCW ODE - Ders 13

Bugunku dersimizin hedefi ozel cozumler bulmak. Formu yazalim

\[ y'' + Ay' + By = f(x) \]

Ve genel cozum $y = y_p + c_1y_1 + c_2y_2$ formunda olacak. 

Gercek su ki esitligin sag tarafina yazilabilecek her fonksiyon o kadar
ilginc degil. Ilginc olanlardan bir tanesi ustel (exponential)
fonksiyonlar, yani $e^{ax}$ formundaki fonksiyonlar, ki cogunlukla $a<0$
kullanilir. Diger bazi ilginc olanlar 

\[ sin \omega x \]

\[ cos \omega x \]

gibi salinim
ornekleri, ki bunlar da elektriksel devrem baglaminda alternatif AC/DC
akimi temsil ediyorlar.

Ya da ``gittikce yokolan salinim'' ilginc, Burada 

\[ e^{ax}sin \omega x \]

\[ e^{ax}cos \omega x \]

gibi ornekler var. Ama aslinda ustteki tum ilginc fonksiyonlar genel tek
bir forma baglanabilir, bu form icin ustel sayinin kompleks olmasina izin
vermek gerekiyor. Form soyle

\[ e^{(a+i\omega)x} \]

$\omega = 0$ ise o zaman $e^{ax}$ elde ederim. $a=0$ ise $sin\omega x$,
$cos\omega x$ elde ederim. Ikisi de sifir degilse o zaman gittikce yokolan
salimi elde ederim. 

Bundan sonra habire $a+i\omega$ yazmamak icin onun yerine $\alpha$
kullanacagiz, $\alpha$'yi gorunce onun bir kompleks sayi oldugunu
anlayin. Yani esitligin sag tarafi $e^{\alpha x}$ olacak. 

Birazdan gorecegimiz uzere, bu tur bir girdi kullanmak aslinda kolaylikla
cozum sagliyor. Yerine gecirme (substitution) kuralini kullanarak cozume
erismek cok kolay. 

\[ y'' + Ay' + By = f(x) \]

Polinom operator kullanirsak

\[ (D^2 + AD + B )y = f(x) \]

Parantez icindekine $p(D)$ diyelim. Ve su formulu ortaya atalim. 

\[ p(D)e^{\alpha x} = p(\alpha)e^{\alpha x}  \]

Bu yerine gecirme kurali (substitution rule). 

Ispat

\[ (D^2 + AD + B )e^{\alpha x} \]

\[ = D^2e^{\alpha x} + ADe^{\alpha x} + Be^{\alpha x} \]

\[ = \alpha^2 e^{\alpha x} + A \alpha e^{\alpha x} + Be^{\alpha x} \]

\[ = e^{\alpha x}(\alpha^2  + A \alpha + B)\]

Parentez icinin $p(\alpha)$ oldugunu goruyoruz. 

\[ = e^{\alpha x}p(\alpha)\]

Simdi bunlari yeni bir teori icin kullanalim

Ustel Girdi Teorisi 

Bu teori Ustel Cevap Formulu (Exponential Response Formulu -ERF- olarak ta
biliniyor).

ODE

\[ y'' + Ay' + By = e^{\alpha x} \]

yani

\[ p(D)y = e^{\alpha x} \]

icin ozel cozum sudur

\[ y_p = \frac{e^{\alpha x}}{p(\alpha)} \]

Bu teori bu dersin en onemli teorilerinden biri. Bu dersteki pek cok
kavrami yanyana getiriyor. 

Ispat

Ispatlamak icin cozum $y_p$'yi ana denkleme koyalim ve alttaki ifadenin dogru
olup olmayacagina bakalim. ODE'yi tekrar ifade edelim, 

\[ p(D)y =  e^{\alpha x} \]

ama $y$ yerine $y_p$ koyalim, ve bakalim ifadenin sol tarafini
donusturunce, sag taraftaki ayni sonuc cikacak mi?

\[ = p(D)y_p\]

\[ = p(D)\frac{e^{\alpha x}}{p(\alpha)}  \]

Ust taraf icin yerine gecirme kuralini kullanalim

\[ = \frac{p(\alpha)e^{\alpha x}}{p(\alpha)} \]

\[ = e^{\alpha x} \]

Sag tarafta ayni ifadeye eristik, demek ki teori dogru. Peki ya $p(\alpha)
= 0$ olsaydi? 
Bu onemli bir istisnai durum, ama problemde boyle olmadigini farzediyoruz.

Ornek 

\[ y'' - y' + 2y = 10e^{-x}sin(x) \]

Sag taraf gittikce yokolan (decaying) bir salinim. Genel cozumu bul.

Ozel cozumu bulalim ve sag tarafi kompleklestirelim. 

\[ (D^2 - D + 2)y = 10 e ^{(-1 + i)x} \]

Komlekslestirmeyi nasil yaptim? Dikkat edersek, $10e^{-x}sin(x)$ ifadesi
$10 e^{(-1 + i)x}$ ifadesinin kompleks kismini temsil ediyor. 

\[ e ^{(-1 + i)x} = e^{-x} e^{ix} \]

Euler acilimina gore $e^{ix} = cos(x) + isin(x)$, sadece hayali kismi alirsak

\[ e^{-x} sin(x)  \]

Basindaki 10'u da eklemeyi unutmuyoruz tabii. Bu teknigi daha once de
kullanmistik, ve kullandigimiz zaman orijinal ODE'deki degiskenin uzerine
bi dalga isareti koymustuk, cunku elimizde farkli bir ODE var, farkli
ODE'den aldigimiz cozumun kompleks kismini almak gerekecek. Yeni formul

\[ (D^2 - D + 2)\tilde{y} = 10 e ^{(-1 + i)x} \]

Ozel cozum ne? ERF'ten hareketle

\[ \tilde{y}_p = \frac{10 e^{(-1+i)x}}{(-1+i)^2 - (-1+i) + 2} \]

\[  = \frac{10 e^{(-1+i)x}}{3 - 3i} \]

\[ = \frac{10}{3}\frac{(1+i)}{2} e^{-x} \bigg( cos(x) + isin(x) \bigg)\]

$y_p$, $\tilde{y}_p$'nin hayali kismi olacak

\[ \frac{5}{3}e^{-x}( cos(x) + sin(x)) \]

Bu son formdan hoslanmiyorsak, onu hemen cevirebiliriz, dik ucgeni
hatirlayalim, kenarlar 1 ve 1 ise hipotenusu $\sqrt{2}$

\[ = \frac{5}{3}e^{-x} \sqrt{2}cos(x - \frac{\pi}{4})\]


\end{document}

\documentclass[12pt,fleqn]{article}\usepackage{../common}
\begin{document}
Veri Analizi - Ders 1

Gaussian Kontrolu

Diyelim ki Gaussian dagilimina sahip oldugunu dusundugumuz $\{ x_i\}$
verilerimiz var. Bu verilerin Gaussian dagilimina uyup uymadigini nasil
kontrol edecegiz? Normal bir dagilimin her veri noktasi icin soyle temsil
edebiliriz,

\[ y_i = \Phi\bigg(\frac{ x_i - \mu}{\sigma}\bigg) \]

Burada $\Phi$ standart Gaussian'i temsil ediyor (detaylar icin {\em Istatistik
Ders 1}). Simdi bir numara yapalim, iki tarafa ters Gaussian formulunu
uygulayalim, yani $\Phi ^{-1} $. 

\[ \Phi ^{-1}(y_i) = \Phi ^{-1}\bigg(\Phi\bigg(\frac{ x_i - \mu}{\sigma}\bigg)\bigg) \]

\[ \Phi ^{-1}(y_i) = \frac{ x_i - \mu}{\sigma}\]

\[  x_i = \Phi^{-1}(y_i) \sigma + \mu  \]

Bu demektir ki elimizdeki verileri $\Phi^{-1}(y_i)$ bazinda grafiklersek,
bu noktalar egimi $\sigma$, baslangici (intercept) $\mu$ olan bir duz cizgi
olmalidir. Eger kabaca noktalar duz cizgi olusturmuyorsa, verimizin 
Gaussian dagilima sahip olmadigina karar verebiliriz. 

Ustte tarif edilen grafik,  olasilik grafigi (probability plot) olarak
bilinir. 

Ters Gaussian teorik fonksiyonunu burada vermeyecegiz, Scipy lkasjdlkfasf
\verb!scipy.stats.invgauss! hesaplar icin kullanilabilir.



\end{document}

\documentclass[12pt,fleqn]{article}\usepackage{../common}
\begin{document}
Veri Analizi - Ders 1

Gaussian Kontrolu

Diyelim ki Gaussian dagilimina sahip oldugunu dusundugumuz $\{ x_i\}$
verilerimiz var. Bu verilerin Gaussian dagilimina uyup uymadigini nasil
kontrol edecegiz? Normal bir dagilimin her veri noktasi icin soyle temsil
edebiliriz,

\[ y_i = \Phi\bigg(\frac{ x_i - \mu}{\sigma}\bigg) \]

Burada $\Phi$ standart Gaussian'i temsil ediyor (detaylar icin {\em
  Istatistik Ders 1}) ve CDF fonksiyonuna tekabul ediyor. CDF fonksiyonunun
ayni zamanda ceyregi (quantile) hesapladigi soylenir, aslinda CDF son
derece detayli bir olasilik degeri verir fakat evet, dolayli yoldan
noktanin hangi ceyrek icine dustugu de gorulecektir.

Simdi bir numara yapalim, iki tarafa ters Gaussian formulunu uygulayalim,
yani $\Phi ^{-1} $.

\[ \Phi ^{-1}(y_i) = \Phi ^{-1}\bigg(\Phi\bigg(\frac{ x_i - \mu}{\sigma}\bigg)\bigg) \]

\[ \Phi ^{-1}(y_i) = \frac{ x_i - \mu}{\sigma}\]

\[  x_i = \Phi^{-1}(y_i) \sigma + \mu  \]

Bu demektir ki elimizdeki verileri $\Phi^{-1}(y_i)$ bazinda grafiklersek,
bu noktalar egimi $\sigma$, baslangici (intercept) $\mu$ olan bir duz cizgi
olmalidir. Eger kabaca noktalar duz cizgi olusturmuyorsa, verimizin 
Gaussian dagilima sahip olmadigina karar verebiliriz. 

Ustte tarif edilen grafik,  olasilik grafigi (probability plot) olarak
bilinir. 

Ters Gaussian teorik fonksiyonunu burada vermeyecegiz, Scipy
\verb!scipy.stats.invgauss! hesaplar icin kullanilabilir. Fakat $y_i$'nin
kendisi nereden geliyor? Eger $y_i$, CDF'in bir sonucu ise, pur veriye
bakarak bir CDF degeri de hesaplayabilmemiz gerekir. Bunu yapmak icin bir
baska numara lazim. 

1. Eldeki sayilari artan sekilde siralayin

2. Her veri noktasina bir derece (rank) atayin (siralama sonrasi hangi
seviyede oldugu yeterli, 1'den baslayarak). 

3. Ceyrek degeri $y_i$ bu sira / $n+1$, $n$ eldeki verinin buyuklugu. 

Bu teknik niye isliyor? $x$'in CDF'i $x_i < x$ sartina uyan $x_i$'lerin
orani degil midir? Yani bir siralama soz konusu ve ustteki teknik te bu
siralamayi biz elle yapmis olduk, ve bu siralamadan gereken bilgiyi aldik. 









\end{document}

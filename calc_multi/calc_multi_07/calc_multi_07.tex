\documentclass[12pt,fleqn]{article}
\setlength{\parindent}{0pt}
\usepackage{graphicx}
\usepackage{listings}
\usepackage[latin5]{inputenc}
\setlength{\parskip}{8pt}
\setlength{\parsep}{0pt}
\setlength{\headsep}{0pt}
\setlength{\topskip}{0pt}
\setlength{\topmargin}{0pt}
\setlength{\topsep}{0pt}
\setlength{\partopsep}{0pt}
\setlength{\mathindent}{0cm}

\begin{document}
MIT OCW Cok Degiskenli Calculus - Ders 7

Bugunun konusu ``hersey''. Simdiye kadar gordugumuz her sey
yani. Vektorleri gorduk, noktasal carpimlari (dot product) gorduk. Iki
vektorun noktasal carpimi

\[ \vec{A} \cdot \vec{B} = \sum a_ib_i\]

yani o vektorlerin tum elemanlarinin sirasiyla birbiriyle carpilip
toplanmasi. O da suna esit

\[ = |\vec{A}||\vec{B}|cos \ \theta \]

Noktasal carpimi acilari olcmek icin kullanabiliriz, eger $cos \ \theta$
terimini tek basina birakirsak, geri kalanlari cozeriz. Bu sekilde iki
vektorun dik olup olmadigini da anlariz. Cunku o zaman sonuc sifir olur, ve
$cos \ \theta = 0$ ise aci dik demektir. 




\end{document}

\documentclass[12pt,fleqn]{article}
\setlength{\parindent}{0pt}
\usepackage{graphicx}
\usepackage{listings}
\usepackage[latin5]{inputenc}
\setlength{\parskip}{8pt}
\setlength{\parsep}{0pt}
\setlength{\headsep}{0pt}
\setlength{\topskip}{0pt}
\setlength{\topmargin}{0pt}
\setlength{\topsep}{0pt}
\setlength{\partopsep}{0pt}
\setlength{\mathindent}{0cm}

\begin{document}
MIT OCW Cok Degiskenli Calculus - Ders 14

Bagimsiz Olmayan Degiskenler (Non-independent Variables)

Ornek

Fizikteki $f(P,V,T)$ formulu, ki bu degiskenler 

\[ PV = nRT \]

seklinde ilintili. Daha genel olarak bir $f(x,y,z)$ formulu var, ve
degiskenler $x,y,z$ birbiriyle $g(x,y,z) = c$ uzerinden baglantili. Aslinda
bir onceki dersteki ayni durum, sadece bu sefer min, maks degil, kismi
turevlere neler oldugunu inceleyecegiz. 

Yine onceki dersteki gibi, belki $g$'yi cebirsel olarak degistirip, $f$'e
sokup degisken yoketmek mumkun degil. Eger oyle yapabilsek, bir $z =
z(x,y)$ 
olabilirdi, ve onun kismi turevlerine bakabilirdik,

\[ \frac{\partial z}{\partial x}, \frac{\partial z}{\partial y}, .. \]

gibi. Peki ya $z$'yi bulamiyorsak? Belki ustteki kismi turevleri $z$'yi
bulmadan elde edebiliriz. 

Ornek

\[ x^2 + yz + z^3 = 8 \]

$(2,3,1)$ noktasina bakalim (yerine koyunca hakikaten 8 ciktigini
goruyoruz). Fakat bu degerlerde azicik degisiklik yapinca, $z$ nasil
degisir? Bu soruyu nasil cevaplarim? 

Formulden $z$'yi cekip cikarmak gerekir, kupsel (cubic) formullerde bunu
yapmanin bir yolu var, fakat cok karmasik bir formul ortaya
cikartiyor. Aradigimiz sonuca ulasmanin daha kolay bir yolu var. 

$g$'nin tam diferansiyeline, yani $dg$'ye bakalim (ustteki formulu $g$
kabul ediyoruz). Tam diferansiyel

\[ 2x dx + z dy + (y+3z^2) dz = 0\]

Sag taraf sifir cunku ustteki $g$ bir sabite esit, $g=8$, sabitin degisimi
sifir, yani $dg=0$. 

Tam diferansiyele $(2,3,1)$ degerini verelim

\[ 4dx + dy + 6dz = 0 \]

Bu formul bize her degiskenin degisiminin digeri ile nasil baglantili
oldugunu gosteriyor. Mesela $dx$ ve $dy$'yi biliyorsak, $dz$'yi, yani
$z$'nin degisimini hesaplayabiliriz. Yani $z=z(x,y)$ uzerinden 

\[ dz = -\frac{1}{6}(4dx + dy) \]

Bu formul bize kismi turevleri de gostermis oluyor aslinda, cunku tam
diferansiyel formulunde kismi turevler vardir, ustteki formulde $dx,dy$'nin
yaninda yer alan degerler onlardir. O zaman

\[ \frac{\partial z}{\partial x} = -\frac{4}{6} = -\frac{2}{3} \]

\[ \frac{\partial z}{\partial y} = -\frac{1}{6} \]

Bunu dusunmenin bir diger yolu su. $\partial z/\partial x$ $z$'nin $x$'e
gore degisimi ise, $y$ sabit demektir, ustteki $dz$ formulunde $dy=0$
deriz, geri kalanlar

\[ dz = -\frac{2}{3}dx \]

ki bu formul $z$'nin $x$'teki degisime gore nasil degistigini gosteriyor. 









\end{document}

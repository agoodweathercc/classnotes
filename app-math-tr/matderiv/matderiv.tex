\documentclass[12pt,fleqn]{article}\usepackage{../common}
\begin{document}
Matrislerin, Vekt�rlerin T�revleri

Gradyan

$d$ boyutlu vektor $x$'i alan ve geriye tek sayi sonucu donduren bir $f$
fonksiyonunun $x$'e gore turevini nasil aliriz? Bu durumda her $x$
elemanina gore kismi turevler (partial derivatives) alinir, ve sonuc
bir $d$ boyutlu vektore yerlestirilir,

$$
\frac{\partial f}{\partial x}  =
\left[\begin{array}{c}
\frac{\partial f}{\partial x_1} \\
\\
\frac{\partial f}{\partial x_2} \\
\vdots \\
\frac{\partial f}{\partial x_d} 
\end{array}\right]
$$

Bu sonuc tanidik gelmis olabilir, sonuc gradyan olarak ta biliniyor. Yani

$$ \frac{\partial f}{\partial x}  = \nabla f = grad \ f(x) $$

Tek Parametreye Gore Matris Turevi

Eger bir $M$ matrisinin tum ogeleri bir $\theta$ parametresine bagli ise, o
matrisin $\theta$'ya gore turevi icin tum elemanlarinin teker teker
$\theta$'ya gore turevleri alinir,

$$ 
\frac{\partial M}{\partial \theta} = 
\left[\begin{array}{cccc}
\frac{\partial m_{11}}{\partial \theta} & 
\frac{\partial m_{12}}{\partial \theta} & \dots & 
\frac{\partial m_{1d}}{\partial \theta} \\

\frac{\partial m_{21}}{\partial \theta} & 
\frac{\partial m_{22}}{\partial \theta} &  \dots & 
\frac{\partial m_{2d}}{\partial \theta}  \\

\vdots & \vdots & \ddots & \vdots \\

\frac{\partial m_{n1}}{\partial \theta} & 
\frac{\partial m_{n2}}{\partial \theta} &  \dots & 
\frac{\partial m_{nd}}{\partial \theta}  

\end{array}\right]
 $$

Matris ve Vektor Turevleri

Eger bir $x$ vektorunden bagimsiz bir $M$ matrisi o $x$ ile carpiliyor ise,
bunlarin $x$'e gore turevi nedir? 

$$ \frac{\partial}{\partial x} [Mx] = M
$$

Ustteki sonuc aslinda tek sayili / boyutlu ortamda $2x$ gibi bir ifadenin
$x$'e gore turevini alinca $2$ elde etmeye esdeger. 




\end{document}

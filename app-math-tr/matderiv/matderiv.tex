\documentclass[12pt,fleqn]{article}\usepackage{../common}
\begin{document}
Matris T�revleri

Gradyan

$m$ boyutlu vektor $x$'i alan ve geriye tek sayi sonucu donduren bir $f(x)$
fonksiyonunun $x$'e gore turevini nasil aliriz? Yani $x \in \mathbb{R}^m$
ve bir vektor,

$$ x = 
\left[\begin{array}{ccc}
x_1 \\ \vdots \\ x_m
\end{array}\right]
 $$

Bu durumda $x$'in her hucresine / ogesine gore kismi turevler (partial
derivatives) alinir, sonucta tek boyutlu / tekil sayili fonksiyon, turev 
sonrasi $m$ boyutlu bir sonuc vektorunu yaratir, yani

$$
\frac{\partial f}{\partial x}  =
\left[\begin{array}{c}
\frac{\partial f}{\partial x_1} \\
\\
\frac{\partial f}{\partial x_2} \\
\vdots \\
\frac{\partial f}{\partial x_m} 
\end{array}\right]
$$

Bu sonuc tanidik gelmis olabilir, bu ifade gradyan olarak ta bilinir.

$$ \frac{\partial f}{\partial x}  = \nabla f = grad \ f(x) $$

Elde edilen vektor surpriz degil cunku tek, skalar bir deger veren bir
fonksiyonun $x$ icindeki {\em her ogensinin} nasil degistigine gore bunun
fonksiyon uzerindeki etkilerini merak ediyorduk, ustteki vektor oge bazinda
bize aynen bunu gosteriyor. Yani tek skalar sonuc $m$ tane turev sonucuna
ayriliyor, cunku tek sonucun $m$ tane secenege gore degisimini gormek
istedik. Not olarak belirtelim, gradyan vektoru matematiksel bir
rahatliktir, bir kisayoldur, bir ziplama noktasidir, yani matematiksel olarak
turetilerek ulasilan ana kurallardan biri denemez. Fakat cok ise yaradigina
suphe yok.

Tek Parametreye Gore Matris Turevi

Eger bir $A$ matrisinin tum ogeleri bir $\theta$ parametresine bagli ise, o
matrisin $\theta$'ya gore turevi icin tum elemanlarinin teker teker
$\theta$'ya gore turevleri alinir,

$$ 
\frac{\partial A}{\partial \theta} = 
\left[\begin{array}{cccc}
\frac{\partial a_{11}}{\partial \theta} & 
\frac{\partial a_{12}}{\partial \theta} & \dots & 
\frac{\partial a_{1n}}{\partial \theta} \\

\frac{\partial a_{21}}{\partial \theta} & 
\frac{\partial a_{22}}{\partial \theta} &  \dots & 
\frac{\partial a_{2n}}{\partial \theta}  \\

\vdots & \vdots & \ddots & \vdots \\

\frac{\partial a_{m1}}{\partial \theta} & 
\frac{\partial a_{m2}}{\partial \theta} &  \dots & 
\frac{\partial a_{mn}}{\partial \theta}  

\end{array}\right]
$$

Cok Parametreli Matris Turevi

Simdi ilginc bir varyasyon; diyelim ki hem fonksiyon $f(x)$'e verilen $x$
cok boyutlu, hem de fonksiyonun sonucu cok boyutlu! Bu gayet mumkun bir
durum. Bu durumda ne olurdu? 

Eger kismi turevlerin her turlu oge degisimini temsil etmesini istiyorsak,
o zaman hem her girdi hucresi, hem de her cikti hucresi icin bu degisimi
saptamaliyiz. Jacobian matrisleri tam da bunu yapar. Eger $m$ boyutlu girdi
ve $n$ boyutlu cikti tanimlayan $f$'in turevini almak istersek,

$$ 
J(x) = \frac{\partial f(x)}{\partial x} =
\left[\begin{array}{ccc}
\frac{\partial f_{1}(x)}{\partial \theta} & \dots & 
\frac{\partial f_{1}(x)}{\partial \theta} \\

\vdots & \ddots & \vdots \\

\frac{\partial f_{n}(x)}{\partial \theta} & \dots & 
\frac{\partial f_{n}(x)}{\partial \theta}  
\end{array}\right]
 $$

Vektor Turevleri

Eger bir $x \in \mathbb{R}^m$ vektorunden bagimsiz bir $A$ matrisi o $x$ ile carpiliyor ise,
bunlarin $x$'e gore turevi nedir? 

$$ \frac{\partial}{\partial x^T} [Ax] = A
$$

Ustteki sonuc aslinda tek sayili / boyutlu ortamda $2x$ gibi bir ifadenin
$x$'e gore turevini alinca $2$ elde etmeye esdeger. Ispat icin soyle
dusunelim, eger $a_i \in \mathbb{R}^n$ ise (ki devrigi alininca bu vektor
yatay hale gelir, yani altta bu yatay vektorleri ust uste istifliyoruz), 

$$ A = \left[\begin{array}{c}
a_1^T \\ \vdots \\ a_m^T
\end{array}\right] $$

Bu durumda $Ax$ ne olur? {\em Matris Carpimi} yazisindaki satir bakis acisi
dusunulurse, $A$'in bir satirinin her ogesi $x$'in tum satirlarini (burada
$x$ vektor oldugu icin her satir tek bir sayidan ibaret) kombine ederek o
sonuc satirini olusturmaktadir, o zaman

$$ A = \left[\begin{array}{c}
a_1^Tx_1 \\ \vdots \\ a_m^Tx_m
\end{array}\right] $$


Kaynaklar

Duda, Hart, {\em Pattern Classification}




\end{document}

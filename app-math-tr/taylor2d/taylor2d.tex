\documentclass[12pt,fleqn]{article}
\setlength{\parindent}{0pt}
\usepackage{graphicx}
\usepackage{cancel}
\usepackage{listings}
\usepackage[latin5]{inputenc}
\setlength{\parskip}{8pt}
\setlength{\parsep}{0pt}
\setlength{\headsep}{0pt}
\setlength{\topskip}{0pt}
\setlength{\topmargin}{0pt}
\setlength{\topsep}{0pt}
\setlength{\partopsep}{0pt}
\setlength{\mathindent}{0cm}

\begin{document}
Iki Boyutlu f(x,y) Fonksiyonunun Taylor Acilimi

Bir $f(x,y)$ fonksiyonunun Taylor acilimini yapmak icin, daha onceden
gordugumuz tek boyutlu fonksiyon acilimindan faydalanabiliriz. 

Once iki boyutlu fonksiyonu tek boyutlu olarak gostermek gerekir. Tek
boyutta isleyen bir fonksiyon $F$ dusunelim ve bu $F$, arka planda iki
boyutlu $f(x,y)$'i kullaniyor olsun

Eger 

\[ f(x_0 +\Delta x, y_o + \Delta y) \]

fonksiyonun acilimini elde etmek istiyorsak, onu

\[ F(t) = f(x_0 + t\Delta x, y_o + t\Delta y) \]

uzerinden $t=1$ oldugu durumda hayal edebiliriz. $x,y$ parametrize 
oldugu icin  $f(x(t),y(t))$, yani

\[ x(t) = x_0 + t\Delta x \]

\[ y(t) = y_0 + t\Delta y \]

$F(t)$ baglaminda $x_o, y_o, \Delta x, \Delta y$ sabit olarak kabul edilecekler. Simdi bildigimiz
tek boyutlu Taylor acilimini bu fonksiyon uzerinde, bir $t_0$ noktasi yakininda 
yaparsak,

\[ F(t) = F(t_0) + F'(t_0)(t-t_0) + \frac{1}{2}F''(t_0)(t-t_0)^2 + ... \]

Eger $t=1,t_0=0$ dersek

\[ F(1) = F(0) + F'(0) + \frac{1}{2}F''(0) + ... \]

olurdu. Bu iki degeri, yani $t=1,t_0=0$'i kullanmamizin sebepleri $t=1$ ile
mesela $x_0 + t\Delta x$'in $x_0 + \Delta x$ olmasi, diger yandan $t=0$ ile
ustteki formulde $t$'nin yokolmasi, basit bir tek boyutlu acilim elde
etmek.

Simdi bize gereken $F',F''$ ifadelerini $x,y$ baglaminda elde edelim, ki bu
diferansiyeller $F$'in $t$'ye gore birinci ve ikinci diferansiyelleri. Ama
$F$'in icinde $x,y$ oldugu icin acilimin Zincirleme Kanunu ile yapilmasi
lazim.

\[ \frac{dF}{dt} = \frac{\partial F}{\partial x}\frac{dx(t)}{dt} +
\frac{\partial F}{\partial y}\frac{dy(t)}{dt} 
 \]

Ayrica

\[ \frac{d}{dt}x(t) = \Delta x \]

\[ \frac{d}{dt}y(t) = \Delta y \]

olduguna gore, tam diferansiyel daha da basitlesir

\[ \frac{dF}{dt} = \frac{\partial F}{\partial x}\Delta x +
\frac{\partial F}{\partial y}\Delta y
 \]

Simdi bu ifadenin bir tam diferansiyelini alacagiz. Ama ondan once sunu
anlayalim ki ustteki ifade icinde mesela birinci terim de aslinda bir
fonksiyon, ve asil hali

\[ \frac{dF}{dt} = \frac{\partial F(x(t),y(t))}{\partial x}\Delta x + ...
 \]

seklinde. O zaman, bu terim uzerinde tam diferansiyel islemini bir daha
uyguladigimizda, Zincirleme Kanunu yine isleyecek, mesela ustte $dx(t)/dt$'nin 
bir daha disari cikmasi gerekecek. O zaman 

\[ \frac{d^2F}{dt} =
\bigg(
\frac{\partial ^2 F}{\partial x^2}\frac{dx}{dt} + 
\frac{\partial ^2 F}{\partial x \partial y}\frac{dy}{dt} + 
\bigg) \Delta x +
\bigg(
\frac{\partial ^2 F}{\partial y \partial x}\frac{dy}{dt} + 
\frac{\partial ^2 F}{\partial y^2}\frac{dx}{dt} + 
\bigg) \Delta y 
\]

\[ =
\bigg(
\frac{\partial ^2 F}{\partial x^2}\Delta x + 
\frac{\partial ^2 F}{\partial x \partial y}\Delta y 
\bigg) \Delta x +
\bigg(
\frac{\partial ^2 F}{\partial y \partial x}\Delta x + 
\frac{\partial ^2 F}{\partial y^2}\Delta y 
\bigg) \Delta y 
 \]

\[ =
\bigg(
\frac{\partial ^2 F}{\partial x^2}\Delta x^2 + 
\frac{\partial ^2 F}{\partial x \partial y}\Delta y \Delta x
\bigg) +
\bigg(
\frac{\partial ^2 F}{\partial y \partial x}\Delta x \Delta y + 
\frac{\partial ^2 F}{\partial y^2}\Delta y^2
\bigg) 
 \]

Calculus'tan biliyoruz ki 

\[ 
\frac{\partial ^2 F}{\partial x \partial y} = 
\frac{\partial ^2 F}{\partial y \partial x} 
 \]

Daha kisa notasyonla

\[ f_{xy} = f_{yx} \]

Yani kismi turevin alinma sirasi farketmiyor. O zaman, ve her seyi daha
kisa notasyonla bir daha yazarsak

\[ =
(f_{xx}\Delta x^2 + f_{xy}\Delta y \Delta x ) +
(f_{xy}\Delta x \Delta y + f_{yy}\Delta y^2 )
 \]

\[
\frac{d^2F}{dt}  =
(f_{xx}\Delta x^2 + 2f_{xy}\Delta y \Delta x + f_{yy}\Delta y^2 )
 \]

Artik elimizde $F$ ve $F'$ var, bunlari 

\[ F(1) = F(0) + F'(0) + \frac{1}{2}F''(0) + ... \]

icine yerlestirebiliriz. En son su kaldi, $F(0)$ nedir? $F$'in $t=0$ oldugu
anda degeridir, 

\[ F(t) = f(x_0 + t\Delta x, y_o + t\Delta y) \]

\[ F(0) = f(x_0 + 0 \cdot \Delta x, y_o + 0 \cdot \Delta y) \]

\[ = f(x_0 , y_o) \]

Benzer sekilde, tum turevler de $t=0$ noktasinda kullanilacaktir, o zaman
onlar da

\[ F'(0) = f_x(x_0,y_0) \Delta x + f_y(x_0,y_0) \Delta y \]

\[ F''(0) =  
f_{xx}(x_0,y_0)\Delta x^2 + 2f_{xy}(x_0,y_0)\Delta y \Delta x + 
f_{yy}(x_0,y_0)\Delta y^2 
\]

seklinde olurlar. Tamam. Simdi ana formulde yerlerine koyalim,

\[ 
\begin{array}{lll}
F(1) &=& f(x_0 , y_o) +  \\ \\ 
&& f_x(x_0,y_0) \Delta x + f_y(x_0,y_0) \Delta y +   \\ \\
&& \frac{1}{2} 
[ 
f_{xx}(x_0,y_0)\Delta x^2 + 
2f_{xy}(x_0,y_0)\Delta y \Delta x +
f_{yy}(x_0,y_0)\Delta y^2 
] + ... 
\end{array}
 \]


Ve

\[ F(1) = f(x_0 +\Delta x, y_o + \Delta y) \]

olduguna gore, Taylor 2D acilimimiz tamamlanmis demektir. 

Kaynaklar 

http://www.math.ubc.ca/~feldman/m200/taylor2dSlides.pdf

http://math.uc.edu/~halpern/Calc.4/Handouts/Taylorseries.pdf



\end{document}

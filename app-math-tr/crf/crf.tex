\documentclass[12pt,fleqn]{article}\usepackage{../common}
\begin{document}
\textbf{Log Lineer Modeller ve Kosulsal Rasgele Alanlar (Log Linear Models and
Conditional Random Fields)}

Ders 2

Charles Elkan ders notlari

Kosulsal Olurluk (Conditional Likelihood)

Diyelim ki elimizde egitim verisi olarak ikili $<x,y>$ veri noktalari
var. O zaman $y$'nin $x$'e kosulsal olarak bagli (conditional on) bir
dagilimi oldugunu soyleyebiliriz. 

$$ y \sim f(x;\theta) $$

Yani her $x$ icin farkli bir $y$ dagilimi ortaya cikabilir. Ve tum bu
farkli dagilimlarin ortak noktasi $\theta$ parametresidir. Kosulsal
olasilik yani soyle yazilabilir, 

$$ P(Y=y | X=x;\theta) $$

Usttekiler $Y$ icin bir model ortaya koydu, peki elimizde $X$'in dagilimi
icin bir olasilik modelimiz var mi? Cevap hayir. Niye? Dusunelim, $p(y,x)$
nedir ?

$$ p(x,y) = p(x)p(y|x) $$

Ustte $p(y|x)$'i tanimlayacak ($\theta$ uzerinden) bir olasilik demeti /
ailesi tanimladik, fakat elimizde $p(x)$ dagilimini verecek bir model
yok, o zaman $p(x,y)$'yi tanimlayacak bir model de yok.

Fakat bu dunyanin sonu degil. Belki de Makine Ogrenimi bransinin bir
slogani su olmali: ``Ogrenmen gerekmeyen seyi ogrenme''. Ustteki ornekte
$p(y|x)$'i ogrenebiliriz, ama $p(x)$'i illa ogrenmemiz gerekir mi?

Siniflayici (classifier) ve takip edilen (supervised) ogrenim durumunu
dusunursek, bize egitim amacli olarak $<x,y>$ ikili veri noktalari
saglanacak. $x$ kaynak veri, $y$ tahmin edilecek (ya da basta egitim hedefi
olan) etiket olacak. $y$ icin bir model ortaya cikartiyoruz, cunku test
zamaninda $y$ olmayacak, fakat $x$ hep olacak. Yani $y$'nin modellenmesi
mecburi, cunku ``genelleyerek'' onun ne oldugunu bulacagiz, ama $x$ hep
verili.

Kosulsal Olurluk Maksimum Olurluk Prensibi

Egitim verisi $<x_1,y_1>,...,<x_n,y_n>$ icin, $\theta$'yi soyle sec

$$ \hat{\theta} =  \arg\max_{\theta} \prod _{i=1}^{n} p(y_i | x_i;\theta) $$

Normal maksimum olurlukta bilindigi gibi olasiliklarin carpimi maksimize
edilir, burada maksimize ettigimiz ``kosulsal'' olasiliklarin carpimi. 

Burada onemli bir soru su: bildigimiz gibi maksimum olurluk hesabi her veri
noktasinin bir digerinden bagimsiz oldugunu farzeder [cunku her olurluk
hesabini bir diger ile carpiyoruz, baska ek carpim, toplama, vs
yapmiyoruz], bu faraziye dogru bir faraziye midir? Bu soru ve ona verilecek
cevap cok onemli. Evet, eger egitim noktalari birbirinden bagimsiz degilse
maksimum olurluk kullanmamaliyiz. Bagimsizligi da iyi tanimlamak gerekiyor
tabii, eger ustteki durumda $x_i$ {\em verildikten sonra} $y_i$'larin
birbirinden bagimsiz olmasi yeterli.

Bu model klasik Istatistik'te cokca kullanilan bir yaklasimdir, hatta
lineer regresyon'un temeli ustteki faraziyedir. 

$$ y = \alpha + \bar{\beta}\bar{x} + N(0,\sigma^2) $$

Bu standart lineer regresyon modeli, ve bu modelde her $y$ ona tekabul eden
$x$'e bagli, bu sayede $x$'ler biliniyorsa $y$'ler birbirinden kosulsal
olarak bagimsiz hale geliyor, boylece $x$'ler birbirine bagimli olsa bile
$\alpha$ ve $\beta$'nin bulunmasi mumkun oluyor. 



\end{document}

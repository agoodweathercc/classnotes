\documentclass[12pt,fleqn]{article}\usepackage{../common}
\begin{document}
QP

Icinde esitsizlikleri de barindiran ve karesel olan bir matematiksel sistemi
cozmek icin karesel programlama (quadratic programming) tekniklerini
kullanabiliriz. Problemler su sekilde verilir:

\[ \frac{1}{2}x^TQx+p^Tx \textrm{ fonksiyonunu minimize et} \]

su kosullara uymak sartiyla (subject to)

\[ Gx \leq h \textrm{ (esitsizlik kosulu)} \]

\[ Ax = b \textrm{ (esitlik kosulu)} \]

Kucuk harfli gosterilen degiskenler vektordur, buyuk harfler ise bir matrisi
temsil ederler. $x$ icinde diger bilinmeyenler $x_1, x_2, ..$ olarak
vardir, bulmak istedigimiz degerler buradadir.

Somut ornek olarak suna bakalim:

\[ 2x_1^2 + x_2^2 + x_1x_2+x_1+x_2 \textrm{ fonksiyonunu minimize et} \]

kosullar:

\[ x_1 \geq 0, x_2 \geq 0 \textrm{ (esitsizlik kosullari)} \]

\[ x_1 + x_2 = 1 \textrm{ (esitlik kosulu)} \]

Fakat bu formul su anda matris formunda degil. Matris formuna gecmek icin iki
asama var. Once $x$ degiskenlerinin birbiri ve kendileri ile carpim durumlarini
halledelim. Oyle bir $Q$ matrisi bulmaliyiz ki, altta bos olan $Q$ matrisinin
degerleri doldurulup, carpim yapildiginda $x$ degiskenlerinin tum carpim
iliskilerini bulsun. Carpim iliskileri nelerdir?  Formulun $2x_1^2 + x_2^2 +
x_1x_2$ kismidir.

\[ 
\left[ \begin{array}{cc}
    x_1 & x_2 
\end{array} \right]
\left[ \begin{array}{cc}
    .. & .. \\
    .. & ..
\end{array} \right]
\left[ \begin{array}{c}
    x_1 \\
    x_2 
\end{array} \right]
 \]

$Q$ matrisinin $1,2,..$ gibi kordinatlari $x_1,x_2,..$'ye tekabul ediyor
olacaklar.  (1,1) kordinatlari $x_1$'in kendisi ile carpimini, $x_1^2$'i temsil
eder, (1,2) ise $x_1x_2$'yi temsil eder, vs. O zaman (1,1) icin 2 sayisini
veriririz, cunku $x_1^2$'nin basinda $2$ degeri var. (2,2) icin $1$ degeri lazim
cunku $x_2^2$'nin basinda sayi yok (yani '1' degeri var).

(1,2) ve (2,1) ilginc cunku ikisi de aslinda $x_1x_2$'i temsil
ediyorlar cunku $x_1x_2 = x_2x_1$. O zaman (1,2) ve (2,1) icin $0.5$
degeri verirsek, $0.5x_1x_2 + 0.5x_2x_1$'i kisaltip $x_1x_2$ haline
getirebiliriz. Sonuc

\[ 
Q = \left[ \begin{array}{cc}
    2 & 0.5 \\
    0.5 & 1
\end{array} \right]
 \]

Kontrol edelim:

\[ 
 \left[ \begin{array}{cc}
    x_1 & x_2 
\end{array} \right]
\left[ \begin{array}{cc}
    2 & 0.5 \\
    0.5 & 1
\end{array} \right]
\left[ \begin{array}{c}
    x_1 \\
    x_2 
\end{array} \right] \\
 \]

\[ 
= \left[ \begin{array}{cc}
    2x_1 + 0.5x_2 & 0.5x_1 + x_2 
\end{array} \right]
\left[ \begin{array}{c}
    x_1 \\
    x_2 
\end{array} \right] 
 \]

\[ = 2x_1^2 + 0.5x_2x_1 + 0.5x_1x_2 + x_2^2  \]

\[ = 2x_1^2 + x_1x_2 + x_2^2  \]

$p$ vektoru ise, her terimin, tek basina ana formule nasil eklenecegini kontrol
ediyor. Elimizde $x_1 + x_2$ olduguna gore $p = [1, 1]$ yeterli olacaktir,
bakalim: $[1, 1]^T[x_1, x_2] = x_1 + x_2$

Simdi esitsizlik kosullari. Bizden istenen $x_1 \geq 0$ ve $x_2 \geq 0$
sartlarini $Gx \leq 0$ formunda temsil etmemiz. Burada onemli nokta matris
formuna gecerken bir yandan da $\geq$ isaretini tersine dondurmemiz, yani $\leq$
yapmamiz. Bu cok dert degil, degiskeni $-1$ ile carparsak isareti tersine
dondurebiliriz cunku $x_1 \leq 0$ ile $-x_1 \geq 0$ aynidir. O zaman $Gx$ soyle
olacak: 

\[ 
\left[ \begin{array}{cc}
    -1 & 0 \\
    0 & -1
\end{array} \right]
\left[ \begin{array}{c}
    x_1 \\
    x_2
\end{array} \right]
\leq
\left[ \begin{array}{c}
    0 \\
    0
\end{array} \right]
 \]

\[ 
\left[ \begin{array}{c}
    -x_1 \\
    -x_2
\end{array} \right]
\leq
\left[ \begin{array}{c}
    0 \\
    0
\end{array} \right]
 \]

Esitlik kosullari

Esitlik kosullari icin problemimizin istediklerini $Ax = b$ formuna uydurmamiz
lazim. $x_1 + x_2$'yi nasil forma sokariz? $A = [1, 1]$, $b = 1$ ile

\[ 
[1, 1] \left[ \begin{array}{c}
    x_1 \\
    x_2
\end{array} \right] 
= 1 \\
 \]

\[ x_1 + x_2 = 1 \]

CVXOPT

Bu paket ile karesel denklemleri minimizasyon / maksimizasyon baglaminda cozmek
mumkundur. Ustte buldugumuz degerleri altta gorebiliyoruz. Q esitliginde 2 ile
carpim var, bunun sebebi karesel denklem formunun basinda $\frac{1}{2}$ carpimi
olmasi, boylece bu iki carpim birbirini dengeliyor.

\begin{lstlisting}[language=Python]
from cvxopt import matrix
from cvxopt import solvers
Q = 2*matrix([ [2, .5], [.5, 1] ])
p = matrix([1.0, 1.0])
G = matrix([[-1.0,0.0],[0.0,-1.0]])
h = matrix([0.0,0.0])
A = matrix([1.0, 1.0], (1,2))
b = matrix(1.0)
sol=solvers.qp(Q, p, G, h, A, b)
print sol['x']
\end{lstlisting}

Sonuc $x_1$ icin [ 2.50e-01] ve $x_2$ icin [ 7.50e-01] olmali.

Bazi notlar: A matrisi yaratilirken (1,2) kullanimi goruluyor, bu matrisin
boyutlarini tanimlamak icin. Cvxopt paketi bu arada Numpy formati degil kendi
matris, vektor objelerini kullaniyor, ama ikisi arasinda gidip gelmek mumkun. 

References

http://abel.ee.ucla.edu/cvxopt/examples/tutorial/qp.html

http://www.mblondel.org/journal/2010/09/19/support-vector-machines-in-python/

\end{document}


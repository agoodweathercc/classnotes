\documentclass[12pt,fleqn]{article}
\setlength{\parindent}{0pt}
\usepackage{graphicx}
\usepackage{listings}
\usepackage[latin5]{inputenc}
\setlength{\parskip}{8pt}
\setlength{\parsep}{0pt}
\setlength{\headsep}{0pt}
\setlength{\topskip}{0pt}
\setlength{\topmargin}{0pt}
\setlength{\topsep}{0pt}
\setlength{\partopsep}{0pt}
\setlength{\mathindent}{0cm}

\begin{document}
Taylor Serisini Turetmek 

Taylor serilerinin arkasindaki fikir, surekli ve sonsuz defa turevi
alinabilen turden bir fonksiyon $f(x)$'i bir $x_0$ noktasinin (burada $a$
sembolu de kullanilabilir) ``cevresinde'', yakin bolgesinde yaklasiksal
olarak temsil edebilmektir.

Turetmek icin

Calculus'un Temel Teorisi der ki:

\[ \int_a^x f' \left({t}\right) \ \mathrm d t = 
f \left({x}\right) - f \left({a} \right)
\]

Bu formulu tekrar duzenlersek, alttakini elde ederiz:

\[ f \left({x}\right) = f \left({a}\right) + \int_a^x f'(t) \ \mathrm d t \]

Bunun uzerinde Parcali Entegral yontemini uygulariz. Parcali Entegral
teknigi genel olarak soyledir:

\[ \int_a^b u \ dv = u \ v - \int_a^b v \ du \]

Simdi iki ustteki formulun entegral icindeki kismini parcali entegrale
uyacak sekilde bolusturelim

$u = f' \left({t}\right)$ ve $dv = dt$

O zaman acilim

\[ f \left({a}\right) + x f' \left({x}\right) - a f' \left({a}\right) - \int_a^x t f'' \left({t}\right) \ \mathrm d t \]

Alttaki formulu kullanarak

\[ \int_a^x x f'' \left({t}\right) \ \mathrm d t = x f' (x)-x f' (a) \]

iki ustteki formulu su hale getiririz

\[ f \left({a}\right) + \int_a^x x f'' \left({t}\right) \ \mathrm d t + x f' \left({a}\right) - a f' \left({a}\right)-\int_a^x \, t f'' \left({t}\right) \ \mathrm d t \]

Bazi ortak terimleri disari cekersek

\[ f \left({a}\right) + (x-a) f' \left({a}\right) + \int_a^x (x-t) f'' \left({t}\right) \ \mathrm d t \]

Ayni teknigi bir daha uygulayinca

\[ f \left({x}\right) = f \left({a}\right)+(x-a) f' \left({a}\right)+ \frac 1 2 (x-a)^2f'' \left({a}\right) + \frac 1 2 \int_a^x (x-t)^2 f''' \left({t}\right) \ \mathrm d t\]

Tum bunlari daha genel olarak kurallastirmamiz gerekirse, tumevarim
(induction) teknigini kullanalim, varsayiyoruz ki Taylor'un Teorisi bir $n$
icin gecerli ve

\[ f(x) = f(x) + \frac{f'(a)}{1!}(x - a) + ... 
\frac{f^{(n)}(a)}{n!}(x - a)^n + 
\int_a^x \frac{f^{(n+1)} (t)}{n!} (x - t)^n \ \mathrm d t
\]

Sonuncu entegrali parcali entegral teknigi ile tekrar yazmamiz
mumkundur. $(x-t)^n$'in anti-turevi (anti-derivative)
$\frac{-(x-t)^{n+1}}{n+1}$ ile verilir, o zaman

\[  \int_a^x \frac{f^{(n+1)} \left({t}\right)}{n!} \left({x - t}\right)^n \ 
\mathrm d t\]

\[ =  - \left[ \frac{f^{(n+1)} \left({t}\right)}{(n+1)n!} \left({x - t}\right)^{n+1} \right]_a^x + \int_a^x \frac{f^{(n+2)} \left({t}\right)}{(n+1)n!} \left({x - t}\right)^{n+1} \ \mathrm d t \]

\[ = \frac{f^{(n+1)} (a)}{(n+1)!} (x - a)^{n+1} + \int_a^x \frac{f^{(n+2)} \left({t}\right)} {(n+1)!} \left({x - t}\right)^{n+1} \ \mathrm d t \]

Son entegral hemen cozulebilir

\[ R_n = \frac{f^{(n+1)}(\xi)}{(n+1)!} (x-a)^{n+1} \]

Alternatif Form

Hesapsal Bilim derslerinde bu serinin alternatif bir formu daha cok
karsimiza cikabilir. $f$'i $t$ yakininda ufak bir $h$ adimi atildigini
farzederek Taylor serileri uzerinden $f(t+h)$'i gelistirmek suretiyle
temsil edebiliriz. Eger $x = t+h$ ve $a = t$ alirsak alttaki orijinal
Taylor serisini

\[ f(x) = f (x_0)+(x-x_0) f'(x_0) + \frac1 2 (x-x_0)^2f''(x) + ...\]

su sekle donusturebiliriz

\[ f(t+h) = f (t)+h f'(t) + \frac 1 2 h^2 f''(t) + ...\]

Bu tanimin, birinci turevin formuluyle olan alakasini gormek icin

\[ f'(x) \approx \frac {f(x+h) - f(x)}{h} \]

ifadesini hatirlamak yararli olabilir, yaklasiksal isareti $\approx$
kullanildi, cunku bu ifade sadece $h \to 0$ iken dogrudur (turevlerin limit
olarak tanimindan hareketle). Biraz cebirsel manipulasyon yaparsak

\[ f(x+h) - f(x) \approx f'(x)h  \]

\[ f(x+h)  \approx f'(x)h + f(x) \]

En son formulun Taylor serisi 1. derece acilimiyla ayni oldugu goruluyor. 

Kaynak

http://www.proofwiki.org/wiki/Taylor's\_Theorem

\end{document}




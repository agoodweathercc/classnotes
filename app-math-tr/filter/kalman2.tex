\documentclass[a4paper,11pt]{article}

\usepackage[latin5]{inputenc}
\usepackage{amsmath}
\usepackage{examplep}
\usepackage{amsthm}
\usepackage{amssymb}
\usepackage{amsfonts}
\usepackage{graphics}

\title{Kalman Filters: Derivation}

\author{Burak Bayraml�}

\begin{document}

\maketitle

\begin{eqnarray*}
p(z_n|z_{n-1}) &=& N(z_n|Az_{n-1},\Gamma)\\
p(x_n|z_{n}) &=& N(x_n|Cz_{n},\Sigma)
\end{eqnarray*}


The quantity, random variable, $\alpha(z_n)$ represent the joint probability of
all of the data $x_1,..,x_n$ and the value of $z_n$. We will try to represent
$\alpha(z_n)$ in terms of $\alpha(z_{n-1})$ in order to get a recursive
relation. 

\begin{eqnarray*}
\alpha(z_n)
&=& p(x_1,..,x_n,z_n) \\
&=& p(x_1,..,x_n|z_n)p(z_n) \\
&=& p(x_n|z_n)p(x_1,..,x_{n-1}|z_n)p(z_n)\\
&=& p(x_n|z_n)p(x_1,..,x_{n-1},z_n)\\
&=& p(x_n|z_n) \int p(x_1,..,x_{n-1},z_{n-1},z_n) d z_{n-1}\\
&=& p(x_n|z_n) \int p(x_1,..,x_{n-1},z_n|z_{n-1})p(z_{n-1}) d z_{n-1}\\
&=& p(x_n|z_n) \int p(x_1,..,x_{n-1}|z_{n-1})p(z_n|z_{n-1})p(z_{n-1}) d z_{n-1}\\
&=& p(x_n|z_n) \int p(x_1,..,x_{n-1},z_{n-1})p(z_n|z_{n-1}) d z_{n-1}
\end{eqnarray*}

Using the definition for $\alpha(z_n)$ we see that the factor after integral
looks like $\alpha(z_{n-1})$. Then we can restate the equation above in terms of
$\alpha$. 

\begin{eqnarray*}
\alpha(z_n)  &=& p(x_n|z_n) \int \alpha(z_{n-1}) p(z_n|z_{n-1}) d z_{n-1}
\end{eqnarray*}

Now all of the factors above are familiar from a Kalman filter based
model. $p(x_n|z_n)$ is the emission probability, $p(z_n|z_{n-1})$ is the
transition probability. These distributions will be known to us before we start
filtering, hence can be used for determining each $\alpha$ as time goes on. 

\subsubsection*{Scaling}

As time passes, since at each time step we are multiplying small probability
values (less than 1) with eachother, we can quickly reach very small numbers
very fast. In order to save $\alpha(z_n)$ from becoming too small, we can use a
normalized version. 

\begin{eqnarray*}
\hat{\alpha}(z_n) &=& \frac{\alpha(z_n)} {p(x_1,..,x_n)}
\end{eqnarray*}

In order to work this into our equation, 

\begin{eqnarray*}
  c_n &=& p(x_n|x_1,..,x_{n-1})\\
  \prod_{m=1}^n c_m  &=& p(x_1,...,x_n)
\end{eqnarray*}

The last statement follows from product rule in probability theory, say for 

\begin{eqnarray*}
  p(x_1)p(x_2|x_1)p(x_3|x_2,x_1)...
\end{eqnarray*}
The first two factors result in $p(x_1,x_2)$ which, in turn multiplied by third
factor result in $p(x_3,x_2,x_1)$, so on. Then, we can turn the recursive
statement $\alpha(z_n)$ into a scaled version

\begin{eqnarray*}
\alpha(z_n)  &=& p(x_n|z_n) \int \alpha(z_{n-1}) p(z_n|z_{n-1}) d z_{n-1}  \\
\bigg(\prod_{m=1}^n c_m \bigg) \hat{\alpha}(z_n)  &=& 
p(x_n|z_n) \int \bigg( \prod_{m=1}^{n-1} c_m\bigg)\hat{\alpha}(z_{n-1}) p(z_n|z_{n-1}) d z_{n-1}  
\end{eqnarray*}
Since the only difference between two $c_m$ products on the left and right
is one $c_n$, once canceled out, only $c_n$ on the left handside
remains. 

\begin{eqnarray*}
c_n\hat{\alpha}(z_n)  &=& p(x_n|z_n) \int \hat{\alpha}(z_{n-1}) p(z_n|z_{n-1}) d z_{n-1} 
\end{eqnarray*}

where

\begin{eqnarray*}
\hat{\alpha}(z_n) = N(z_n|\mu_n,V_n)
\end{eqnarray*}


\subsubsection*{Filtering}

Substituting into recursive equation

\begin{eqnarray*}
     c_n
     N(z_n|\mu_n,V_n)
&=& 
     N(x_n|Cz_{n},\Sigma)
     \int 
     N(z_n|Az_{n-1},\Gamma)
     N(z_{n-1}|\mu_{n-1},V_{n-1})     
     d z_{n-1} 
\end{eqnarray*}


\begin{thebibliography}{99}
  
\bibitem{bishop} C. Bishop
  {\em Pattern Recognition and Machine Learning },
   2006.
  
\end{thebibliography}

\end{document}

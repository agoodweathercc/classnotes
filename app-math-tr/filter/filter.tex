\documentclass[12pt,fleqn]{article}
\setlength{\parindent}{0pt}
\usepackage{graphicx}
\usepackage{cancel}
\usepackage{listings}
\usepackage[latin5]{inputenc}
\setlength{\parskip}{8pt}
\setlength{\parsep}{0pt}
\setlength{\headsep}{0pt}
\setlength{\topskip}{0pt}
\setlength{\topmargin}{0pt}
\setlength{\topsep}{0pt}
\setlength{\partopsep}{0pt}
\setlength{\mathindent}{0cm}

\begin{document}
Filtrelemek

Filtreler dis dunyadaki bir aksiyon hakkinda elde edilen gurultulu
sinyalleri, tersine cevirerek arka plandaki aksiyon hakkinda hesaplama
yapabilmemizi saglar. Mesela Kalman Filtreleri (KF) icin gizlenmis konum
bir robotun nerede oldugu, bir senetin fiyati gibi bir sey olabilir, gizli
konum bilgisi $x_t$ degiskeninde o konum hakkindaki gurultulu olcum $y_t$
icindedir. Hem gizli konumlar arasindaki gecis, hem de olcumun gurultusu
lineer bir fonksiyon uzerindendir.

\[ x_{t+1} = Ax_t + v \]

\[ y_t = Hx_t + w \]

$v$ ve $w$'in dagilimi Gaussian'dir ve kovaryans sirasiyla $Q$ ve $R$
icindedir. 

Zaman faktorunu de dahil etmek gerekirse;

\[ \hat{x}_t^t = E[x_t|y_0,..,y_t] \]

\[ P_t^t = E[(x_t - \hat{x}_{t|t}) (x_t - \hat{x}_{t|t})'| y_0,...,y_t   ] \]

Filtremenin amaci $x_{t+1}$ ve $P_{t+1}$ hesabini yeni bir olcum $y_{t+1}$
uzerinden yapmak olacak. ``Gizli'' $x_t$ derken bunu kastediyorduk, bu
deger bize verilmiyor, sadece xt ve $x_{t+1}$ arasindaki gecisin nasil oldugunu
biliyoruz, gurultunun nasil eklendigini biliyoruz, ama bunlarin bilsek bile
elde bir suru bilinmeyen var. Filtrelemenin matematiksel numaralari
sayesinde bunu hesaplayabiliyor olacagiz.  Yani yapmamiz gereken ``oku
tersine cevirmek'', yani $x_t$'nin $y_t$ uzerindeki sartsal bagliligini
(conditional dependence) ortaya cikartmak, bunu $y_t$'nin $x_t$'ye olan sartsal
bagimliligini tersine cevirerek yapmak. Ana denklemin iki tarafinin da
beklentisini (expectation) alalim:

\[ E \ x_{t+1} = \hat{x}_{t+1} = A \mu_t = A \hat{x}_t \]

Simdi iki tarafin kovaryansini alalim ve $P_t$'yi $cov \ x(t)$ olarak
belirtelim:















\end{document}

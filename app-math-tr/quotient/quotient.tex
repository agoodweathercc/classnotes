\documentclass[12pt,fleqn]{article}\usepackage{../common}
\begin{document}
Turev Alirken Bolum Kurali (Quotient Rule)

Bolum kurali soyle gosterilir 

\[ \frac{ u(x)}{v(x)} = \frac{u(x)'v(x) - u(x)v(x)'}{v(x)^2} \]

Ya da $x$ gostermeden

\[ \frac{ u}{v} = \frac{u'v - uv'}{v^2} \]

Bu formulu hatirlamak biraz zor gelebilir. Eger hatirlamiyorsak hatirlamasi
daha basit olan Carpim Kurali (product rule) uzerinden
turetebiliriz. Carpim kurali bildigimiz gibi

\[ (uv)' = u'v + uv' \]

Burada bir numara yaparak 

\[ (u \ \frac{ 1}{v})' \]

uzerinde Carpim Kuralini kullanarak turev alacagiz, boylece otomatik olarak
arka planda aslinda $u/v$'nin turevini aldirtmis olacagiz. 

\[ \bigg(u \ \frac{ 1}{v}\bigg)' = 
u \bigg(\frac{1}{v}\bigg)' + u' \bigg(\frac{1}{v}\bigg)
\]

Bu arada

\[ \bigg(\frac{ 1}{v}\bigg)'  = -\frac{v'}{v^2} \]

Bolumde $v'$ var, cunku unutmayalim $v$ aslinda $v(x)$, o zaman ana
formulde yerine koyalim

\[  = 
-u\frac{v'}{v^2}  + u' \bigg(\frac{1}{v}\bigg)
\]

\[  = 
\frac{u' }{v} -\frac{uv'}{v^2}
\]


Birinci terimde bolum ve boleni $v$ ile carpalim, ki iki terimi
birlestirebilelim, 

\[  = 
\frac{u'v }{v^2} -\frac{uv'}{v^2} = 
\frac{u'v-uv'}{v^2}
\]

\end{document}

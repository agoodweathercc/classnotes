\documentclass[12pt,fleqn]{article}\usepackage{../common}
\begin{document}
$e^{-x^2}$ Nasil Entegre Edilir? 

\[ \int_{-\infty}^{\infty} e^{-x^2} dx \ \ \ \label{1} \]

ifadesi ozellikle olasilik matematiginde cokca gorulen bir ifadedir. Bu
hesabi yapmak icin kutupsal kordinatlar kullanacagiz. 

Simdi ustteki ifadeyle alakali su ifadeye bakalim. 

\[ \int_{-\infty}^{\infty} \int_{-\infty}^{\infty} e^{-x^2-y^2} dx dy \]

Iddia ediyorum ki bu son ifade (1)'in sadece karesi, yani (1)'in kendisiyle
carpimi. Niye boyle? Cunku $e$ ifadelerini carpim olarak gosterirsek

\[ \int_{-\infty}^{\infty} 
\underbrace{\int_{-\infty}^{\infty} e^{-x^2} e^{-y^2} dx}
dy \]

cift entegral icinde isaretlenen blokta yer alan $e^{-y^2}$ $x$'ten
bagimsiz, o zaman bloktaki entegralin disina alinabilir. Yani soyle olabilir

\[ \int_{-\infty}^{\infty} 
e^{-y^2} \int_{-\infty}^{\infty} e^{-x^2}  dx
dy \]

Devam edelim: ustteki ic entegral (1) ifadesi degil mi? Evet. Simdi bir
ilginc durum daha ortaya cikti, 

\[ \int_{-\infty}^{\infty}  e^{-y^2} 
\underbrace{\int_{-\infty}^{\infty} e^{-x^2}  dx}
dy \]

simdi de isaretlenen blok $y$ entegraline gore sabit, o da ikinci
entegralin disina cikarilabilir! (1) yerine $I$ kullanirsak 

\[ I \int_{-\infty}^{\infty}  e^{-y^2} dy \]

Icinde $y$ iceren entegral nedir? O da $I$'dir! Niye, cunku bu ifade 
(1)'in icinde $y$ olan versiyonundan ibaret. O zaman 

\[ I \cdot I = I^2 \]

Tum bu taklalari niye attik peki? Cunku cift entegralli ifadenin
entegralini almak daha kolay, eger onu hesaplarsak, sonucun karekokunu
aldigimiz anda $I$'yi bulmus olacagiz. 

O zaman ifadeyi hesaplayalim, 

\[ I^2 = \int_{-\infty}^{\infty} \int_{-\infty}^{\infty} e^{-x^2-y^2} dx dy \]

Basta soyledigimiz gibi kullanacagimiz numara kutupsal forma
gecmek. Entegralin sinirlarina bakalim, tum $x$ ve tum $y$ ekseni uzerinden
entegral aliyoruz. Kutupsal formda bu $r$'nin 0'dan sonsuza ve $\theta$'nin
0'dan $2\pi$'a gitmesi anlamina geliyor. 

\[ r^2 = x^2 + y^2 \]

Peki $e^{-x^2-y^2}$ kutupsal formda nedir? 

\[ e^{-x^2-y^2} = e^{-(x^2+y^2)} = e^{-r^2} \]

Entegrali yazalim

\[ \int_{0}^{\infty} \int_{0}^{2\pi} e^{-r^2} r d\theta dr 
\ \ \ \label{2}
\]

Niye entegral sirasinda $\theta$'yi once yazdim? Cunku entegral icindeki
ifadede $\theta$'ya bagli hicbir terim yok, o zaman ic entegral bana sadece
$2\pi$ verir. Geriye kalanlar

\[=  2\pi \int_{0}^{\infty} e^{-r^2} r dr \]

Bu entegral cok daha kolay. Yerine koyma (subtitution) teknigi ile bu
problemi cozebiliriz. 

\[ u = -r^2 \]

\[ du =  -2r \ dr\]

\[=  2\pi \int_{0}^{\infty} e^u \frac{-1}{2} du \]

\[=  -\pi \int_{0}^{\infty} e^u  du \]

\[=  -\pi  e^u  \bigg|_{0}^{\infty} = -\pi  e^{-r^2}  \bigg|_{0}^{\infty} \]

\[ = \pi \]

Bu sonuc $I^2$. Eger $I$ degerini istiyorsak, karekok almaliyiz, yani
aradigimiz sonuc $\sqrt{\pi}$. 

Tek degiskenli bir problemi aldik ve cift degiskenli problem haline
getirdik. Isleri kolaylastiran (2) denklemindeki $r$ degiskeni oldu, onun
sayesinde yerine gecirme islemi cok kolaylasti, ve sonuca ulastik. 



http://www.youtube.com/watch?v=fWOGfzC3IeY


\end{document}

\documentclass[12pt,fleqn]{article}\usepackage{../common}
\begin{document}
Regresyon, Ridge, Lasso, Capraz Saglama, Regularize Etmek

Konumuz regresyon cesitleri, ve ornek veri olarak diyabet hastaligi
olan kisilerden alinmis bazi temel verilerle hastaligin bir sene
sonraki ilerleme miktari kullanilacak. Regresyon sayesinde temel
veriler ile hastaligin ilerlemesi arasinda bir baglanti bulunabilir,
bu sayede hem veri aciklanir / daha iyi anlasilir (hangi degisken
onemlidir, hangisi degildir), hem de baska bir hastanin temel
verilerini kullanarak o hastanin diyabetinin bir sene sonra ne
olacagini tahmin etmek mumkun olur. Kullanilan temel veriler kisinin
yasi, cinsiyeti, vucut kutle endeksi (body mass index) ortalama
tansiyonu ve alti kere alinmis kan serum olcumleridir.

\begin{minted}[fontsize=\footnotesize]{python}
from pandas import *
diabetes = read_csv("diabetes.csv",sep=';')
diabetes_y = diabetes['response']
diabetes_x = diabetes.drop("response",axis=1)
diabetes_x_train = diabetes_x[:-20]
diabetes_x_test  = diabetes_x[-20:]
diabetes_y_train = diabetes_y[:-20]
diabetes_y_test  = diabetes_y[-20:]
\end{minted}

Ilk once basit regresyonu hatirlayalim (ordinary least regression). Bu
teknigi daha once pek cok yonden gorduk. {\em Lineer Cebir}, {\em Cok
Degiskenli Calculus} ders notlarinda, ve bizim {\em Uygulamali Matematik}
yazilarinin hepsinde bu teknigin turetilmesi mevcut. Formul

$$ \hat{w} = (X^TX) ^{-1} X^{T}y $$

Numerik olarak hemen bu hesabi yapabiliriz. Bir hatirlatma: veri
setine y ekseninin nerede kesildiginin bulunabilmesi icin suni bir
ekstra 'intercept' adli kolon ekleyecegiz, bu kolon iki boyutta
$y=ax+c$ formulundeki $c$'nin bulunabilmesi icindir. Pandas ile bu ekstra
kolonu eklemek cok basit, ismen mevcut olmayan kolon erisildigi anda
o kolon hemen yoktan yaratilir.

\begin{minted}[fontsize=\footnotesize]{python}
import numpy.linalg as la
x_tmp = diabetes_x_train.copy()
x_tmp['intercept'] = 1
xTx = np.dot(x_tmp.T,x_tmp )
ws = np.dot(la.inv(xTx),np.dot(x_tmp.T,diabetes_y_train))
print ws
\end{minted}

\begin{verbatim}
[  3.03499452e-01  -2.37639315e+02   5.10530605e+02   3.27736981e+02
  -8.14131711e+02   4.92814589e+02   1.02848453e+02   1.84606489e+02
   7.43519617e+02   7.60951724e+01   1.52764307e+02]
\end{verbatim}

Ayni hesabi bir de \verb!scikit-learn! paketini kullanarak
yapalim. Bu paketin \verb!LinearRegression! cagrisi intercept
isini otomatik olarak hallediyor, eger intercept olmasin isteseydik,
\verb!fit_intercept=False! diyecektik.

\begin{minted}[fontsize=\footnotesize]{python}
from sklearn import linear_model, cross_validation
lin = linear_model.LinearRegression()
lin.fit(diabetes_x_train, diabetes_y_train)
print lin.coef_
print "score", lin.score(diabetes_x_test, diabetes_y_test), 
\end{minted}

\begin{verbatim}
[  3.03499452e-01  -2.37639315e+02   5.10530605e+02   3.27736981e+02
  -8.14131711e+02   4.92814589e+02   1.02848453e+02   1.84606489e+02
   7.43519617e+02   7.60951724e+01]
score 0.585075302278
\end{verbatim}

Sonuclar birbirine oldukca yakin. Simdi diger tekniklere gelelim.

S�rt Regresyonu (Ridge Regression)

Klasik regresyon ile

$$ \hat{w} = \arg \min_w ||y-Xw||^2  $$

problemini cozdugumuzu biliyoruz, ki $||\cdot||^2$ Oklit normunun
karesini temsil ediyor. Fakat bazi durumlarda $X^TX$'in tekil
(singular) olmasi mumkun ki boyle bir durumda $(X^TX)^{-1}$'in tersini
almamiz mumkun olmazdi. Tekillik ne zaman ortaya cikar? Eger elimizde
veri noktasindan daha fazla boyut var ise mesela... Diyelim ki veri
olarak 10 tane kolon var, ama sadece 9 tane veri satiri. Sirt
Regregyonunun cikis noktasi budur.

Fakat ek olarak bu teknik kestirme hesaplarimiza (estimation) bir
meyil / yanlilik (bias) eklemek icin de kullanilabilir ve bu meyil
tahminlerin / kestirme hesaplarin iyilesmesine faydali olabilir.

Meyili nasil ekleriz? Diyelim ki bizim tanimlayacagimiz bir $\lambda$ ile
tum $ws$'lerin toplamina bir ust sinir tanimlayabiliriz. Boylelikle
regresyonun bulacagi katsayilarin cok fazla buyumesine bir "ceza" getirmis
olacagiz, ve bu cezayi iceren regresyon hesabi o cezadan kacinmak icin
mecburen bulacagi katsayilari ufak tutacak, hatta bazilarini sifira indirebilecek.
Bu azaltmaya istatistikte kuculme (shrinkage) ismi veriliyor. 

Sirt regresyonu icin bu kucultme soyle

$$ \hat{w}_{sirt} = \arg \min_w ( ||y-Xw||^2 + \lambda||w||^2 )  $$

Goruldugu uzere $w$'nin buyuklugunu, bir $\lambda$ katsayisi uzerinden
minimizasyon problemine dahil ettik, boylece diger parametreler ile buyukluk te
minimize edilecek. Ustteki tanim siniri tanimlanmamis (unconstrained) bir
optimizasyon problemidir. Sinirli olarak

$$ \min_w ||y-Xw||^2  $$

$$ \textit{Su kosula gore (subject to) }||w||^2 \le \tau $$

ki $\lambda$ Lagrange carpanidir. Aslinda simdiye kadar ustteki
cevrimin tersini gorduk cogunlukla (yani sinirli problemden sinirsiza
gitmeyi), bu gidis tarzini gormek te iyi oldu.

Neyse bastaki sinirsiz problemi cozmek icin ifadenin gradyanini alalim,

$$ \nabla \big( ||y-Xw||^2 + \lambda||w||^2 \big) $$

$$ \nabla \big( (y-Xw)^T (y-Xw) + \lambda w^Tw \big) $$

$$ \nabla \big(  (y^T-w^TX^T)(y-Xw) + \lambda w^Tw  \big) $$

$$ \nabla ( y^Ty - y^TXw - w^TX^Ty + w^TX^TXw + \lambda w^Tw   )  $$

$$  - y^TX - X^Ty + 2X^TXw + 2\lambda w   $$

$$  - 2X^Ty + 2X^TXw + 2\lambda w   $$

$$   2X^TXw + 2\lambda w - 2X^Ty   $$

$$   2(X^TX + \lambda I ) w - 2X^Ty   $$

Minimizasyon icin ustteki ifadeyi sifira esitleyebiliriz

$$   2(X^TX + \lambda I ) w - 2X^Ty  = 0 $$

O zaman

$$   (X^TX + \lambda I ) w = X^Ty  $$

$$   \hat{w} = (X^TX + \lambda I)^{-1} X^Ty  $$

Bu son ifade en az kareler (least squares) yani normal regresyon cozum
formulune cok benziyor, sadece ek olarak bir $\lambda I$ toplama
islemi var.  Demek ki sirt regresyonunu kullanmak icin zaten
yaptigimiz hesaba, zaten bizim kendimizin karar verdigi bir $\lambda$
uzerinden $\lambda I$ eklersek, geri kalan tum islemler ayni olacak. 

Kontrol edelim

\begin{minted}[fontsize=\footnotesize]{python}
lam = 0.2
wridge = np.dot(la.inv(xTx+lam*np.eye(xTx.shape[0])),\
                np.dot(x_tmp.T,diabetes_y_train))
print wridge
\end{minted}

\begin{verbatim}
[  16.70807829 -179.42288145  447.64999897  285.41866481  -51.7991733
  -75.09876191 -192.46341288  123.61066573  387.91385823  105.53294479
  152.7637018 ]
\end{verbatim}

Simdi \verb!scikit-learn! ile ayni hesabi yapalim

\begin{minted}[fontsize=\footnotesize]{python}
ridge = linear_model.Ridge(alpha=0.2)
ridge.fit(diabetes_x_train, diabetes_y_train) 
print ridge.score(diabetes_x_test, diabetes_y_test), ridge.coef_
\end{minted}

\begin{verbatim}
0.553680030106 [  16.69330211 -179.414259    447.63706059  285.40960442  -51.79094255
  -75.08327488 -192.45037659  123.60400024  387.91106403  105.55514774]
\end{verbatim}

Bir yontem daha var, bu yonteme Lasso ismi veriliyor. Lasso'ya gore cezalandirma

$$ \sum_{k=1}^{n} w_k^2 \le \lambda $$

uzerinden olur. Bu yontemin tum detaylarina simdilik
inmeyecegiz.

Ornek olarak bir $\lambda$ ile onun buldugu katsayilara bakalim.

\begin{minted}[fontsize=\footnotesize]{python}
lasso = linear_model.Lasso(alpha=0.3)
lasso.fit(diabetes_x_train, diabetes_y_train)
print lasso.coef_
\end{minted}

\begin{verbatim}
[   0.           -0.          497.3407568   199.17441037   -0.           -0.
 -118.89291549    0.          430.93795945    0.        ]
\end{verbatim}

Lasso bazi katsayilari sifira indirdi! Bu katsayilarin agirlik verdigi
degiskenleri, eger Lasso'ya inanirsak, modelden tamamen atmak mumkundur.

Bu arada Sirt ve Lasso yontemlerinin metotlarina "regularize etmek
(regularization)" ismi de veriliyor. 

k-Katlamali Capraz Saglama (k-fold Cross-Validation)

Bir yapay ogrenim algoritmasini kullanmadan once veriyi iki parcaya
ayirmak ise yarar; bu parcalar tipik olarak egitim verisi (training
set) digeri ise test verisi (validation set) olarak
isimlendirilir. Isimlerden belli olacagi uzere, algoritma egitim seti
uzerinde egitilir; ve basarisi test verisi uzerinden rapor edilir. Bir
bakima modelin olusturulmasi bir set uzerindendir, sonra "al simdi hic
gormedigin bir veri seti, bakalim ne yapacaksin" sorusunun cevabi,
saglamasi bu sekilde yapilir.

k-Katlamali Capraz Saglama bu iki parcali egitim / test kavramini bir
adim oteye tasir. Ufak bir k seceriz, ki bu genellikle 5 ila 10
arasinda bir sayi olur, ve tum verimizi rasgele bir sekilde ama k tane
ve esit buyuklukte olacak sekilde parcalara ayiririz. Bu parcalara
"katlar (folds)" ismi verilir bazen (ki isim buradan geliyor). Sonra
teker teker her parcayi test verisi yapariz ve geri kalan tum
parcalari egitim verisi olarak kullaniriz. Bu islemi tum parcalar icin
tekrarlariz.

Bu yaklasim niye faydalidir? Cunku veriyi rasgele sekillerde bolup,
pek cok yonden egitim / test icin kullaninca verinin herhangi bir
sekilde bizi yonlendirmesi / aldatmasi daha az mumkun hale gelir.

Ve iste bu ozelligi, ek olarak, capraz saglamayi "model secmek" icin
vazgecilmez bir arac haline getirir.

Model secmek nedir? Model secimi ustteki baglamda optimal bir
$\lambda$ bulmaktir mesela, yani her modeli temsil eden bir $\lambda$
var ise, en iyi $\lambda$'yi bulmak, en iyi modeli bulmak anlamina
geliyor, capraz saglama bunu sagliyor. Capraz saglama icin
\verb!scikit-learn!'un sagladigi fonksiyonlar vardir, once
katlari tanimlariz, sonra bu degistirilmis regresyon fonksiyonlarina
katlama usulunu geceriz.

\begin{minted}[fontsize=\footnotesize]{python}
k_fold = cross_validation.KFold(n=420, n_folds=7)
\end{minted}

Katlari ustteki gibi tanimladik. 420 tane veri noktasini 7 kata bol dedik.
Simdi bu katlari kullanalim,

\begin{minted}[fontsize=\footnotesize]{python}
ridge_cv = linear_model.RidgeCV(cv=k_fold)
ridge_cv.fit(np.array(diabetes_x), np.array(diabetes_y))
print ridge_cv.alpha_
\end{minted}

\begin{verbatim}
0.1
\end{verbatim} 

Ustteki sonuc $\lambda = 0.1$'i gosteriyor. Bu $\lambda$ daha optimalmis
demek ki. Lasso icin benzer sekilde

\begin{minted}[fontsize=\footnotesize]{python}
lasso_cv = linear_model.LassoCV(cv=k_fold)
print lasso_cv.fit(diabetes_x, diabetes_y)
\end{minted}

\begin{verbatim}
LassoCV(alphas=None, copy_X=True,
    cv=sklearn.cross_validation.KFold(n=420, n_folds=7), eps=0.001,
    fit_intercept=True, max_iter=1000, n_alphas=100, normalize=False,
    precompute=auto, tol=0.0001, verbose=False)
\end{verbatim}

\begin{minted}[fontsize=\footnotesize]{python}
print lasso_cv.alpha_
\end{minted}

\begin{verbatim}
0.00283958719118
\end{verbatim}

\begin{minted}[fontsize=\footnotesize]{python}
print lasso_cv.score(diabetes_x_test, diabetes_y_test) 
\end{minted}

\begin{verbatim}
0.597090337358
\end{verbatim}

Simdi veri setinin bir kismi uzerinde teker teker hangi algoritmanin
daha basarili oldugunu gorelim. 

\begin{minted}[fontsize=\footnotesize]{python}
def predict(row):
    j = row; i = row-1
    new_data = diabetes_x[i:j]
    print diabetes_y[i:j], "lasso",lasso_cv.predict(new_data), \
    	    "ridge",ridge_cv.predict(new_data), \
      	    "linear",lin.predict(new_data)	    

predict(-2) # sondan ikinci veri satiri
predict(-3)
predict(-4)
predict(-5)
predict(-8)
\end{minted}

\begin{verbatim}
439    132
Name: response, dtype: int64 lasso [ 122.2361344] ridge [ 127.1821212] linear [ 123.56604986]
438    104
Name: response, dtype: int64 lasso [ 101.85154189] ridge [ 108.89678818] linear [ 102.5713971]
437    178
Name: response, dtype: int64 lasso [ 192.95670241] ridge [ 189.58095011] linear [ 194.03798086]
436    48
Name: response, dtype: int64 lasso [ 52.8903924] ridge [ 57.66611598] linear [ 52.5445869]
433    72
Name: response, dtype: int64 lasso [ 60.42852107] ridge [ 66.3661042] linear [ 61.19831285]
\end{verbatim}

Ustteki sonuclara gore gercek degeri 132 olan 439. satirda lasso
122.2, sirt (ridge) 127.1, basit regresyon ise 123.5 bulmus. O veri
noktasi icin sirt yontemi daha basarili cikti.

Sonuclara bakinca bazen sirt, bazen normal regresyon basarili cikiyor.
Hangi yontem kazanmis o zaman? Bir o, o bir bu ondeyse, hangi yontemi
kullanacagimizi nasil bilecegiz?

Aslinda her seferinde tek bir metotu kullanmak gerekmiyor. Bu metotlari
bir takim (ensemble) halinde isletebiliriz. Her test noktasini, her seferinde
tum metotlara sorariz, gelen sonuclarin mesela.. ortalamasini aliriz. Bu
sekilde tek basina isleyen tum metotlardan tutarli olarak her seferinde daha
iyi sonuca ulasacak bir sonuc elde edebiliriz. Zaten Kaggle gibi yarismalarda
cogunlukla birinciligi kazanan metotlar bu turden takim yontemlerini kullanan
metotlar, mesela Netflix yarismasini kNN ve SVD metotlarini takim halinde
isleten bir grup kazandi. 

Kaynaklar

\url{www.lx.it.pt/~mtf/Figueiredo_Linear_Regression.pdf}

\url{www.cs.nyu.edu/~mohri/mls/lecture_8.pdf}

Harrington, P., {\em Machine Learning in Action}

Shalizi, C., Data Analysis from an Elementary Point of View

\end{document}

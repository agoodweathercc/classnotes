\documentclass[12pt,fleqn]{article}\usepackage{../common}
\begin{document}
Seriler

Bir guc serisi tek boyut icin 

\[ f(x) = \sum _{ n=0}^{\infty} a_n (x-c)^n\]

olarak gosterilir, $a_n$ katsayilari bilinmesi gereken katsayilardir. Cogu
durumda $c=0$'dir. O zaman 

\[ f(x) = \sum _{ n=0}^{\infty} a_n x_n^n = 1+x+x^2+x^3+..\]

diye gider. Herhangi bir polinom herhangi bir $c$ merkezi etrafinda rahat
bir sekilde bir guc serisi (power series) olarak temsil edilebilir
(muhakkak bu serinin cogu katsayisi sifir degerinde olacaktir). 

Unlu ustel baz $e^x$'in sayisinin acilimi, 

\[ e^x = 1 + x + \frac{1}{2}x^2 + \frac{1}{6}x^3 + ... \]

Ispat

Hatta ispattan once, ``bu seriyi kendimiz nasil kesfedebilirdik?'' diye
sormamiz lazim. $e^x$'in ozelligi nedir? Turevinin kendisine esit
olmasidir. O zaman oyle bir seri dusunelim ki turevini alinca kendisine
esiti olsun. Mesela

\[ 1 + x + x^2 + x^3 + ... \]

serisi ``neredeyse'' bu sarta uyuyor, cunku turevini alinca 

\[ 0 + 1 + 2x + 3x^2 + ... \]

Bu seri, $e^x$ acilimina benzer, ustel degerler dogru, ama katsayilar
tam uymuyor. Onu telafi edebiliriz. $2x$'i $2$ ile, $3x^2$'i $3$ ile, vs
bolersek, yani $n=0,1,2,..$ icin $n!$ ile bolersek, katsayilar da uyumlu
hale gelir, yani 

\[ e^x = \sum _{ n=0}^{\infty} \frac{ x^n}{n!} \]

\[ \square \]

Ustteki aslinda hem ispat hem de kesif amacli kullanilabilir. Bir sezgi ile
baslariz, ve olabilecek bir esitlikten bastaki formule erismeye ugrasiriz. 

[devami gelecek]

Kaynak

http://en.wikipedia.org/wiki/Power\_series

\end{document}

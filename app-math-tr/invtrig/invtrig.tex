\documentclass[12pt,fleqn]{article}\usepackage{../common}
\begin{document}
Ters Trigonometrik Formuller (Inverse Trigonometric Functions)

$\cos(x)$ icin $\cos^{-1}(x)$ ile, ya da $\arccos(x)$ ile gosterilen ters
trigonometrik formuldur. $\sin(x)$ ve $\tan(x)$ icin ayni sekilde. 

Bu ters fonksiyonlarin turevi nasil alinir? $\theta = \tan^{-1}(x)$ orneginde
gorelim. Elde etmek istedigimiz $d\theta/dx$. 

Eger

\[ \tan^{-1}(x) = \theta\]

ise, o zaman 

\[ \tan(\theta) = x \]

$x$'i aslinda $\theta$'ya bagli bir $x(\theta)$ fonksiyonu olarak gorebiliriz. 
Eger iki tarafin $\theta$'ya gore turevini alirsak

\[ \frac{dx}{d\theta} = \sec^{2}\theta \]

Bizim istedigimiz bunun tersi, o zaman bolumu tersine cevirelim

\[ \frac{d\theta}{dx} = \frac{1}{\sec^{2}\theta} \]

Pitagor Esitliklerinden bildigimize gore

\[ \sec^{2}(\theta) = \tan^{2}(\theta) + 1 \]

Yerine gecirelim

\[ \frac{d\theta}{dx} = \frac{1}{\tan^{2}\theta + 1} \]

Ilk basta tanimladigimiza gore $\tan(\theta) = x$, bunu da ustte yerine
koyalim

\[  = \frac{1}{x^2 + 1} \]



\end{document}

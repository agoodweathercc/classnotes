\documentclass[12pt,fleqn]{article}\usepackage{../common}
\begin{document}
Isi Denklemi

\[ \frac{\partial u}{\partial t} = \frac{\partial^2u}{\partial x^2} \]

olarak gosterilen denklem fizikte isi denklemi olarak bilinir, u
fonksiyonu iki degiskenlidir $u(x,t)$. Ornek icin bu denklemin
cozumunu tek boyutta gosterecegiz, yani bir genisligi onemli olmayan
bir demir cubugu uzerinde isinin dagilmasi konusuna bakacagiz, boyutu
temsil icin $x$ degiskeni kullanilacak. $t$ degiskeni zamani temsil
ediyor olacak. Baslangic sartlari (initial conditions) olarak isinin
t=0 aninda demir cubuk uzerinde $x$'e bagli bir sinus fonksiyonu ile
dagildigini farzedecegiz, sinir sartlari ise (boundary conditions)
cubugun iki ucunun sifir derecede tutulmasi olacak. Sonucta isinin
nereye gidecegini tahmin ederek te soyleyebiliriz -- isi demirin iki
ucundan kacarak tum cubuk boyunca sifir dereceye inecektir.

Ustteki denklem bir kismi diferansiyel denklemdir (partial
differential equation).

Matematiksel cozumler ya analitik, ya da yaklasiksal olur. Biz bu ornegi cozmek
icin yaklasiksal, hesapsal bir teknik kullanacagiz. Elimizde bir diferansiyel
denklem varsa cozum bulmak demek bir fonksiyon bulmak demektir, bir sayi degil;
yaklasiksal yontemle de oyle bir $u$ fonksiyonu bulacagiz ki, test / belli
noktalarda gercek fonksiyonla olabildigince ayni sonuclar verecek.

Cozumde sinirli farklar (finite differences) denen bir metot kullanilacak. Bu
yaklasiksal metotta calculus'un sonsuz ufakliklar icin kullanilan turevleri,
bildigimiz sayisal cikartma islemi uzerinden tanimlanan ``farkliliklara''
donusecekler. Mesela $d^2/dx^2$ nedir? $x$'e gore turevin turevidir, hesapsal
olarak ise farkin farkidir. Sonsuzluktan yaklasiga soyle geceriz: Eger $u_{j,i}$
bir 2 boyutlu dizin uzerinde $u$ fonksiyonunun sayisal degerlerini tasiyor
olsaydi, ve $j, i$ indis degerleri $t, x$'i temsil ediyorlar ise, $x$ uzerinden
birinci turev yani birinci fark (first difference) soyle olur:

\[ \frac{u_{j,i+1}-u_{j,i}}{h} \]

$h$ hangi degiskenin farkini aliyorsak, o farkin buyuklugunu
tanimlayan aralik degeridir, $h=\Delta x$, ve $u_{j,i+1} = u(t,x +
\Delta x)$.

Ikinci fark, farkin farkidir:

\[ \frac{1}{h}
\bigg[
\bigg( \frac{u_{j,i+1}-u_{j,i}}{h} \bigg) -
\bigg( \frac{u_{j,i}-u_{j,i-1}}{h} \bigg)
\bigg] 
 \]

\begin{equation} = \frac{u_{j,i+1}-2u_{j,i}+u_{j,i-1}}{h^2} \label{diff} \end{equation}

Bu carpimi tum $i$ degerleri icin ve matris uzerinden temsil etmenin yolu sudur:
Bir ikinci farkliliklar matrisi A yaratiriz:

\[ 
A = \frac{1}{\Delta x^2}
\left[ \begin{array}{ccccccc}
-2 & 1 & 0 & 0 \ldots 0 & 0 & 0 \\
1 & -2 & 1 & 0 \ldots 0 & 0 & 0 \\
\vdots & \vdots & \vdots & \vdots & \vdots & \vdots \\
0 & 0 & 0 & 0 \ldots 1 & -2 & 1 \\
0 & 0 & 0 & 0 \ldots 0 & 1 & -2
\end{array} \right]
 \]

Ve u degerlerinin bir vektor icine cekeriz:

\[ U_j =
\left[ \begin{array}{c}
u_{j,0} \\
u_{j,1} \\
u_{j,2} \\
\vdots \\
u_{j,n}
\end{array} \right]
 \]

$AU_j$ carpiminin \ref{diff} denklemindeki toplamlari her u icin teker teker
verecegini gorebiliriz. Indislerden $j$ zaman, $i$ mesafedir, yani ustteki
denklem simdilik sadece mesafeyi yani $x$'i parcalara bolmustur.

Zamani da modele dahil edelim ve cozumu elde etmeye ugrasalim. Isi
denkleminin tamamini simdiye kadar elde ettiklerimizi kullanarak ve
ayriksal olarak yazalim:

\begin{equation}
\frac{U_{j+1}-U_j}{\Delta t} = AU_j \label{main}
\end{equation}

$\frac{\partial^2u}{\partial x^2} \approx AU_j$, ve $\frac{\partial
  u}{\partial t} \approx (U_{j+1}-U_j) / \Delta t$ olarak
alindi. $U_j$ tanimindaki $j$ indisi zaman icin kullaniliyor, mesafe
yani $x$'i temsil eden indislerin tamami $U$'nun icinde var zaten.

Yaklasiksal tekniklerden Crank-Nicholson'a gore $AU_j$'i ardi ardina
iki zaman indisi uzerinden hesaplanan bir ortalama olarak temsil
edebiliriz, yani

\[ AU_j \approx \frac{1}{2}(AU_{j+1}+AU_j) \]

Niye bu acilim yapildi? Cunku elimizde $U_{j+1}$ ve $U_j$ degerleri var, bu
degerleri tekrar ortaya cikararak bir ``denklem sistemi'' yaratmis olacagiz, iki
bilinmeyen icin iki formul yanyana gelebilecek ve cozume erisilebilecek. 

Ustteki formulu \ref{main} denklemindeki $AU_j$ degerleri
icinkullanalim ve tekrar duzenleyelim.

\[ \frac{\Delta t}{2}AU_{j+1} + \frac{\Delta t}{2}AU_j = U_{i+1} - U_i  \]

\[ U_{i+1} - \frac{\Delta t}{2}AU_{j+1} = U_i + \frac{\Delta t}{2}AU_j  \]

\[ (I - \frac{\Delta t}{2}A) U_{j+1} = (I + \frac{\Delta t}{2}A)U_i \]

Artik bu formulu lineer cebirden bilinen $Ax=b$ formuna sokarak
cozebiliriz. Forma gore formulun sag tarafi $b$ olur, sol tarafta parantez ici A
olacak, $U_{j+1}$ ise bilinmeyen $x$ olacak (bizim $x$'ten farkli). Hesapsal
kodlar bir dongu icinde, her zaman dilimi icin bilinmeyen $U_{j+1}$ degerini
bulacak. Dongunun sonunda yeni $U_{j+1}$ eski $U_j$ olacak ve hesap devam
edecek. 

Sinir Sartlari

Her iki ucta $u$'nun sifir olma sarti uygulamali matematikte Dirichlet sinir
sarti olarak biliniyor. Bu sart $A$ matrisinin olusturulmasi sirasinda
kendiliginden olusuyor. Ufaltilmis bir D2 matrisi uzerinde gostermek gerekirse, 

\[ \left[ \begin{array}{ccccc}
1 & -2 & 1 & 0 & 0 \\
0 & 1 & -2 & 1 & 0 \\
0 & 0 & 1 & -2 & 1
\end{array} \right]
 \]

degerlerinin her satirinin \ref{diff} denklemini temsil ettigini
soylemistik. Eger sartlarimizdan biri $u_1$ ve $u_5$'un sifir olmasi ise, carpim
sirasinda ona tekabul eden D2'nin en soldaki ve en sagdaki kolonlarin tamamen
sifir yapmamiz yeterli olurdu, cunku carpim sirasinda $U_j$ icinde o kolonlar
$u_1$ ve $u_5$ ile carpilip onu sifir yaparlardi. O zaman yeni matris soyle
olurdu:

\[ 
\left[ \begin{array}{ccccc}
0 & -2 & 1 & 0 & 0 \\
0 & 1 & -2 & 1 & 0 \\
0 & 0 & 1 & -2 & 0
\end{array} \right]
 \]

Bu isler. Alternatif olarak sifir kolon yerine, o kolonlari tamamen matristen
atabilirdik, ayni sekilde $u$ degerlerini uretirken birinci ve sonuncu degerleri
de atmamiz gerekirdi, nasil olsa onlar``bilinmeyen'' degisken degiller. Bu yeni
matris soyle olurdu:

\[ \left[ \begin{array}{ccc}
-2 & 1 & 0  \\
1 & -2 & 1  \\
0 & 1 & -2 
\end{array} \right]
 \]

Alttaki kod icinde \verb!x = x[1:-1]! ibaresi $x$ ve dolayli olarak $u$'nun
ilk ve son degerlerini atmak icin kullanilmakta.

Kod

\lstinputlisting[language=Python]{hfdiff.py}

Ustteki \verb!data! degiskeni icinde zamana gore degisen tum u vektorleri
bulunabilir. Bu vektorleri grafik olarak gostermek icin alttaki kod eklenebilir:

\lstinputlisting[language=Python]{hfdiff2.py}

Python kodlari zip icinde hfdiff2.py dosyasindadir. Kullanilan
matrislerde bol sifir oldugu icin seyrek (sparse) matris teknikleri
kullanilabiliyor, sparse metotlari kullanan kodlar ise J. Wiens
kodlarindan alinan hfdiff.py icinde bulunabilir.

Kaynaklar

Wiens, J., http://www.jkwiens.com/2010/01/02/finite-difference-heat-equation-using-numpy

Stackoverflow, http://stackoverflow.com/questions/4843034/application-of-boundary-conditions-in-finite-difference-solution-for-the-heat-equ

Strang, G., Computational Science and Engineering

\end{document}


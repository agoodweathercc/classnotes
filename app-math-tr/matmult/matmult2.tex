\documentclass[12pt,fleqn]{article}\usepackage{../common}
\begin{document}
Matris �arp�m�

Matrix carpiminin tarifini lise derslerinden hatirlayabiliriz. Sol el
sol taraftaki matriste bir satir boyunca, sag el sagdaki matris
uzerinde kolon boyunca oge oge hareket ettirilir, ve bu hareket
sirasindaki ogeler carpilip, o carpimlar surekli toplanir. Sol ve sag
elin bir hareketi bittiginde, ele gecen tek bir sayi vardir, ve o sayi
uzerinden gecilen satir $i$ ve kolon $j$ icin sonuc matrisi, mesela
$C$'nin, $i$'inci satiri ve $j$'inci kolonuna yazilir.

Daha basit bir $Ax$ ornegine bakarsak, yani solda $A$ ve sagda $x$
var, carpim

$$
\begin{bmatrix}
1 & 1 & 6 \\
3 & 0 & 1 \\
1 & 1 & 4 \\
\end{bmatrix}
\begin{bmatrix}
2 \\
5 \\
0 \\
\end{bmatrix}
$$

Noktasal �arp�m Bak���

Bu carpimi bir kac sekilde gorebiliriz. Eger ustte tarif edilen gibi gorduysek,

$$
\begin{bmatrix}
1\cdot 2 + 1\cdot 5 + 6\cdot 0 \\
3\cdot 2 + 0\cdot 1 + 3\cdot 0 \\
1\cdot 2 + 1\cdot 5 + 4\cdot 0 
\end{bmatrix}
=
\begin{bmatrix}
7 \\
6 \\
7 \\
\end{bmatrix}
$$

Kolonsal Kombinasyon Bak���

Fakat matris carpimina bakmanin bir yolu daha var, hatta bu bakis acisinin daha
onemli bile oldugu soylenebilir, o da $A$'nin kolonlarinin kombine edilerek
saga sonuc olarak gecilmesi bakisidir. Buna gore

$$
2\cdot 
\begin{bmatrix}
1 \\
3 \\
1 \\
\end{bmatrix}
+
5\cdot 
\begin{bmatrix}
1 \\
0 \\
1 \\
\end{bmatrix}
+
0\cdot 
\begin{bmatrix}
6 \\
3 \\
4 \\
\end{bmatrix}
=
\begin{bmatrix}
7 \\
6 \\
7 \\
\end{bmatrix}
$$

Tabii burada ikinci "matris" aslinda bir vektor, ama o vektor de matris
olsaydi,

\begin{minted}{python}
A = np.array([[1 ,1 , 6],[3 , 0 , 1],[1 , 1 , 4]])
x = np.array([[2], [5], [0]])
print ('vektor ile\n')
print (np.dot(A,x))
B = np.array([[2, 2, 2],[5, 5, 5],[0, 0, 0]])
print ('\nmatris ile\n')
print (np.dot(A,B))
\end{minted}

\begin{verbatim}
vektor ile

[[7]
 [6]
 [7]]

matris ile

[[7 7 7]
 [6 6 6]
 [7 7 7]]
\end{verbatim}

\end{document}

\documentclass[12pt,fleqn]{article}\usepackage{../common}
\begin{document}
Icice Fonksiyonlar (Composite Functions)

\[ y = \frac{3}{2}x = \frac{1}{2}3x \]

bir icice fonksiyon olarak gorulebilir. 

\[ y = \frac{1}{2}u, \ u=3x \]

dersek, $y$ icindeki $u$ bir baska fonksiyon olabilir. Yani aslinda 

\[ y = f(u) \]

\[ u = g(x) \]

Yani

\[ y = f(g(x)) \]

Ustteki form bazen 

\[ y = f \circ g \]

olarak ta gosterilebiliyor. 

Zincirleme Kanunu (Icice Fonksiyonlar Icin)

Eger $f(u)$, $u=g(x)$ noktasinda, ve $g(x)$, $x$ noktasinda turevi
alinabilir durumda ise, o zaman icice fonksiyon $(f \circ g)(x) = f(g(x))$
$x$ noktasinda turevi alinabilir demektir, ve

\[ (f \circ g)'(x) = f'(g(x)) \cdot g'(x) \]

dogru olacaktir. Leibniz notasyonu ile 

\[ \frac{ dy}{dx} = \frac{ dy}{du} \cdot \frac{ du}{dx} \]

Ustteki formulu kesirlerin carpimi olarak gormek kismen dogru olabilir, en
azindan hatirlamak icin iyi, ama formel ispat baska sekilde yapiliyor,
detaylar icin [2] ve ``$dy/dx$ bir kesir olarak gorulebilir mi?'' yazisina
bakabilirsiniz.

Turev alirken $'$ isaretinin kullanilabilme sebebi fonksiyonda tek degisken
oldugu zaman neye gore turev alindiginin bariz olmasi.

Ornek 

Basta verilen ornek icin $dy/dx$' i bulun. 

\[ \frac{ dy}{dx} = \frac{ 3}{2}, \
\frac{dy}{du} = \frac{ 1}{2}, \
\frac{ du}{dx} = 3
 \]

\[ \frac{ dy}{dx} = \frac{ dy}{du} \cdot \frac{ du}{dx} \]

O zaman 

\[ \frac{ 1}{2} \cdot 3 = \frac{ 3}{2} \]


Kaynaklar 

[1] Thomas Calculus 11th Edition, pg 227

[2] Thomas Calculus 11th Edition, pg 191


\end{document}

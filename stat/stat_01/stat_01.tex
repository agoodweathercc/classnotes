\documentclass[12pt,fleqn]{article}
\setlength{\parindent}{0pt}
\usepackage{graphicx}
\usepackage{cancel}
\usepackage{listings}
\usepackage[latin5]{inputenc}
\usepackage{color}
\setlength{\parskip}{8pt}
\setlength{\parsep}{0pt}
\setlength{\headsep}{0pt}
\setlength{\topskip}{0pt}
\setlength{\topmargin}{0pt}
\setlength{\topsep}{0pt}
\setlength{\partopsep}{0pt}
\setlength{\mathindent}{0cm}
\usepackage{latexsym}
\usepackage{amsfonts}
\usepackage{showkeys}
\renewcommand*\showkeyslabelformat[1]{(#1)}

\begin{document}
Ders 1

Bu notlar makine ogrenimi, veri madenciligi gibi konularda gerekli olasilik
ve istatistik bilgisini paylasmak icin hazirlaniyor. Notlarda olasilik ve
istatistik ayni anda anlatilacak, ve uygulamalara agirlik verilecek. 

Orneklem Uzayi (Sample Space)

Orneklem uzayi $\Omega$ bir deneyin mumkun tum olasiliksal sonuclarin
(outcome) kumesidir. Eger deneyimiz ardi ardina iki kere yazi (T) tura (H)
atip sonucu kaydetmek ise, bu deneyin mumkun tum sonuclari soyledir

\[\Omega = \{HH,HT,TH,TT\} \]

Sonuclar ve Olaylar (Outcomes and Events)

$\Omega$ icindeki her nokta bir sonuctur (outcome). Olaylar $\Omega$'nin
herhangi bir alt kumesidir ve sonuclardan olusurlar. Mesela ustteki
yazi-tura deneyinde ``iki atisin icinden ilk atisin her zaman H gelmesi
olayi'' boyle bir alt kumedir, bu olaya $A$ diyelim, $A =
\{HH,HT\}$.

Ya da bir deneyin sonucu $\omega$ fiziksel bir olcum , diyelin ki sicaklik
olcumu. Sicaklik $\pm$, reel bir sayi olduguna gore, $\Omega = (-\infty,
+\infty)$, ve
sicaklik olcumunun 10'dan buyuk ama 23'ten kucuk ya da esit
olma ``olayi'' $A = (10,23]$. Koseli parantez kullanildi cunku sinir
degerini dahil ediyoruz. 

Ornek 

10 kere yazi-tura at. $A$ = ``en az bir tura gelme'' olayi olsun. $T_j$ ise
$j$'inci yazi-tura atisinda yazi gelme olayi olsun. $P(A)$ nedir? 

Bunun hesabi icin en kolayi, hic tura gelmeme, yani tamamen yazi gelme
olasiligini, $A^c$'yi hesaplamak, ve onu 1'den cikartmaktir. $^c$ sembolu
``tamamlayici (complement)'' kelimesinden geliyor.

\[ P(A) = 1 - P(A^c) \]

\[ = 1 - P(\textit{hepsi yazi}) \]

\[ = 1-P(T_1)P(T_2)...P(T_{10}) \]

\[ = 1 - \bigg(\frac{1}{2}\bigg)^{10} \approx .999 \]


Rasgele Degiskenler (Random Variables)

Bir rasgele degisken $X$ bir eslemedir, ki bu esleme $X: \Omega \to \mathbb{R}$
her sonuc ile bir reel sayi arasindaki eslemedir. 

Olasilik derslerinde bir noktadan sonra artik ornekleme uzayindan
bahsedilmez, ama bu kavramin arkalarda bir yerde her zaman devrede oldugunu
hic aklimizdan cikartmayalim. 

Ornek

10 kere yazi-tura attik diyelim. VE yine diyelim ki $X(\omega)$ rasgele
degiskeni her $\omega$ siralamasinda (sequence) olan tura sayisi. Iste bir
esleme. Mesela eger $\omega = HHTHHTHHTT$ ise $X(\omega) = 6$. Tura sayisi
eslemesi $\omega$ sonucunu 6 sayisina esledi. 

Ornek 

$\Omega = \{ (x,y); x^2+y^2 \le 1 \}$, yani kume birim cember ve icindeki
reel sayilar (unit disc). Diyelim ki bu kumeden rasgele secim
yapiyoruz. Tipik bir sonuc $\omega = (x,y)$'dir. Tipik rasgele degiskenler
ise $X(\omega) = x$, $Y(\omega) = y$, $Z(\omega) = x+y$ olabilir. Goruldugu
gibi bir sonuc ile reel sayi arasinda esleme var. $X$ rasgele degiskeni
bir sonucu $x$'e eslemis, yani $(x,y)$ icinden sadece $x$'i cekip
cikartmis. Benzer sekilde $Y,Z$ degiskenleri var. 

Toplamsal Dagilim Fonksiyonu (Cumulative Distribution Function -CDF-)

Tanim

$X$ rasgele degiskeninin CDF'i $F_X: \mathbb{R} \to [0,1]$ tanimi

\[ F_X(x) = P(X \ge x) \]

Eger $X$ ayriksal ise, yani sayilabilir bir kume $\{x_1,x_2,...\}$ icinden
degerler aliyorsa olasilik fonksiyonu (probability function), ya da
olasilik kutle fonksiyonu (probability mass function -PMF-) 

\[ f_X(x) = P(X = x) \]

Bazen $f_X$, ve $F_X$ yerine sadece $f$ ve $F$ yazariz. 

Tanim

Eger $X$ surekli (continuous) ise, yani tum $x$'ler icin $f_X(x) > 0$,
$\int_{-\infty}^{+\infty}f(x) dx = 1$ olacak sekilde bir $f_X$ mevcut ise, o zaman her $a \le b$ icin

\[ P(a<X<b) = \int_{a}^{b}f_X(x)dx \]

Bu durumda $f_X$ olasilik yogunluk fonksiyonudur (probability density function
-PDF-). 

\[ F_X = \int_{\infty}^{x}f_X(t)dt \]

Ayrica $F_X(x)$'in turevi alinabildigi her $x$ noktasinda  $f_X(x) = F'_X(x)$
demektir. 

Dikkat! Eger $X$ surekli ise o zaman $P(X = x) = 0$ degerindedir. $f(x)$
fonksiyonunu $P(X=x)$ olarak gormek hatalidir. Bu sadece ayriksal rasgele
degiskeninler icin isler. Surekli durumda olasilik hesabi icin belli iki
nokta arasinda entegral hesabi yapmamiz gereklidir. Ek olarak PDF 1'den
buyuk olabilir, ama PMF olamaz. PDF'in 1'den buyuk olabilmesi entegrali
bozmaz mi? Unutmayalim, entegral hesabi yapiyoruz, noktasal degerlerin 1
olmasi tum 1'lerin toplandigi anlamina gelmez. Bakiniz {\em Entegralleri
  Nasil Dusunelim} yazimiz.

Tanim

$X$ rasgele degiskeninin CDF'i $F$ olsun. Ters CDF (inverse cdf), ya da
ceyrek fonksiyonu (quantile function)

\[ F^{-1}(q) = inf \bigg\{ x: F(x) \le q \bigg\} \]

ki $q \in [0,1]$. Eger $F$ kesinlikle artan ve surekli bir fonksiyon ise
$F^{-1}(q)$ tekil bir $x$ sayisi ortaya cikarir, ki $F(x) = q$. 

Eger $inf$ kavramini bilmiyorsak simdilik onu minimum olarak
dusunebiliriz. 

$F^{-1}(1/4)$ birinci ceyrek

$F^{-1}(1/2)$ medyan (median, ya da ikinci ceyrek), 

$F^{-1}(3/4)$ ucuncu ceyrek 

olarak bilinir. 

Iki rasgele degisken $X$ ve $Y$ dagilimsal olarak birbirine esitligi, yani
$X \ \buildrel d \over = \ Y$ eger $F_X(x) = F_Y(x)$, $\forall x$. Bu $X,Y$ birbirine esit, birbirinin 
aynisi demek degildir. Bu degiskenler hakkindaki tum olasiliksal islemler, 
sonuclar ayni olacak demektir.

Uyari! ``$X$'in dagilimi $F$'tir'' beyanini $X \sim F$ seklinde yazmak bir
gelenek. Bu biraz kotu bir gelenek aslinda cunku $\sim$ sembolu ayni
zamanda yaklasiksallik kavramini belirtmek icin de kullaniliyor.

Bernoulli Dagilimi

$X$'in bir yazi-tura atisini temsil ettigini
dusunelim. O zaman $P(X = 1) = p$, ve $P(X = 0) = 1 - p$ olacaktir, 
ki $p
\in [0,1]$ olmak uzere. O zaman $X$'in dagilimi Bernoulli deriz, 
ve $X \sim
Bernoulli(p)$ diye gosteririz. Olasilik fonksiyonu 
$f(x) = p^x(1-p)^{(1-x)}$, $x \in \{0,1\}$. 

Yani $x$ ya 0, ya da 1. Parametre $p$, 0 ile 1 arasindaki herhangi bir reel 
sayi. 

Uyari! 

$X$ bir rasgele degisken; $x$ bu degiskenin alabilecegi spesifik bir deger;
$p$ degeri ise bir \textbf{parametre}, yani sabit, onceden belirlenmis reel
sayi. Tabii istatistiki problemlerde (olasilik problemlerinin tersi olarak
dusunursek) cogunlukla o sabit parametre bilinmez, onun veriden
hesaplanmasi, kestirilmesi gerekir. Her halukarda, cogu istatistiki modelde
rasgele degiskenler vardir, ve onlardan ayri olarak parametreler vardir. Bu
iki kavrami birbiriyle karistirmayalim.

Normal (Gaussian) Dagilim

$X \sim N(\mu, \sigma^2)$ ve PDF

\[ f(x) = \frac{1}{\sigma\sqrt{2\pi}} 
exp \bigg\{ - \frac{1}{2\sigma^2}(x-\mu)^2  \bigg\}
, \ x \in \mathbb{R}
\]

ki $\mu \in \mathbb{R}$ ve $\sigma > 0$ olacak sekilde.

Ileride gorecegiz ki $\mu$ bu dagilimin ``ortasi'', ve $\sigma$ onun
etrafa ne kadar ``yayildigi'' (spread). Normal dagilim olasilik ve
istatistikte cok onemli bir rol oynar. Dogadaki pek cok olay
yaklasiksal olarak Normal dagilima sahiptir. Sonra gorecegimiz uzere,
mesela bir rasgele degiskenin degerlerinin toplami her zaman Normal
dagilima yaklasir (Merkezi Limit Teorisi -Central Limit Theorem-). 

Eger $\mu = 0$ ve $\sigma = 1$ ise $X$'in standart Normal dagilim oldugunu
soyleriz. Gelenege gore standart Normal dagilim rasgele degiskeni $Z$ ile
gosterilmelidir, PDF ve CDF $\phi(z)$ ve $\Phi(z)$ olarak gosterilir. 

$\Phi(z)$'nin kapali form (closed-form) tanimi yoktur. Bu, matematikte
``analitik bir forma sahip degil'' demektir, formulu bulunamamaktadir,
bunun sebebi ise Normal PDF'in entegralinin analitik olarak alinamiyor
olusudur. 

Bazi faydali puf noktalari

1. Eger $X \sim N(\mu, \sigma^2)$ ise, o zaman $Z = (X-\mu) / \sigma \sim
N(0,1)$. 

2. Eger $Z \sim N(0,1)$ ise, o zaman $X = \mu + \sigma Z \sim N(\mu,\sigma^2)$

3. Eger $X_i \sim N(\mu_i, \sigma_i^2)$, $i=1,2,...$ ve her $X_i$
digerlerinden bagimsiz ise, o zaman 

\[ \sum_{i=1}^n X_i = N\bigg( \sum_{i=1}^n\mu_i, \sum_{i=1}^n\sigma^2 \bigg) \]

Tekrar $X \sim N(\mu, \sigma^2)$ alirsak ve 1. kuraldan devam edersek /
temel alirsak su da dogru olacaktir. 

\[ P(a < X < b) = ? \]

\[ 
= P\bigg(
\frac{a-\mu}{\sigma} < 
\frac{X-\mu}{\sigma} < 
\frac{b-\mu}{\sigma}
\bigg) 
\]

\[
= P\bigg(\frac{a-\mu}{\sigma} < Z < \frac{b-\mu}{\sigma}\bigg) 
= 
\Phi\bigg(\frac{b-\mu}{\sigma}\bigg) - 
\Phi\bigg(\frac{a-\mu}{\sigma}\bigg) 
\]

Ilk gecisi nasil elde ettik? Bir olasilik ifadesi $P(\cdot)$ icinde esitligin iki
tarafina ayni anda ayni toplama, cikarma operasyonlarini yapabiliriz. 

Son ifadenin anlami sudur. Eger standart Normal'in CDF'ini
hesaplayabiliyorsak, istedigimiz Normal olasilik hesabini yapabiliriz
demektir, cunku artik $X$ iceren bir hesabin $Z$'ye nasil tercume
edildigini goruyoruz. 

Tum istatistik yazilimlari $\Phi(z)$ ve $\Phi(z)^{-1}$ hesabi icin gerekli
rutinlere sahiptir. Tum istatistik kitaplarinda $\Phi(z)$'nin belli
degerlerini tasiyan bir tablo vardir.





\end{document}

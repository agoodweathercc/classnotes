\documentclass[12pt,fleqn]{article}
\setlength{\parindent}{0pt}
\usepackage{graphicx}
\usepackage{cancel}
\usepackage{listings}
\usepackage[latin5]{inputenc}
\usepackage{color}
\setlength{\parskip}{8pt}
\setlength{\parsep}{0pt}
\setlength{\headsep}{0pt}
\setlength{\topskip}{0pt}
\setlength{\topmargin}{0pt}
\setlength{\topsep}{0pt}
\setlength{\partopsep}{0pt}
\setlength{\mathindent}{0cm}

\begin{document}
Istatistik - Ders 1

Orneklem Uzayi (Sample Space)

Orneklem uzayi $\Omega$ bir deneyin mumkun tum olasiliksal sonuclarin
(outcome) kumesidir.

Rasgele Degiskenler (Random Variables)

Bir rasgele degisken $X$ bir eslemedir, ki bu esleme $X: \Omega \to \Re$
seklindedir, yani bir sonucu bir reel sayi esler. 


\end{document}

\documentclass[12pt,fleqn]{article}\usepackage{../common}
\begin{document}
Algilayicilardan Gelen Gauss Dagilimlarinin Fuzyonu (Gaussian Sensor Fusion)

Diyelim ki tek boyutlu ortamda bir buyuklugu mesela bir lokasyon bilgisi
$x$'i, iki kere olcuyoruz, ve bu olcumu iki degisik algilayiciya
yaptiriyoruz. Yani iki alet bir cismin oldugu yeri bize geri donduruyor, bu
bilgilerde belli olcude gurultu var ki bu belirsizlik aletler yuzunden
olabilir. Diyelim ki iki $z_1,z_2$ olcumu icin iki degisik belirsizlik
(uncertainty) $\sigma_1,\sigma_2$. 

Soru su, bu iki olcumu kullanarak daha iyi bir $x$ tahmini yapabilir miyiz?

Bunun iki olcumu bir sekilde birlestirmemiz gerekiyor. Her olcumu Gaussian
/ Normal dagilim olarak modelleyebiliriz, o zaman iki Normal dagilimi bir
sekilde birlestirmemiz (fusion) lazim. Olcumleri temsil etmek icin Gaussian
bicilmis kaftan. Olcumdeki belirsizligi standart sapma (standart deviation)
uzerinden rahatlikla temsil edebiliriz. Peki birlestirimi nasil yapalim? 

Bu tur problemlerde maksimum olurluk (maximum likelihood) kullanilmasi
gerektigini asagi yukari tahmin edebiliriz, cunku maksimum olurluk verinin
olurlugunu maksimize ederek bilinmeyen parametreleri tahmin etmeye
ugrasir. Cogunlukla bu teknigi hep {\em tek} bir dagilim baglaminda
gormusuzdur herhalde (ders kitaplarinda vs), o tek dagilima degisik veri
noktalari verilerek olasilik sonuclari carpilir, ve bu maksimize edilmeye
ugrasilir. Fakat maksimum olurluk illa tek bir dagilimla kullanilabilir
diye bir kural yok. Ustteki problemde iki olcumu iki Gaussian ile temsil
ederiz (buna mecburuz, iki degisik belirsizlik var), ve bu iki Gaussian'a
verilen iki olcum noktasini olurlugunu bu Gaussian'larin sonuclarini
carparak elde edebiliriz. Peki bilinmeyen $x$ nedir? Onu da Gaussian'in
orta noktasi (mean) olarak aliriz! Yani

$$ L(x) = p(z_1|x,\sigma_1) p(z_2|x,\sigma_2) $$

$$ L(x) \sim \exp{\frac{-(z_1-x)^2}{2\sigma_1^2} } 
\times \frac{-(z_2-x)^2}{2\sigma_2^2} $$

Ders notlari [1]'de bu formulun nasil turetilerek bir $x_{MLE}$ formulune
erisildigini gorebiliriz. Biz alternatif olarak daha temiz bir yoldan
gidecegiz. Her iki olcumu iki farkli 1 boyutlu Gaussian olarak temsil etmek
bir yontemdir. Diger bir yontem su olabilir. \textbf{2 boyutlu} bir
Gaussian yaratiriz, olcum belirsizliklerini bu Gaussianin kovaryansinda
capraza (diagonal) koyariz, diger matris ogeleri sifir olur, boylecek iki
olcumun birbirinden bagimsizligini temsil etmis oluruz.



[1] \url{www.robots.ox.ac.uk/~az/lectures/est/lect34.pdf}


\end{document}

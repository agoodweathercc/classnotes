%% This file was auto-generated by IPython.
%% Conversion from the original notebook file:
%% patent.ipynb
%%
\documentclass[11pt,english,fleqn]{article}

%% This is the automatic preamble used by IPython.  Note that it does *not*
%% include a documentclass declaration, that is added at runtime to the overall
%% document.

\usepackage{amsmath}
\usepackage{amssymb}
\usepackage{graphicx}
\usepackage{ucs}
\usepackage[utf8x]{inputenc}

% needed for markdown enumerations to work
\usepackage{enumerate}

% Slightly bigger margins than the latex defaults
\usepackage{geometry}
\geometry{verbose,tmargin=3cm,bmargin=3cm,lmargin=2.5cm,rmargin=2.5cm}

% Define a few colors for use in code, links and cell shading
\usepackage{color}
\definecolor{orange}{cmyk}{0,0.4,0.8,0.2}
\definecolor{darkorange}{rgb}{.71,0.21,0.01}
\definecolor{darkgreen}{rgb}{.12,.54,.11}
\definecolor{myteal}{rgb}{.26, .44, .56}
\definecolor{gray}{gray}{0.45}
\definecolor{lightgray}{gray}{.95}
\definecolor{mediumgray}{gray}{.8}
\definecolor{inputbackground}{rgb}{.95, .95, .85}
\definecolor{outputbackground}{rgb}{.95, .95, .95}
\definecolor{traceback}{rgb}{1, .95, .95}

% Framed environments for code cells (inputs, outputs, errors, ...).  The
% various uses of \unskip (or not) at the end were fine-tuned by hand, so don't
% randomly change them unless you're sure of the effect it will have.
\usepackage{framed}

% remove extraneous vertical space in boxes
\setlength\fboxsep{0pt}

% codecell is the whole input+output set of blocks that a Code cell can
% generate.

% TODO: unfortunately, it seems that using a framed codecell environment breaks
% the ability of the frames inside of it to be broken across pages.  This
% causes at least the problem of having lots of empty space at the bottom of
% pages as new frames are moved to the next page, and if a single frame is too
% long to fit on a page, will completely stop latex from compiling the
% document.  So unless we figure out a solution to this, we'll have to instead
% leave the codecell env. as empty.  I'm keeping the original codecell
% definition here (a thin vertical bar) for reference, in case we find a
% solution to the page break issue.

%% \newenvironment{codecell}{%
%%     \def\FrameCommand{\color{mediumgray} \vrule width 1pt \hspace{5pt}}%
%%    \MakeFramed{\vspace{-0.5em}}}
%%  {\unskip\endMakeFramed}

% For now, make this a no-op...
\newenvironment{codecell}{}

 \newenvironment{codeinput}{%
   \def\FrameCommand{\colorbox{inputbackground}}%
   \MakeFramed{\advance\hsize-\width \FrameRestore}}
 {\unskip\endMakeFramed}

\newenvironment{codeoutput}{%
   \def\FrameCommand{\colorbox{outputbackground}}%
   \vspace{-1.4em}
   \MakeFramed{\advance\hsize-\width \FrameRestore}}
 {\unskip\medskip\endMakeFramed}

\newenvironment{traceback}{%
   \def\FrameCommand{\colorbox{traceback}}%
   \MakeFramed{\advance\hsize-\width \FrameRestore}}
 {\endMakeFramed}

% Use and configure listings package for nicely formatted code
\usepackage{listingsutf8}
\lstset{
  language=python,
  inputencoding=utf8x,
  extendedchars=\true,
  aboveskip=\smallskipamount,
  belowskip=\smallskipamount,
  xleftmargin=2mm,
  breaklines=true,
  basicstyle=\small \ttfamily,
  showstringspaces=false,
  keywordstyle=\color{blue}\bfseries,
  commentstyle=\color{myteal},
  stringstyle=\color{darkgreen},
  identifierstyle=\color{darkorange},
  columns=fullflexible,  % tighter character kerning, like verb
}

% The hyperref package gives us a pdf with properly built
% internal navigation ('pdf bookmarks' for the table of contents,
% internal cross-reference links, web links for URLs, etc.)
\usepackage{hyperref}
\hypersetup{
  breaklinks=true,  % so long urls are correctly broken across lines
  colorlinks=true,
  urlcolor=blue,
  linkcolor=darkorange,
  citecolor=darkgreen,
  }

% hardcode size of all verbatim environments to be a bit smaller
\makeatletter 
\g@addto@macro\@verbatim\small\topsep=0.5em\partopsep=0pt
\makeatother 

% Prevent overflowing lines due to urls and other hard-to-break entities.
\sloppy

\setlength{\mathindent}{0pt}
\setlength{\parindent}{0pt}
\setlength{\parskip}{8pt}
\begin{document}

Hadoop ile Patent Verisi Islemek

75-99 yillari arasinda hangi patentin hangi hangi patentlere referans
verdigi ve patentler hakkinda detayli verileri Hadoop ile isleyecegiz.
Veriler alttaki baglantidan alinabilir, gerekli dosyalar Dosyalar
cite75\_99.txt ve apat63\_99.txt

http://www.nber.org/patents/

Referans verisine bakarsak,

\begin{codecell}
\begin{codeinput}
\begin{lstlisting}
!head -10 $HOME/Downloads/cite75_99.txt
\end{lstlisting}
\end{codeinput}
\begin{codeoutput}
\begin{verbatim}
"CITING","CITED"
3858241,956203
3858241,1324234
3858241,3398406
3858241,3557384
3858241,3634889
3858242,1515701
3858242,3319261
3858242,3668705
3858242,3707004
\end{verbatim}
\end{codeoutput}
\end{codecell}
Detayli patent verisine bakalim

\begin{codecell}
\begin{codeinput}
\begin{lstlisting}
!head -10 $HOME/Downloads/apat63_99.txt
\end{lstlisting}
\end{codeinput}
\begin{codeoutput}
\begin{verbatim}
"PATENT","GYEAR","GDATE","APPYEAR","COUNTRY","POSTATE","ASSIGNEE","ASSCODE","CLAIMS","NCLASS","CAT","SUBCAT","CMADE","CRECEIVE","RATIOCIT","GENERAL","ORIGINAL","FWDAPLAG","BCKGTLAG","SELFCTUB","SELFCTLB","SECDUPBD","SECDLWBD"
3070801,1963,1096,,"BE","",,1,,269,6,69,,1,,0,,,,,,,
3070802,1963,1096,,"US","TX",,1,,2,6,63,,0,,,,,,,,,
3070803,1963,1096,,"US","IL",,1,,2,6,63,,9,,0.3704,,,,,,,
3070804,1963,1096,,"US","OH",,1,,2,6,63,,3,,0.6667,,,,,,,
3070805,1963,1096,,"US","CA",,1,,2,6,63,,1,,0,,,,,,,
3070806,1963,1096,,"US","PA",,1,,2,6,63,,0,,,,,,,,,
3070807,1963,1096,,"US","OH",,1,,623,3,39,,3,,0.4444,,,,,,,
3070808,1963,1096,,"US","IA",,1,,623,3,39,,4,,0.375,,,,,,,
3070809,1963,1096,,"US","AZ",,1,,4,6,65,,0,,,,,,,,,
\end{verbatim}
\end{codeoutput}
\end{codecell}
Simdi patent detay verisinden bir orneklem (sample) alalim. Daha ufak
bir veri kumesiyle calismak ilk basta faydali olabilir, gelistirme test
etme surecini hizlandirir.

\begin{codecell}
\begin{codeinput}
\begin{lstlisting}
!chmod a+r $HOME/Downloads/apat63_99.txt
!head -1 $HOME/Downloads/apat63_99.txt > $HOME/Downloads/apat63_99_sampled.txt
!cat $HOME/Downloads/apat63_99.txt | perl -n -e 'print if (rand() < .05)' >> $HOME/Downloads/apat63_99_sampled.txt
\end{lstlisting}
\end{codeinput}
\end{codecell}
\begin{codecell}
\begin{codeinput}
\begin{lstlisting}
Hadoop baslatalim
\end{lstlisting}
\end{codeinput}
\end{codecell}
\begin{codecell}
\begin{codeinput}
\begin{lstlisting}
!ssh localhost -l hduser /home/hduser/Downloads/hadoop*/bin/stop-all.sh
!ssh localhost -l hduser /home/hduser/Downloads/hadoop*/bin/start-all.sh
\end{lstlisting}
\end{codeinput}
\begin{codeoutput}
\begin{verbatim}
no jobtracker to stop
\end{verbatim}
\begin{verbatim}
localhost: no tasktracker to stop
\end{verbatim}
\begin{verbatim}
no namenode to stop
\end{verbatim}
\begin{verbatim}
localhost: no datanode to stop
\end{verbatim}
\begin{verbatim}
localhost: no secondarynamenode to stop
\end{verbatim}
\begin{verbatim}
starting namenode, logging to /home/hduser/Downloads/hadoop-1.0.4/libexec/../logs/hadoop-hduser-namenode-burak-Aspire-S3.out
\end{verbatim}
\begin{verbatim}
localhost: starting datanode, logging to /home/hduser/Downloads/hadoop-1.0.4/libexec/../logs/hadoop-hduser-datanode-burak-Aspire-S3.out
\end{verbatim}
\begin{verbatim}
localhost: starting secondarynamenode, logging to /home/hduser/Downloads/hadoop-1.0.4/libexec/../logs/hadoop-hduser-secondarynamenode-burak-Aspire-S3.out
\end{verbatim}
\begin{verbatim}
starting jobtracker, logging to /home/hduser/Downloads/hadoop-1.0.4/libexec/../logs/hadoop-hduser-jobtracker-burak-Aspire-S3.out
\end{verbatim}
\begin{verbatim}
localhost: starting tasktracker, logging to /home/hduser/Downloads/hadoop-1.0.4/libexec/../logs/hadoop-hduser-tasktracker-burak-Aspire-S3.out
\end{verbatim}
\end{codeoutput}
\end{codecell}
\begin{codecell}
\begin{codeinput}
\begin{lstlisting}
/home/hduser/Downloads/hadoop*/bin/hadoop dfs -mkdir /user/hduser/patent
\end{lstlisting}
\end{codeinput}
\end{codecell}
\begin{codecell}
\begin{codeinput}
\begin{lstlisting}
!ssh localhost -l hduser /home/hduser/Downloads/hadoop*/bin/hadoop dfs -ls /user/hduser/patent
\end{lstlisting}
\end{codeinput}
\begin{codeoutput}
\begin{verbatim}
Found 2 items
-rw-r--r--   1 hduser supergroup  236903179 2013-02-21 14:16 /user/hduser/patent/apat63_99.txt
-rw-r--r--   1 hduser supergroup   11878646 2013-02-21 16:36 /user/hduser/patent/apat63_99_sampled.txt
\end{verbatim}
\end{codeoutput}
\end{codecell}
\begin{codecell}
\begin{codeinput}
\begin{lstlisting}
!ssh localhost -l hduser /home/hduser/Downloads/hadoop*/bin/hadoop dfs -copyFromLocal /home/burak/Downloads/apat63_99_sampled.txt /user/hduser/patent/apat63_99_sampled.txt
\end{lstlisting}
\end{codeinput}
\begin{codeoutput}
\begin{verbatim}
copyFromLocal: Target /user/hduser/patent/apat63_99_sampled.txt already exists
\end{verbatim}
\end{codeoutput}
\end{codecell}
Amacimiz patent verisindeki ulke (country) kodunu kullanarak her ulke
basina ortalama ne kadar patent uretildigini hesaplamak.
Esleme-Indirgeme (Map-Reduce) dongusunde esleme kismini yapacak program
asagida.

\begin{codecell}
\begin{codeinput}
\begin{lstlisting}
print open("pat2.py").read()
\end{lstlisting}
\end{codeinput}
\begin{codeoutput}
\begin{verbatim}
#!/usr/bin/python
import os,sys
os.environ['MPLCONFIGDIR']='/tmp' 
import pandas as pd
data = pd.read_csv(sys.stdin,sep=",",index_col=0,usecols=[0,4,8])
df = data[pd.notnull(data.ix[:,0]) & pd.notnull(data.ix[:,1])].ix[:,0:2]
df.to_csv(sys.stdout,sep="\t",index=False,header=False)
\end{verbatim}
\end{codeoutput}
\end{codecell}
\begin{codecell}
\begin{codeinput}
\begin{lstlisting}
!cp pat2.py /tmp/
!chmod a+r /tmp/pat2.py
!chmod a+x /tmp/pat2.py
\end{lstlisting}
\end{codeinput}
\end{codecell}
Indirgeyici yazmadan once, bu noktada, programimizi iki sekilde
isletelim. Bu sekillerden birisi hic indirgeyici olmadan, ikincisi
IdentityReducer denen kendisine gecilen veriyi oldugu gibi disari atan
(ama yine de ortaa bir indirgeyici oldugu icin sonradan bazi islemlerin
yine de yapilacagi) bir ortamda test yapmak.

\begin{codecell}
\begin{codeinput}
\begin{lstlisting}
!ssh localhost -l hduser /home/hduser/Downloads/hadoop*/bin/hadoop dfs -rmr /user/hduser/output
!ssh localhost -l hduser /home/hduser/Downloads/hadoop*/bin/hadoop  jar /home/hduser/Downloads/hadoop*/contrib/streaming/hadoop-*streaming*.jar -input patent/apat63_99_sampled.txt  -output output  -mapper /tmp/pat2.py -numReduceTasks 0 
\end{lstlisting}
\end{codeinput}
\begin{codeoutput}
\begin{verbatim}
Deleted hdfs://localhost:54310/user/hduser/output
\end{verbatim}
\begin{verbatim}
packageJobJar: [/app/hadoop/tmp/hadoop-unjar2555196345671652661/] [] /tmp/streamjob5013687273729997973.jar tmpDir=null
\end{verbatim}
\begin{verbatim}
13/02/24 16:30:26 INFO util.NativeCodeLoader: Loaded the native-hadoop library
13/02/24 16:30:26 WARN snappy.LoadSnappy: Snappy native library not loaded
13/02/24 16:30:26 INFO mapred.FileInputFormat: Total input paths to process : 1
\end{verbatim}
\begin{verbatim}
13/02/24 16:30:27 INFO streaming.StreamJob: getLocalDirs(): [/app/hadoop/tmp/mapred/local]
13/02/24 16:30:27 INFO streaming.StreamJob: Running job: job_201302241611_0012
13/02/24 16:30:27 INFO streaming.StreamJob: To kill this job, run:
13/02/24 16:30:27 INFO streaming.StreamJob: /home/hduser/Downloads/hadoop-1.0.4/libexec/../bin/hadoop job  -Dmapred.job.tracker=localhost:54311 -kill job_201302241611_0012
13/02/24 16:30:27 INFO streaming.StreamJob: Tracking URL: http://localhost:50030/jobdetails.jsp?jobid=job_201302241611_0012
\end{verbatim}
\begin{verbatim}
13/02/24 16:30:28 INFO streaming.StreamJob:  map 0%  reduce 0%
\end{verbatim}
\begin{verbatim}
13/02/24 16:30:43 INFO streaming.StreamJob:  map 100%  reduce 0%
\end{verbatim}
\begin{verbatim}
13/02/24 16:30:49 INFO streaming.StreamJob:  map 100%  reduce 100%
13/02/24 16:30:49 INFO streaming.StreamJob: Job complete: job_201302241611_0012
13/02/24 16:30:49 INFO streaming.StreamJob: Output: output
\end{verbatim}
\end{codeoutput}
\end{codecell}
\begin{codecell}
\begin{codeinput}
\begin{lstlisting}
!ssh localhost -l hduser /home/hduser/Downloads/hadoop*/bin/hadoop dfs  -copyToLocal output /tmp/
\end{lstlisting}
\end{codeinput}
\end{codecell}
\begin{codecell}
\begin{codeinput}
\begin{lstlisting}
!head -30 /tmp/output/part-00000
\end{lstlisting}
\end{codeinput}
\begin{codeoutput}
\begin{verbatim}
FR	12.0
US	5.0
US	1.0
US	4.0
US	4.0
US	21.0
US	4.0
US	8.0
US	7.0
US	11.0
DE	12.0
US	30.0
US	14.0
US	11.0
US	5.0
JP	21.0
US	23.0
US	5.0
CH	14.0
DE	11.0
US	4.0
US	14.0
US	4.0
US	1.0
US	4.0
IT	3.0
US	1.0
US	7.0
US	8.0
US	6.0
\end{verbatim}
\end{codeoutput}
\end{codecell}
\begin{codecell}
\begin{codeinput}
\begin{lstlisting}
Simdi <code>IdentityReducer</code> uzerinden. 
\end{lstlisting}
\end{codeinput}
\end{codecell}
\begin{codecell}
\begin{codeinput}
\begin{lstlisting}
!ssh localhost -l hduser /home/hduser/Downloads/hadoop*/bin/hadoop dfs -rmr /user/hduser/output
!ssh localhost -l hduser /home/hduser/Downloads/hadoop*/bin/hadoop  jar /home/hduser/Downloads/hadoop*/contrib/streaming/hadoop-*streaming*.jar -input patent/apat63_99_sampled.txt  -output output  -mapper /tmp/pat2.py -reducer org.apache.hadoop.mapred.lib.IdentityReducer -numReduceTasks 1 
\end{lstlisting}
\end{codeinput}
\begin{codeoutput}
\begin{verbatim}
Deleted hdfs://localhost:54310/user/hduser/output
\end{verbatim}
\begin{verbatim}
packageJobJar: [/app/hadoop/tmp/hadoop-unjar4791053218220591275/] [] /tmp/streamjob2130002507404697820.jar tmpDir=null
\end{verbatim}
\begin{verbatim}
13/02/24 16:29:31 INFO util.NativeCodeLoader: Loaded the native-hadoop library
13/02/24 16:29:31 WARN snappy.LoadSnappy: Snappy native library not loaded
13/02/24 16:29:31 INFO mapred.FileInputFormat: Total input paths to process : 1
\end{verbatim}
\begin{verbatim}
13/02/24 16:29:31 INFO streaming.StreamJob: getLocalDirs(): [/app/hadoop/tmp/mapred/local]
13/02/24 16:29:31 INFO streaming.StreamJob: Running job: job_201302241611_0011
13/02/24 16:29:31 INFO streaming.StreamJob: To kill this job, run:
13/02/24 16:29:31 INFO streaming.StreamJob: /home/hduser/Downloads/hadoop-1.0.4/libexec/../bin/hadoop job  -Dmapred.job.tracker=localhost:54311 -kill job_201302241611_0011
13/02/24 16:29:31 INFO streaming.StreamJob: Tracking URL: http://localhost:50030/jobdetails.jsp?jobid=job_201302241611_0011
\end{verbatim}
\begin{verbatim}
13/02/24 16:29:32 INFO streaming.StreamJob:  map 0%  reduce 0%
\end{verbatim}
\begin{verbatim}
13/02/24 16:29:45 INFO streaming.StreamJob:  map 50%  reduce 0%
\end{verbatim}
\begin{verbatim}
13/02/24 16:29:48 INFO streaming.StreamJob:  map 100%  reduce 0%
\end{verbatim}
\begin{verbatim}
13/02/24 16:29:57 INFO streaming.StreamJob:  map 100%  reduce 100%
\end{verbatim}
\begin{verbatim}
13/02/24 16:30:03 INFO streaming.StreamJob: Job complete: job_201302241611_0011
13/02/24 16:30:03 INFO streaming.StreamJob: Output: output
\end{verbatim}
\end{codeoutput}
\end{codecell}
\begin{codecell}
\begin{codeinput}
\begin{lstlisting}
!ssh localhost -l hduser /home/hduser/Downloads/hadoop*/bin/hadoop dfs  -copyToLocal output /tmp/
\end{lstlisting}
\end{codeinput}
\end{codecell}
\begin{codecell}
\begin{codeinput}
\begin{lstlisting}
!head -30 /tmp/output/part-00000
\end{lstlisting}
\end{codeinput}
\begin{codeoutput}
\begin{verbatim}
AE	12.0
AG	24.0
AN	15.0
AR	9.0
AR	16.0
AR	2.0
AR	19.0
AR	19.0
AR	11.0
AR	11.0
AR	4.0
AR	10.0
AR	6.0
AR	11.0
AR	8.0
AR	19.0
AR	22.0
AR	1.0
AR	5.0
AR	3.0
AR	10.0
AR	10.0
AR	7.0
AR	11.0
AR	24.0
AR	12.0
AR	3.0
AR	6.0
AT	16.0
AT	7.0
\end{verbatim}
\end{codeoutput}
\end{codecell}

\end{document}

\documentclass[12pt,fleqn]{article}\usepackage{../common}
\begin{document}
Guven Araliklari, Hipotez Testleri

Guven



Hipotez Testleri (Hypothesis Testing)

Hipotez testi (bir veriye dayanarak) farzedilen bir parametreyi bir
sabit degerle karsilastirmak, ya da iki parametreyi birbiriyle
karsilastirmak icin kullanilir. 

Bir hipotez testi, sonucta sadece iki cevap verebilecek bir sorudur;
bu sonuclar "reddetmek" ya da "reddetmemek" olabilir. Dikkat: bu
sonuclardan biri "kabul etmek" degil, bir istatistiki hipotezi kabul
etmek mumkun degildir. Tek soyleyebildigimiz "bir hipotezi reddetmek
icin elimizde yeterli veri olmadigini" soylemektir. Ama
reddedebiliyorsak, bu sonucta daha bir kesinlik vardir. 

Tek Orneklem t Testi (One-sample t test)

Bu test verinin Normal dagilimdan geldigini farzeder, tek orneklem
durumunda elde $x_1,...,x_n$ verisi vardir, ve bu veri $N(\mu,\Sigma)$
dagilimindan gelmistir ve test etmek istedigimiz hipotez /
karsilastirma $\mu = \mu_0$. 

\begin{minted}[fontsize=\footnotesize]{python}
from scipy.stats import ttest_1samp, wilcoxon, ttest_ind
import pandas as pd
daily_intake = np.array([5260,5470,5640,6180,6390,6515, 6805,7515,7515,8230,8770])
df = pd.DataFrame(daily_intake)
print df.describe()
\end{minted}

\begin{verbatim}
                 0
count    11.000000
mean   6753.636364
std    1142.123222
min    5260.000000
25%    5910.000000
50%    6515.000000
75%    7515.000000
max    8770.000000
\end{verbatim}

\begin{minted}[fontsize=\footnotesize]{python}
t_statistic, p_value = ttest_1samp(daily_intake, 7725)
print "one-sample t-test", p_value
\end{minted}

\begin{verbatim}
one-sample t-test 0.0181372351761
\end{verbatim}

Sonuc \verb!p_value! \verb!0.05!'ten kucuk cikti yani
yuzde 5 onemliligini (significance) baz aldik bu durumda veri
hipotezden onemli derecede (significantly) uzakta. Demek ki
ortalamanin 7725 oldugu hipotezini reddetmemiz gerekiyor.

Testi iki orneklemli kullanalim, gruplar 0/1 degerleri ile
isaretlendi, ve test etmek istedigimiz iki grubun ortalamasinin (mean)
ayni oldugu hipotezini test etmek. t-test bu arada varyansin ayni
oldugunu farzeder.

\begin{minted}[fontsize=\footnotesize]{python}
energ = np.array([
[9.21, 0],
[7.53, 1],
[7.48, 1],
[8.08, 1],
[8.09, 1],
[10.15, 1],
[8.40, 1],
[10.88, 1],
[6.13, 1],
[7.90, 1],
[11.51, 0],
[12.79, 0],
[7.05, 1],
[11.85, 0],
[9.97, 0],
[7.48, 1],
[8.79, 0],
[9.69, 0],
[9.68, 0],
[7.58, 1],
[9.19, 0],
[8.11, 1]])
group1 = energ[energ[:, 1] == 0][:, 0]
group2 = energ[energ[:, 1] == 1][:, 0]
t_statistic, p_value = ttest_ind(group1, group2)
print "two-sample t-test", p_value
\end{minted}

\begin{verbatim}
two-sample t-test 0.00079899821117
\end{verbatim}

$p-value < 0.05$ yani iki grubun ortalamasi ayni degildir. Ayni oldugu
hipotezi reddedildi.

Eslemeli t-Test (Paired t-test)

Eslemeli testler ayni deneysel birimin olcumu alindigi zaman
kullanilabilir, yani olcum alinan ayni grupta, deney sonrasi deneyin
etki edip etmedigi test edilebilir. Bunun icin ayni olcum deney
sonrasi bir daha alinir ve "farklarin ortalamasinin sifir oldugu"
hipotezi test edilebilir. Altta bir grup hastanin deney oncesi ve
sonrasi ne kadar yiyecek tukettigi listelenmis. 

\begin{minted}[fontsize=\footnotesize]{python}
intake = np.array([
[5260, 3910],
[5470, 4220],
[5640, 3885],
[6180, 5160],
[6390, 5645],
[6515, 4680],
[6805, 5265],
[7515, 5975],
[7515, 6790],
[8230, 6900],
[8770, 7335],
])
pre = intake[:, 0]
post = intake[:, 1]
t_statistic, p_value = ttest_1samp(post - pre, 0)
print "paired t-test", p_value
\end{minted}

\begin{verbatim}
paired t-test 3.05902094293e-07
\end{verbatim}

Wilcoxon isaretli-sirali testi (Wilcoxon signed-rank test)

t Testleri Normal dagilima gore sapmalari yakalamak acisindan,
ozellikle buyuk orneklemler var ise, oldukca saglamdir. Fakat bazen
verinin Normal dagilimdan geldigi faraziyesini yapmak istemeyebiliriz.
Bu durumda {\em dagilimdan bagimsiz metotlar} daha uygundur, bu tur
metotlar icin verinin yerine cogunlukla onun sira istatistiklerini
(order statistics) kullanir.

Tek orneklemli Wilcoxon testi icin prosedur $\mu_0$'i tum veriden
cikartmak ve geri kalan (farklari) isaretine bakmadan numerik degerine
gore siralamak, ve bu sira degerini bir kenara yazmak. Daha sonra geri
donup bu sefer cikartma islemi sonucunun isaretine bakmak, ve eksi
isareti tasiyan sira degerlerini toplamak, ayni islemi arti isareti
icin yapmak, ve eksi toplami arti toplamindan cikartmak. Sonucta
elimize bir istatistik $W$ gelecek. Bu test istatistigi aslinda $1..n$
tane sayi icinden herhangi birini $1/2$ olasiligiyla secmek, ve
sonuclari toplamaya tekabul etmektedir. Ve bu sonuc yine \verb!0.05!
ile karsilastirilir.

\begin{minted}[fontsize=\footnotesize]{python}
z_statistic, p_value = wilcoxon(daily_intake - 7725)
print "one-sample wilcoxon-test", p_value
\end{minted}

\begin{verbatim}
one-sample wilcoxon-test 0.0279991628713
\end{verbatim}

Hipotezi yine reddettik.

Ustte yaptigimiz eslemeli t-testi simdi Wilcoxon testi ile yapalim,

\begin{minted}[fontsize=\footnotesize]{python}
z_statistic, p_value = wilcoxon(post - pre)
print "paired wilcoxon-test", p_value
\end{minted}

\begin{verbatim}
paired wilcoxon-test 0.00463608893545
\end{verbatim}

Binom Testi

Binom dagilimi belli sayida "deney" icinde her seferinde $p$ olasiligi
tasiyan iki kategorili bir olaydan {\em kac tane} olabilecegini
modeller. Dagilim

$$
P(K = k) = \left( \begin{array}{ccc}
n \\ k
\end{array} \right)
p^k q^{n-k}
$$

olarak belirtilir, ki $q = 1-p$ degeridir. Bu dagilimin $n>20$, yani
"yeterince buyuk" degerleri icin Normal (Gaussian) Dagilima yaklastigi
/ onun gibi oldugu bilinmektedir. Bu yaklasilan dagilim ortalamasi
$np$ ve standart sapmasi $\sqrt{npq}$ olan bir Normal dagilim olacaktir, yani
$N(np,\sqrt{npq})$.

Devam edelim, madem elimizde bir normal dagilim var, bu normal dagilimi diger
her normal dagilim gibi standardize edebiliriz,

$$
Z = \frac{ K - np}{\sqrt{npq}} \sim N_{0,1}
$$

Burada rasgele degisken $K$, yani her binom deneyi ardindan ele
gececek basari sayisi burada. Binom testi icin test etmek istedigimiz
sey budur.  O zaman $p$ yerine test ettigimiz ana binom dagilimindan
gelen ana orani $\hat{p}$ kullaniriz, ki bu basari / tum deney sayisi
olarak hesaplanir, $K$ yeni deneydeki ele gecen basari sayisidir, n
ornekleminin buyuklugudur, bu sayilari yerine koyarak $Z$ dagilimindan
bir guven rakami (confidence) elde edebiliriz.

Bir ornek uzerinde gorelim: diyelim ki elimizde bir Web sitesinin
gunluk ziyaret, tiklama sayilarini gosteren bir veri seti var (CVR
ziyaretcilerin, sitedeki tiklayan musteriye "cevirme' orani,
-conversion-)

\begin{minted}[fontsize=\footnotesize]{python}
import pandas as pd
from scipy import stats
a = pd.DataFrame({'tiklama': [20.,2.,40.,5.,10.,100.],
                  'ziyaret': [100.,10.,300.,400.,30.,800.]})
a['cvr'] = a['tiklama'] / a['ziyaret'] 
print a
\end{minted}

\begin{verbatim}
   tiklama  ziyaret       cvr
0       20      100  0.200000
1        2       10  0.200000
2       40      300  0.133333
3        5      400  0.012500
4       10       30  0.333333
5      100      800  0.125000
\end{verbatim}

Diyelim ki bu veri seti icin cvr'in 0.16, yani yuzde 16 oldugunu
onceden biliyoruz. Ustteki basari orani binom dagili ile
modellenebilir, ziyaretler "deneylerdir", yani orneklem buyuklugunu
gosterirler. Tiklama ise basaridir,

\begin{minted}[fontsize=\footnotesize]{python}
p_hat = 0.16
btest = lambda x: (x['cvr']-p_hat) / np.sqrt( p_hat*(1-p_hat)/x['ziyaret'])
a['guven'] = a.apply(btest, axis=1)
a['guven'] = np.round(stats.zprob(a['guven'])*100,2)
print a
\end{minted}

\begin{verbatim}
   tiklama  ziyaret       cvr  guven
0       20      100  0.200000  86.24
1        2       10  0.200000  63.50
2       40      300  0.133333  10.39
3        5      400  0.012500   0.00
4       10       30  0.333333  99.52
5      100      800  0.125000   0.35
\end{verbatim}

Gaussian Kontrolu

Diyelim ki Gaussian dagilimina sahip oldugunu dusundugumuz $\{ x_i\}$
verilerimiz var. Bu verilerin Gaussian dagilimina uyup uymadigini nasil
kontrol edecegiz? Normal bir dagilimin her veri noktasi icin soyle temsil
edebiliriz,

$$ y_i = \Phi\bigg(\frac{ x_i - \mu}{\sigma}\bigg) $$

Burada $\Phi$ standart Gaussian'i temsil ediyor (detaylar icin
*Istatistik Ders 1*) ve CDF fonksiyonuna tekabul ediyor. CDF
fonksiyonunun ayni zamanda ceyregi (quantile) hesapladigi soylenir,
aslinda CDF son derece detayli bir olasilik degeri verir fakat evet,
dolayli yoldan noktanin hangi ceyrek icine dustugu de gorulecektir.

Simdi bir numara yapalim, iki tarafa ters Gaussian formulunu uygulayalim,
yani $\Phi^{-1}$.

$$ \Phi^{-1}(y_i) = \Phi^{-1}\bigg( \Phi\bigg(\frac{ x_i - \mu}{\sigma}\bigg)\bigg) $$

$$ \Phi^{-1}(y_i) = \frac{ x_i - \mu}{\sigma}$$

$$ x_i = \Phi^{-1}(y_i) \sigma + \mu  $$ 

Bu demektir ki elimizdeki verileri $\Phi^{-1}(y_i)$ bazinda grafiklersek,
bu noktalar egimi $\sigma$, baslangici (intercept) $\mu$ olan bir duz cizgi
olmalidir. Eger kabaca noktalar duz cizgi olusturmuyorsa, verimizin 
Gaussian dagilima sahip olmadigina karar verebiliriz. 

Ustte tarif edilen grafik,  olasilik grafigi (probability plot) olarak
bilinir. 

Ters Gaussian teorik fonksiyonunu burada vermeyecegiz, Scipy
\verb!scipy.stats.invgauss! hesaplar icin kullanilabilir. Fakat $y_i$'nin
kendisi nereden geliyor? Eger $y_i$, CDF'in bir sonucu ise, pur veriye
bakarak bir CDF degeri de hesaplayabilmemiz gerekir. Bunu yapmak icin bir
baska numara lazim. 

1. Eldeki sayilari artan sekilde siralayin

2. Her veri noktasina bir derece (rank) atayin (siralama sonrasi hangi
seviyede oldugu yeterli, 1'den baslayarak). 

3. Ceyrek degeri $y_i$ bu sira / $n+1$, $n$ eldeki verinin buyuklugu. 

Bu teknik niye isliyor? $x$'in CDF'i $x_i < x$ sartina uyan $x_i$'lerin
orani degil midir? Yani bir siralama soz konusu ve ustteki teknik te bu
siralamayi biz elle yapmis olduk, ve bu siralamadan gereken bilgiyi aldik. 


[1] Introductory Statistics with R

[2] Introduction to Probability and Statistics Using R

[3] \verb!https://gist.github.com/mblondel/1761714!

\end{document}


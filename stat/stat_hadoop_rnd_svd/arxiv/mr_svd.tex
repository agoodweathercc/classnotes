\documentclass[11pt]{article} 

\usepackage[latin5]{inputenc}
\usepackage{amsmath}
\usepackage{examplep}
\usepackage{graphicx}
\usepackage[ruled]{algorithm2e}
\SetKw{Emit}{emit}

\begin{document}
\title{SVD Factorization for Tall-and-fat Matrices on Map/Reduce
  Architectures} \author{Burak Bayraml�} \date{October 15, 2013}

\maketitle

\begin{abstract}
  We demonstrate an implementation for an approximate rank-k SVD
  factorization, combining well-known randomized projection techniques with
  previously implemented map/reduce solutions in order to compute steps of
  the random projection based SVD procedure, such QR and SVD. We structure
  the problem in a way that it reduces to Cholesky and SVD factorizations
  on $k \times k$ matrices computed on a single machine, greatly easing the
  computability of the problem.
\end{abstract}

\section{Introduction} \label{intro}

\cite{gleich} presents many excellent techniques for utilizing map/reduce
architectures to compute QR and SVD for the so-called tall-and-skinny
matrices. The idea is based on the fact that QR factorization can be turned
into an $A^TA$ computation problem computer in parallel, en masse using
map/reduce, and through this to a Cholesky decomposition performed on a
single machine. Since

$$ A^TA = (QR)^T(QR) = R^TQ^TQR = R^TR $$

and because Cholesky factorization of an $n \times n$ symmetric positive
definite matrix is

$$ A = LL^T $$

where $L$ is an $n \times n$ lower triangular matrix, and R is upper
triangular, we can conclude if we factorize $A$ into $L$ and $L^T$, this
implies $LL^T = RR^T$, we have a method of calculating $R$ of QR using
Cholesky factorization on $A^TA$. The key observation here is $A^TA$
computation results an $n \times n$ matrix and if $A$ is ``skinny'' then
$n$ is relatively small (in the thousands), and Cholesky decomposition can
be executed on a small $n \times n$ matrix on a single
computer. $Q$ is computed simply as $Q = AR^{-1}$. This again is relatively
cheap because R is $n \times n$, the inverse is computed locallly, matrix
multiplication with $A$ can be performed through map/reduce.

SVD is an additional step. SVD decomposition is 

$$ A = U \Sigma V^T $$

If we expand it with $A = QR$

$$ QR =  U \Sigma V^T $$

$$ R =  Q^T U \Sigma V^T $$

Let's call $\tilde{U} = Q^T U$

$$ R =  \tilde{U} \Sigma V^T $$

This means if we run a local SVD on $R$ (we just calculated above with
Cholesky) which is an $n \times n$ matrix, we will have calculated
$\tilde{U}$, the real $\Sigma$, and real $V^T$. 

Now we have a map/reduce way of calculating QR and SVD on $m \times n$
matrices where $n$ is small.

\subsection{Approximate rank-k SVD}

Switching gears, we look at another method for calculating SVD. The
motivation is computing SVD if $n$ is large, creating a ``fat'' matrix
which might have columns in the billions would require reducing the
dimensionality of the problem. According to \cite{halko}, one way to
achieve is through random projection. First we draw an $n \times k$
Gaussian random matrix $\Omega$. Then we calculate

$$ Y = A \Omega $$

We perform QR decomposition on $Y$

$$ Y = QR $$

Then form $k \times n$ matrix

$$ B = Q^T A $$

Then we can calculate SVD on this small matrix

$$ B = \hat{U} \Sigma V^T $$

Then form the matrix 

$$ U = Q \hat{U} $$

The main idea is based on

$$ A = QQ^T A $$

if replace $Q$ which comes from random projection $Y$, 

$$ A \approx \tilde{Q}\tilde{Q}^T A $$

$Q$ and $R$ of the projection are close to that of $A$. In the
multiplication above $R$ is called $B$ where $B = \tilde{Q}^T A $, and,

$$ A \approx \tilde{Q}B $$

then, as in \cite{gleich}, we can take SVD of $B$ and apply the same
transition rules to obtain an approximate $U$ of $A$.

This approximation works because of the fact that projecting points to a
random subspace preserves distances between points, or in detail,
projecting the n-point subset onto a random subspace of $O(\log
n/\epsilon^2)$ dimensions only changes the interpoint distances by $(1 \pm
\epsilon)$ with positive probability \cite{gupta}. It is also said that $Y$
is a good representation of the span of $A$.

\subsection{Combining Both Methods}

What if $n$ is also very large? In this case local Cholesky or SVD
computations would take a long time as well. Our idea was using approximate
k-rank SVD where $k << n$, before map/reduce based QR and SVD methods
presented in section~\ref{intro}, to reduce dimension before using
map/reduce methods presented in Section~\ref{intro}, this way, we are again
able to work with small matrices locally, $k \times k$ this time on which
Cholesky, SVD can be performed. Below we outline each map/reduce
job. 

\begin{algorithm}[H]
  \DontPrintSemicolon  
  \SetKwInOut{Input}{input}
  \SetKwInOut{Output}{output}
  \SetKwProg{map}{function}{}{}
  \Input{A}
  \Output{Y}
  \map{MAP{(key, value)}}{
    Tokenize $value$ and pick out id value pairs\;
    result $\leftarrow$ zeros(1,$k$)\;
    \For{each $j^{th}$ $token$ $\in value$}{
      Initialize seed with $j$\;
      r $\leftarrow$ generate $k$ random numbers\;
      result $\leftarrow$ result + $r \cdot token[j]$
    }
    \Emit{key, result}
  }
  \SetKwProg{reduce}{function}{}{}
  \reduce{REDUCE{(key, value)}}{
    noop\;}
  \caption{Random Projection Job}
\end{algorithm} 

Each value of $A$ will arrive to the algorithm as a key and value pair. Key
is line number or other identifier per row of $A$. Value is a collection of
id value pairs where id is column id this time, and value is the value for
that column. Sparsity is handled through this format, if an id for a column
does not appear in a row of A, it is assumed to be zero. The resulting $Y$
matrix has dimensions $m \times k$.

\begin{algorithm}[H]
  \DontPrintSemicolon  
  \SetKwInOut{Input}{input}
  \SetKwInOut{Output}{output}
  \SetKwProg{map}{function}{}{}
  \Input{Y}
  \Output{R}
  \map{MAP{(key $k$, val $a$)}}{
    \For{$i,row$ in enumerate($a^Ta$)}{
      \Emit{i, row}
    }
  }
  \SetKwProg{reduce}{function}{}{}
  \reduce{REDUCE{(key, value)}}{
    \Emit (k,sum($<v_j^k>$)\;}
  \SetKwProg{reduce}{function}{}{}
  \reduce{FINAL LOCAL REDUCE {(key, value)}}{
    result $\leftarrow$ Cholesky($A_{sum}$)\;
    \Emit (result)\;}
  \caption{$A^TA$ Cholesky Job}
\end{algorithm} 

The FINAL\_LOCAL\_REDUCE step is a function provided in most map/reduce
frameworks, it is a central point that collects the output of all reducers,
naturally a single machine which makes it ideal to execute the final
Cholesky call on a $k \times k$ matrix. The output is $R$.

\begin{algorithm}[H]
  \DontPrintSemicolon  
  \SetKwInOut{Input}{input}
  \SetKwInOut{Output}{output}
  \SetKwProg{map}{function}{}{}
  \SetKwProg{initialize}{function}{}{}
  \Input{Y,R}
  \Output{Q}
  \initialize{INIT{()}}{ 
    $R_{inv} = R^{-1}$ 
  }
  \map{MAP{(key, value)}}{
    \For{$row$ in $Y$}{
      \Emit{(key, $row \cdot R_{inv}$)}
    }
  }
  \caption{$Q$ Job}
\end{algorithm} 

There is no reducer in the Q Job, it is a very simple job, it merely
computes multiplication between row of $Y$ and a local matrix $R$. Matrix
$R$ is very small, $k \times k$, hence it can be kept locally in every
node. The $INIT$ function is used to store the inverse of $R$ locally, once
the mapper is initialized, it will always use the same $R^{-1}$ for every
multiplication. 

\begin{algorithm}[H]
  \DontPrintSemicolon  
  \SetKwInOut{Input}{input}
  \SetKwInOut{Output}{output}
  \Input{A,Q}
  \Output{$B^T$}
  \SetKwProg{reduce}{function}{}{}
  \reduce{REDUCE {(key, value)}}{
    \For{$row$ in $value$}{
      \If{$row$ is from $A$}{ 
        $left=row$
      }
      \If{$row$ is from $Q$}{ 
        $right=row$
      }
    }
    \For{nonzero $j^{th}$ $cell$ in $left$}{
      \Emit{$j$, $left[j] \cdot right$ }
    }
  }
  \SetKwProg{reducesum}{function}{}{}
  \reducesum{REDUCESUM {(key, value)}}{
    result $\leftarrow$ zeros(1,$k$) \;
    \For{$row$ in $value$}{
      result $\leftarrow$ result + $row$
     }
     \Emit{key, result}
  }
  \caption{$A^TQ$ Job}
\end{algorithm} 

The job above takes $A$ and $Q$ matrices at the same time. Both of these
matrices are based on the same key (line number, of preexisting id) and we
need to join them first. If a mapper is a pass-through mapper, in other
words if it does not exist, it is assumed to simply re-emit the key and
value, which will, indirectly force matching rows with the same id from $A$
and $Q$ to be sent to the first reducer. Then for each unique id, the first
reducer gets exactly two rows. This is an indirect way of performing a join
in map/reduce environment.

Then in the reducer we deduce if the row is a $Q$ row or an $A$ row. We
need this information because we will iterate cells of $A$ one by one,
which is assumed to be sparse. In this iteration for each $j^{th}$ non-zero
cell of $A$, we multiply the cell's value with the row from $Q$ and emit
the multiplication result with key $j$.

\begin{thebibliography}{1}

\bibitem{gleich}
Gleich, Benson, Demmel, \emph{Direct QR factorizations for tall-and-skinny matrices in MapReduce architectures}

\bibitem{halko}
N.~Halko, \emph{Randomized methods for computing low-rank approximations of matrices}

\bibitem{gupta}
S.~Dangupta, A.~Gupta \emph{An Elementary Proof of a Theorem of Johnson and Lindenstrauss}

\end{thebibliography}

\end{document}

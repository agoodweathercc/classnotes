\documentclass[11pt]{article} 

\usepackage[latin5]{inputenc}
\usepackage{amsmath}
\usepackage{examplep}
\usepackage{graphicx}

\begin{document}
\title{SVD Factorization for Tall and Fat Matrices on Map/Reduce Architectures}
\author{Burak Bayraml�}
\date{October 15, 2013}

\maketitle

\begin{abstract}
  We demonstrate an implementation for an approximate rank-k SVD
  factorization combining well-known randomized projection techniques with
  previously implemented map/reduce solutions in order to compute various
  steps of this procedure, such as local QR and local SVD implementations
  that can run on a single machine. We structure the problem in a way
  reduces to single machine Cholesky and SVD factorizations on $k \times k$
  matrices, thereby greatly easing the computability of the problem.
\end{abstract}

\section{Introduction}

\cite{gleich} presents many excellent techniques for utilizing map/reduce
architectures to compute QR and SVD for the so-called tall-and-skinny
matrices. The ideas are based on the fact that QR factorization can be
turned into an $A^TA$ computation problem which is easy to compute using
map/reduce. First idea is,

$$ A^TA = (QR)^T(QR) = R^TQ^TQR = R^TR $$

Then, we take a look at Cholesky factorization of an $n \times n$ symmetric
positive definite matrix which is

$$ A = LL^T $$

where $L$ is an $n \times n$ lower triangular matrix.  R is upper
triangular. Then if we factorize $A$ into $L$ and $L^T$, and $LL^T = RR^T$,
we have a method of calculating QR using Cholesky factorization on
$A^TA$. The key observation here is that after $A^TA$ computation is
completed we will have an $n \times n$ matrix and if $A$ is ``skinny'' then
$n$ is relatively small (in the thousands), and Cholesky decomposition can
be executed on this small matrix on a single computer. We can calculate
SVD based on QR. SVD decomposition is represented as

$$ A = U \Sigma V^T $$

Expand it with $A = QR$

$$ QR =  U \Sigma V^T $$

$$ R =  Q^T U \Sigma V^T $$

Let's call $\tilde{U} = Q^T U$

$$ R =  \tilde{U} \Sigma V^T $$

This means if we run a local SVD on $R$ (we just calculated above with
Cholesky) which is an $n \times n$ matrix, we will have calculated
$\tilde{U}$, and the real $\Sigma$, and real $V^T$. Hence we have a
map/reduce way of calculating QR and SVD on $m \times n$ matrices where $n$
is small.

\subsection{Approximate rank-k SVD}

Computing SVD with large $n$ which are ``fat'' that might have columns in
the billions would require reducing the dimensionality of the problem. According
to \cite{halko}, one way to achieve is through random projection. First we
draw an $n \times k$ Gaussian random matrix $\Omega$. Then we calculate

$$ Y = A \Omega $$

We perform QR decomposition on $Y$

$$ Y = QR $$

Then form $k \times n$ matrix

$$ B = Q^T A $$

Then we can calculate SVD on this small matrix

$$ B = \hat{U} \Sigma V^T $$

Then form the matrix 

$$ U = Q \hat{U} $$

As \cite{halko} shows $U$ is approximately close the real $U$ of $A = U
\Sigma V^T$. 

\begin{thebibliography}{1}

\bibitem{gleich}
Gleich, Benson, Demmel, \emph{Direct QR factorizations for tall-and-skinny matrices in MapReduce architectures}

\bibitem{halko}
N.~Halko, \emph{Randomized methods for computing low-rank approximations of matrices}

\end{thebibliography}

\end{document}

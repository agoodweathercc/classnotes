%% This file was auto-generated by IPython.
%% Conversion from the original notebook file:
%% stat_06.ipynb
%%
\documentclass[11pt,english,fleqn]{article}

%% This is the automatic preamble used by IPython.  Note that it does *not*
%% include a documentclass declaration, that is added at runtime to the overall
%% document.

\usepackage{amsmath}
\usepackage{amssymb}
\usepackage{graphicx}
\usepackage{ucs}
\usepackage[utf8x]{inputenc}

% needed for markdown enumerations to work
\usepackage{enumerate}

% Slightly bigger margins than the latex defaults
\usepackage{geometry}
\geometry{verbose,tmargin=3cm,bmargin=3cm,lmargin=2.5cm,rmargin=2.5cm}

% Define a few colors for use in code, links and cell shading
\usepackage{color}
\definecolor{orange}{cmyk}{0,0.4,0.8,0.2}
\definecolor{darkorange}{rgb}{.71,0.21,0.01}
\definecolor{darkgreen}{rgb}{.12,.54,.11}
\definecolor{myteal}{rgb}{.26, .44, .56}
\definecolor{gray}{gray}{0.45}
\definecolor{lightgray}{gray}{.95}
\definecolor{mediumgray}{gray}{.8}
\definecolor{inputbackground}{rgb}{.95, .95, .85}
\definecolor{outputbackground}{rgb}{.95, .95, .95}
\definecolor{traceback}{rgb}{1, .95, .95}

% Framed environments for code cells (inputs, outputs, errors, ...).  The
% various uses of \unskip (or not) at the end were fine-tuned by hand, so don't
% randomly change them unless you're sure of the effect it will have.
\usepackage{framed}

% remove extraneous vertical space in boxes
\setlength\fboxsep{0pt}

% codecell is the whole input+output set of blocks that a Code cell can
% generate.

% TODO: unfortunately, it seems that using a framed codecell environment breaks
% the ability of the frames inside of it to be broken across pages.  This
% causes at least the problem of having lots of empty space at the bottom of
% pages as new frames are moved to the next page, and if a single frame is too
% long to fit on a page, will completely stop latex from compiling the
% document.  So unless we figure out a solution to this, we'll have to instead
% leave the codecell env. as empty.  I'm keeping the original codecell
% definition here (a thin vertical bar) for reference, in case we find a
% solution to the page break issue.

%% \newenvironment{codecell}{%
%%     \def\FrameCommand{\color{mediumgray} \vrule width 1pt \hspace{5pt}}%
%%    \MakeFramed{\vspace{-0.5em}}}
%%  {\unskip\endMakeFramed}

% For now, make this a no-op...
\newenvironment{codecell}{}

 \newenvironment{codeinput}{%
   \def\FrameCommand{\colorbox{inputbackground}}%
   \MakeFramed{\advance\hsize-\width \FrameRestore}}
 {\unskip\endMakeFramed}

\newenvironment{codeoutput}{%
   \def\FrameCommand{\colorbox{outputbackground}}%
   \vspace{-1.4em}
   \MakeFramed{\advance\hsize-\width \FrameRestore}}
 {\unskip\medskip\endMakeFramed}

\newenvironment{traceback}{%
   \def\FrameCommand{\colorbox{traceback}}%
   \MakeFramed{\advance\hsize-\width \FrameRestore}}
 {\endMakeFramed}

% Use and configure listings package for nicely formatted code
\usepackage{listingsutf8}
\lstset{
  language=python,
  inputencoding=utf8x,
  extendedchars=\true,
  aboveskip=\smallskipamount,
  belowskip=\smallskipamount,
  xleftmargin=2mm,
  breaklines=true,
  basicstyle=\small \ttfamily,
  showstringspaces=false,
  keywordstyle=\color{blue}\bfseries,
  commentstyle=\color{myteal},
  stringstyle=\color{darkgreen},
  identifierstyle=\color{darkorange},
  columns=fullflexible,  % tighter character kerning, like verb
}

% The hyperref package gives us a pdf with properly built
% internal navigation ('pdf bookmarks' for the table of contents,
% internal cross-reference links, web links for URLs, etc.)
\usepackage{hyperref}
\hypersetup{
  breaklinks=true,  % so long urls are correctly broken across lines
  colorlinks=true,
  urlcolor=blue,
  linkcolor=darkorange,
  citecolor=darkgreen,
  }

% hardcode size of all verbatim environments to be a bit smaller
\makeatletter 
\g@addto@macro\@verbatim\small\topsep=0.5em\partopsep=0pt
\makeatother 

% Prevent overflowing lines due to urls and other hard-to-break entities.
\sloppy

\setlength{\mathindent}{0pt}
\setlength{\parindent}{0pt}
\setlength{\parskip}{8pt}
\begin{document}

Ders 5 - Hipotez Testleri (Hypothesis Testing)

Hipotez testi (bir veriye dayanarak) farzedilen bir parametreyi bir
sabit degerle karsilastirmak, ya da iki parametreyi birbiriyle
karsilastirmak icin kullanilir.

Bir hipotez testi, sonucta sadece iki cevap verebilecek bir sorudur; bu
sonuclar ``reddetmek'' ya da ``reddetmemek'' olabilir. Dikkat: bu
sonuclardan biri ``kabul etmek'' degil, bir istatistiki hipotezi kabul
etmek mumkun degildir. Tek soyleyebildigimiz ``bir hipotezi reddetmek
icin elimizde yeterli veri olmadigini'' soylemektir. Ama
reddedebiliyorsak, bu sonucta daha bir kesinlik vardir.

Tek Orneklem t Testi (One-sample t test)

Bu test verinin Normal dagilimdan geldigini farzeder, tek orneklem
durumunda elde $x_1,...,x_n$ verisi vardir, ve bu veri $N(\mu,\Sigma)$
dagilimindan gelmistir ve test etmek istedigimiz hipotez / karsilastirma
$\mu = \mu_0$.

\begin{codecell}
\begin{codeinput}
\begin{lstlisting}
import numpy as np
from scipy.stats import ttest_1samp, wilcoxon, ttest_ind
import pandas as pd
\end{lstlisting}
\end{codeinput}
\end{codecell}
\begin{codecell}
\begin{codeinput}
\begin{lstlisting}
daily_intake = np.array([5260,5470,5640,6180,6390,6515, 6805,7515,7515,8230,8770])
df = pd.DataFrame(daily_intake)
df.describe()
\end{lstlisting}
\end{codeinput}
\begin{codeoutput}
\begin{verbatim}
                 0
count    11.000000
mean   6753.636364
std    1142.123222
min    5260.000000
25%    5910.000000
50%    6515.000000
75%    7515.000000
max    8770.000000
\end{verbatim}
\end{codeoutput}
\end{codecell}
\begin{codecell}
\begin{codeinput}
\begin{lstlisting}
t_statistic, p_value = ttest_1samp(daily_intake, 7725)
print "one-sample t-test", p_value
\end{lstlisting}
\end{codeinput}
\begin{codeoutput}
\begin{verbatim}
one-sample t-test 0.0181372351761
\end{verbatim}
\end{codeoutput}
\end{codecell}
Sonuc p\_value 0.05'ten kucuk cikti yani yuzde 5 onemliligini
(significance) baz aldik bu durumda veri hipotezden onemli derecede
(significantly) uzakta. Demek ki ortalamanin 7725 oldugu hipotezini
reddetmemiz gerekiyor.

Testi iki orneklemli kullanalim, gruplar 0/1 degerleri ile isaretlendi,
ve test etmek istedigimiz iki grubun ortalamasinin (mean) ayni oldugu
hipotezini test etmek. t-test bu arada varyansin ayni oldugunu farzeder.

\begin{codecell}
\begin{codeinput}
\begin{lstlisting}
energ = np.array([
[9.21, 0],
[7.53, 1],
[7.48, 1],
[8.08, 1],
[8.09, 1],
[10.15, 1],
[8.40, 1],
[10.88, 1],
[6.13, 1],
[7.90, 1],
[11.51, 0],
[12.79, 0],
[7.05, 1],
[11.85, 0],
[9.97, 0],
[7.48, 1],
[8.79, 0],
[9.69, 0],
[9.68, 0],
[7.58, 1],
[9.19, 0],
[8.11, 1]])
group1 = energ[energ[:, 1] == 0][:, 0]
group2 = energ[energ[:, 1] == 1][:, 0]
t_statistic, p_value = ttest_ind(group1, group2)
print "two-sample t-test", p_value
\end{lstlisting}
\end{codeinput}
\begin{codeoutput}
\begin{verbatim}
two-sample t-test 0.00079899821117
\end{verbatim}
\end{codeoutput}
\end{codecell}
$p-value < 0.05$ yani iki grubun ortalamasi ayni degildir. Ayni oldugu
hipotezi reddedildi.

Eslemeli t-Test (Paired t-test)

Eslemeli testler ayni deneysel birimin olcumu alindigi zaman
kullanilabilir, yani olcum alinan ayni grupta, deney sonrasi deneyin
etki edip etmedigi test edilebilir. Bunun icin ayni olcum deney sonrasi
bir daha alinir ve ``farklarin ortalamasinin sifir oldugu'' hipotezi
test edilebilir. Altta bir grup hastanin deney oncesi ve sonrasi ne
kadar yiyecek tukettigi listelenmis.

\begin{codecell}
\begin{codeinput}
\begin{lstlisting}
intake = np.array([
[5260, 3910],
[5470, 4220],
[5640, 3885],
[6180, 5160],
[6390, 5645],
[6515, 4680],
[6805, 5265],
[7515, 5975],
[7515, 6790],
[8230, 6900],
[8770, 7335],
])
pre = intake[:, 0]
post = intake[:, 1]
t_statistic, p_value = ttest_1samp(post - pre, 0)
print "paired t-test", p_value
\end{lstlisting}
\end{codeinput}
\begin{codeoutput}
\begin{verbatim}
paired t-test 3.05902094293e-07
\end{verbatim}
\end{codeoutput}
\end{codecell}
Wilcoxon isaretli-sirali testi (Wilcoxon signed-rank test)

t Testleri Normal dagilima gore sapmalari yakalamak acisindan, ozellikle
buyuk orneklemler var ise, oldukca saglamdir. Fakat bazen verinin Normal
dagilimdan geldigi faraziyesini yapmak istemeyebiliriz. Bu durumda
\emph{dagilimdan bagimsiz metotlar} daha uygundur, bu tur metotlar icin
verinin yerine cogunlukla onun sira istatistiklerini (order statistics)
kullanir.

Tek orneklemli Wilcoxon testi icin prosedur $\mu_0$'i tum veriden
cikartmak ve geri kalan (farklari) isaretine bakmadan numerik degerine
gore siralamak, ve bu sira degerini bir kenara yazmak. Daha sonra geri
donup bu sefer cikartma islemi sonucunun isaretine bakmak, ve eksi
isareti tasiyan sira degerlerini toplamak, ayni islemi arti isareti icin
yapmak, ve eksi toplami arti toplamindan cikartmak. Sonucta elimize bir
istatistik $W$ gelecek. Bu test istatistigi aslinda $1..n$ tane sayi
icinden herhangi birini $1/2$ olasiligiyla secmek, ve sonuclari
toplamaya tekabul etmektedir. Ve bu sonuc yine 0.05 ile karsilastirilir.

\begin{codecell}
\begin{codeinput}
\begin{lstlisting}
z_statistic, p_value = wilcoxon(daily_intake - 7725)
print "one-sample wilcoxon-test", p_value

\end{lstlisting}
\end{codeinput}
\begin{codeoutput}
\begin{verbatim}
one-sample wilcoxon-test 0.0279991628713
\end{verbatim}
\end{codeoutput}
\end{codecell}
Hipotezi yine reddettik.

Ustte yaptigimiz eslemeli t-testi simdi Wilcoxon testi ile yapalim,

\begin{codecell}
\begin{codeinput}
\begin{lstlisting}
z_statistic, p_value = wilcoxon(post - pre)
print "paired wilcoxon-test", p_value
\end{lstlisting}
\end{codeinput}
\begin{codeoutput}
\begin{verbatim}
paired wilcoxon-test 0.00463608893545
\end{verbatim}
\end{codeoutput}
\end{codecell}
Kaynaklar

https://gist.github.com/mblondel/1761714

\emph{Introductory Statistics with R}


\end{document}

\documentclass[12pt,fleqn]{article}\usepackage{../common}
\begin{document}
Ders 19

Eslenik Gradyan (Conjugate Gradient) Yontemi 

Arnoldi metotu Gram-Schmidt'e tam olmsa da benzeyen bir yontemdir ve bir
ortogonal baz ortaya cikartir. Bu baz, Krylov altuzayinin bazidir, ki bu
altuzaydaki her yeni baz vektor, $e$'nin baska bir ustu alinip carpilarak
elde edilir. Fakat bu pek iyi bir baz degildir, bazlarin ortogonalize
edilmesi gerekir, ve Arnoldi'nin yaptigi budur. 

Arnoldi-Lanczos yontemi ozdegerler (eigenvalue) bulmak icin de kullanilir.

\[ AQ = QH \]

esitligindeki $H$ matrisinin alt-matrisine bakilirsa, aranilan ozdegerler
buradan okunabilir. Bu alt-matris simetrik ve ust-caprazsaldir
(upper-diagonal). 

\[ H = Q^{-1}AQ \]

formulunde $H,A$ matrisleri birbirine benzerdir (similar) ve benzer
matrislerin ozdegerleri aynidir. 

Bu kavramlardan soyle bir bahsetmek istedim, belki gunun birinde cok buyuk
bir matrisin ozdegerlerini bulmak istersiniz, aklinizda olsun. Yazilim
\verb!arpack! bunun icin kullanilabiliyor. Bahsi yaptik bir diger sebep
lineer cebirin yarisi lineer sistemlerse, diger yarisi ozdeger
problemleridir denebilir. Buraya gelmisken ustteki ozdeger yonteminden
bahsetmemek olmazdi. 

Konumuza donelim. 

$A$ pozitif kesin ve simetrik. 




\end{document}

\documentclass[12pt,fleqn]{article}\usepackage{../common}
\begin{document}
Ders 18

[bazi multigrid yorumlari atlandi]

Krylov Matrisleri 

Bu matrislerden $K$ olarak bahsedecegiz ve bu yontem baglaminda 

\[ Ax = b \]

sistemini cozuyor olacagiz. Krylov matrisleri soyle yaratilir

\[ K = \left[\begin{array}{rrrrr}
b & Ab & A^2b & .. & A^{j-1}b
\end{array}\right] \]

Krylov altuzayi ise ustteki kolonlarin lineer kombinasyonudur (span), ya da
ustteki matrisin kolon uzayidir da denebilir. Bu tur bir matrisle niye
ilgilenirim? Jacobi islemi aslinda bu kolonlarin kombinasyonlarindan birini
her adimda yavas yavas secer, yani aslinda Krylov altuzayinin bir
parcasinda calisir. Daha dogrusu ufak ufak baslar, o altuzayda yavas yavas
buyur.

Jacobi surekli bir kombinasyon secimi yapar, tabii bu secimin en iyi secim
oldugu soylenemez. Secimin en iyisini yapsak daha iyi olmaz mi? 

En iyiyi secmek icin kullanilacak metot eslenik gradyan (conjugate
gradient) olacak. Bu metot $K$ icinde $x_j$'yi secer. 

$K$ uzayi yaklasiksal cozumumuzu aradigimiz yer tabii ki. Bu arada ustteki
$K$ uzayinin elemanlarini yaratmak cok kolay, matris carpimi yapiyoruz, ve
bir sonraki eleman bir oncekinin $A$ katidir, ve $A$ cogunlukla seyrektir
(sparse), bazen de simetriktir (eslenik gradyan metotu icin $A$ simetrik,
pozitif kesin olmali).

Ama EG metotundan once Arnoldi kavramini gormemiz lazim. 

Uygulamali Matematikte surekli bir seyler ``seceriz'', ve cogunlukla baz
vektorleri seceriz ve birkac ozellik arariz. Aradigimiz ozellikler
oncelikle hizdir, yukarida gordugumuz gibi, matris carpimi var, bu cok
hizli. Bir diger ozellik bagimsizlik. Bir digeri baz vektorlerinin
ortonormal olmasi. Bu son ozellik elde edilebilirse en iyisidir. Ustteki
$K$ pek iyi bir baz degildir. Arnoldi'nin amaci Krylov bazini ortogonalize
etmektir. $b,Ab,..$'yi alip $q_1,q_2,..,q_j$ olusturmaktir.













\end{document}

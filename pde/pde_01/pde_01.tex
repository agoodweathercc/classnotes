\documentclass[12pt,fleqn]{article}
\setlength{\parindent}{0pt}
\usepackage{graphicx}
\usepackage{cancel}
\usepackage{listings}
\usepackage[latin5]{inputenc}
\usepackage{color}
\setlength{\parskip}{8pt}
\setlength{\parsep}{0pt}
\setlength{\headsep}{0pt}
\setlength{\topskip}{0pt}
\setlength{\topmargin}{0pt}
\setlength{\topsep}{0pt}
\setlength{\partopsep}{0pt}
\setlength{\mathindent}{0cm}

\begin{document}
Analitik PDE - Ders 1

Dersin kullanacagi ana kitap L. C. Evans'in Kismi Turevsel Denklemler
(partial differential equations -PDE-) kitabi olacak. Bir sonraki ders icin
okuma odevi soyle:

1. sf. 1-13'teki ozet

2. Alt bolum 2.1 sf. 17-19

3. Bolum 3 sf. 91-115 arasini tamamen. 

PDE'leri incelerken cogunlukla onlarin temsil ettigi fiziksel fenomenleri
de inceleyecegiz. Mesela transportasyon (transport) denklemleri, ki

\[ \partial_t u + \vec{b} \cdot \vec{\nabla} u = 0 \]

Ustteki ifadede gradyan operatoru var, bu bilindigi gibi

\[ \vec{\nabla} = \bigg( 
\frac{\partial }{\partial x_1},.., 
\frac{\partial }{\partial x_n},
\bigg)
\]

$\vec{b}$ icinde sabitler olan bir vektor olabilir

\[ \vec{b} = (b_1,...,b_n)
 \]

Bu denklem 1. derece PDE'lerin ozel bir durumudur bu arada. 1. derece
PDE'ler 

\[ F(x, u(x), Du(x)) = 0 \]

seklindedir. $D$ notasyonu Evans'in gradyan icin kullandigi notasyon,
alissak iyi olur. Yani

\[ Du = \nabla u \]





















\end{document}

\documentclass[12pt,fleqn]{article}
\setlength{\parindent}{0pt}
\usepackage{graphicx}
\usepackage{cancel}
\usepackage{listings}
\usepackage[latin5]{inputenc}
\setlength{\parskip}{8pt}
\setlength{\parsep}{0pt}
\setlength{\headsep}{0pt}
\setlength{\topskip}{0pt}
\setlength{\topmargin}{0pt}
\setlength{\topsep}{0pt}
\setlength{\partopsep}{0pt}
\setlength{\mathindent}{0cm}

\begin{document}
PDE - Ders 1

Kismi Turevsel Denklemler (partial differential equtions -PDE-) Bu dersin
on gerekliliklerinden en onemlisi normal diferansiyel denklemlerdir (ordinary
differential equtions -ODE-), cunku pek cok PDE'yi cozmenin teknigi onlari
bir ODE sistemine indirgemekten geciyor. Yani PDE cozmek icin ODE cozme
tekniklerini de bilmek gerekiyor. Bir diger gerekli bilgi Lineer Cebir
dersi. 


\end{document}

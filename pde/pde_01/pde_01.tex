\documentclass[12pt,fleqn]{article}
\setlength{\parindent}{0pt}
\usepackage{graphicx}
\usepackage{cancel}
\usepackage{listings}
\usepackage[latin5]{inputenc}
\setlength{\parskip}{8pt}
\setlength{\parsep}{0pt}
\setlength{\headsep}{0pt}
\setlength{\topskip}{0pt}
\setlength{\topmargin}{0pt}
\setlength{\topsep}{0pt}
\setlength{\partopsep}{0pt}
\setlength{\mathindent}{0cm}

\begin{document}
PDE - Ders 1

Konumuz Kismi Turevsel Denklemler (partial differential equtions -PDE-). Bu
dersin on gerekliliklerinden en onemlisi normal diferansiyel denklemlerdir
(ordinary differential equtions -ODE-), cunku pek cok PDE'yi cozmenin
teknigi onlari bir ODE sistemine indirgemekten geciyor. Yani PDE cozmek
icin ODE cozme tekniklerini de bilmek gerekiyor. Bir diger gerekli bilgi
Lineer Cebir dersi.

Bu dersin ana amaci, bir muhendislik dersi olarak, denklem cozmek, ve pek
cok denklemin cikis noktasi fiziksel problemler. Mesela sicaklik yayilmasi
(heat diffusion), dalga hareketi (wave motion), titresen hucre zari
(vibrating membrane) gibi. Fakat PDE kavrami finansta bile ortaya cikabilen
bir kavram, mesela Black-Sholes denklemlerinde oldugu gibi. 

Yani dersimiz cok teori odakli olmayacak, bazi ispatlardan bahsedecegiz,
ama onun haricinde teori uzerinde fazla durmayacagiz. 

PDE nedir? Ilk once ODE tanimindan baslayalim. 

\[ y = y(x) \]

\[ \frac{dy}{dx} = y \]

Baslangic sartlari 

\[ y(0) = y_0 \]

Cozum 

\[ y = y_0e^x \]

Bu bir ODE cunku sadece bir tane bagimsiz degisken var ($x$), ve bir tane
bagimli degisken var ($y$). 

PDE ise icinde kismi turevleri, ve bir veya {\em birden fazla} bagimsiz
degiskeni barindiran bir denklemdir.

Eger gunes etrafindaki yorungeleri temsil etmek istiyorsaniz gezegenleri
boyutsuz parcaciklar gibi kabul ederek ODE'ler ile temsil etmek yeterli
olabilir, ama diger problemlerde daha fazla bagimsiz degisken gerekecegi
icin ODE yetmez, mesela zaman, cismin 3D uzaydaki boyutlari gibi.







\end{document}

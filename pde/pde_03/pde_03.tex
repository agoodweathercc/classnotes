\documentclass[12pt,fleqn]{article}
\setlength{\parindent}{0pt}
\usepackage{graphicx}
\usepackage{cancel}
\usepackage{listings}
\usepackage[latin5]{inputenc}
\usepackage{color}
\setlength{\parskip}{8pt}
\setlength{\parsep}{0pt}
\setlength{\headsep}{0pt}
\setlength{\topskip}{0pt}
\setlength{\topmargin}{0pt}
\setlength{\topsep}{0pt}
\setlength{\partopsep}{0pt}
\setlength{\mathindent}{0cm}

\begin{document}
Ders 3

Dalga Denklemi Baslangic Deger Problemi (Initial Value Problem) 

Baslangic deger problemi 

\[ u_{tt} = c^2u_{xx} \]

\[ -\infty < x < \infty \]

ve baslangic sartlari 

\[ u(x,0) = \phi(x) \]

\[ u_t(x,0) = \psi(x) \]

ki $\phi$ ve $\psi$, $x$'in herhangi bir fonksiyonu. 

Her $\phi$ ve $\psi$ icin bu problemde tek bir cozum vardir, yani tek bir
$u$ elde edilir. Mesela $\phi(x) = sin(x)$ ve $\psi(x) = 0$ ise, o zaman
$u(x,t) = sin \ x \ cos \ ct$. 

Kontrol edelim. Cozum soyleydi:

\[ u(x,t) = f(x+ct) + g(x-ct) \ \ \ (5)\]

$t=0$ dersek 

\[ u(x,0)  = f(x) + g(x) \]

$u(x,0) = \phi(x)$ demistik

\[ \phi(x) = f(x) + g(x) \ \ \ (1) \]

Zincirleme Kanunu kullanarak $t$'ye gore turev alalim ve $t=0$ diyelim

\[ u_t(x,0) = cf' - cg' \]

$u_t(x,0) = \psi(x)$ demistik

\[ \psi(x) = cf'(x) - cg'(x) \ \ \ (2) \]

Formul (1) ve (2)'yi bir denklem sistemi olarak gorebiliriz, ve ayni anda
cozmeye ugrasabiliriz. (1)'in turevini alalim, ve (2)'yi $c$'ye bolelim

\[ \phi' = f' + g' \]

\[ \frac{1}{c}\psi = f' - g' \]

Ustteki iki denklemi toplarsak

\[ f' = \frac{1}{2} \bigg( \phi' + \frac{\psi}{c}  \bigg) \]

Cikarirsak

\[ g' = \frac{1}{2}  \bigg( \phi' - \frac{\psi}{c}  \bigg) \]


$f'$'in entegralini alalim. Ek: $x$ yerine $s$ kullanacagiz simdi,

\[ \int f' = \int \frac{1}{2} \bigg( \phi' + \frac{\psi}{c}  \bigg) \]

\[ f = \frac{1}{2}\phi(s) + \frac{1}{2}\int_0^s \frac{\psi(x)}{c} + A\]

\[ f = \frac{1}{2}\phi(s) + \frac{1}{2c} \int_0^s \psi(x) + A 
\ \ \ (3)
\]

$g'$'nin entegrali

\[ g = \frac{1}{2}\phi(s) - \frac{1}{2c} \int_0^s \psi(x) + B
\ \ \ (4)
\]


ki $A$ ve $B$ sabitler. $A=0,B=0$ cunku (3) ve (4) toplaninca icinde $A,B$
olmayan (1) elde edilir, o zaman $A=0,B=0$ olmalidir. Boylece genel formul
(5) icin gereken $f,g$ fonksiyonlarini elde etmis oluyoruz. Genel formulde
biraz once buldugumuz $f,g$ yerine gecirirsek, ve $f$ icin $s = x+ct$, $g$
icin $s = x-ct$ kullanirsak,

\[ u(x,t) = 
\frac{1}{2}\phi(x+ct) + 
\frac{1}{2c} \int_0 ^{x+ct} \psi + 
\frac{1}{2}\phi(x-ct) -
\frac{1}{2c} \int_0 ^{x+ct} \psi 
 \]

Basitlestirirsek 

\[ = \frac{1}{2}[\phi(x+ct) + \phi(x-ct)] + 
\frac{1}{2c} \int_{x-ct} ^{x+ct} \psi (s) ds
 \]

Bu cozum matematikci d'Alembert'in 1746'da buldugu baslangic deger
probleminin cozumdur. 

Ornek

\[ \phi(x) = 0, \psi(x) = cos(x) \] 

ise cozum 

\[ u(x,t) = \frac{1}{2c}[sin(x+ct) - sin(x-ct)] \]

Trigonometrik esitlik kullanirsak, her iki terimin acilimi

\[ sin(x+ct) = cos(x)sin(ct) + sin(x)cos(ct) \]

\[ sin(x-ct) = -cos(x)sin(ct) + sin(x)cos(ct) \]

Ustteki formullerden 2.'yi 1.'den cikartalim, 

\[ sin(x+ct) - sin(x-ct) = 2cos(x)sin(ct) \]

$u(x,t)$ icinde yerine koyalim

\[ u(x,t) = \frac{1}{c}cos(x)sin(ct) \]

Cozum bu. Kontrol edersek, 

\[ u_{tt} = -c cos(x)sin(ct) \]

\[ u_{xx} = -(1/c)cos(x)sin(ct) \]

boylece 

\[ u_{tt} = c^2 u_{xx} \]

Baslangic sartini kontrol etmek daha da kolay, ilk sart

\[ u(x,0) = \frac{1}{c}cos(x)sin(c \cdot 0)  = 0\]

Ikinci sart

\[ u_t(x,t) = \frac{1}{c}cos(x) \ c \ cos(ct) \]

\[ u_t(x,0) = \frac{1}{c}cos(x) \ c \]

\[  = cos(x) \]


* * * 


Yayilim (Diffusion) Denklemi

Bu bolumde cozmeye ugrasacagimiz problem

\[  u_t = k u_{xx}  \]

\[ u(x,0) = \phi(x) \]

oyle ki 

\[ -\infty < x < \infty, 0 < t < \infty \]

Bu denklemi ozel (particular) bir $\phi$ icin cozecegiz, ve sonra genel
cozumu bu ozel cozumden elde edecegiz. Bunu yaparken yayilim denklemin 5
tane degismezlik (invariance) ozelligini kullanacagiz.

a) Bir sabit $y$ icin $u(x,y)$ cozumunu $u(x-y,t)$ olarak tercume edilir /
tasinirsa (translate), ortaya cikan denklem yeni bir cozumdur.

b) Cozumun her turevi, mesela $u_x$, $u_t$, $u_{xx}$ gibi, yeni bir
cozumdur. 

c) Cozumlerin lineer kombinasyonlari yeni bir cozumdur. Bu lineerligin bir
sonucu zaten.

d) Iki cozumun entegrali bir cozumdur, mesela $S(x,t)$ bir cozum ise, o
zaman $S(x-y,t)$ de bir cozumdur ((a)'nin soyledigi gibi), o zaman

\[ v(x,t) = \int_{-\infty}^{\infty}S(x-y,t)g(y) dy   \]

bir cozumdur. Aslinda bu beyan, (c)'nin limite gidiyorkenki versiyonu,
sekli.

e) Eger $u(x,t)$ yayilim denkleminin bir cozumu ise, o zaman genisletilmis
(dilated) fonksiyon $u(\sqrt{a} x, at)$ da, herhangi bir $a>0$ icin, bir
cozumdur. Ispat (Zincirleme Kanunu ile)

\[ v(x,t) =  u(\sqrt{a} x, at) \]

\[ v_t = [\partial(at) / \partial t]u_t = au_t \]

\[ v_x = [\partial(\sqrt{a}x) / \partial x]u_x = \sqrt{a}u_x \]

\[ v_{xx} = \sqrt{a} \cdot \sqrt{a} u_{xx} = a u_{xx} \]

Hedefimiz ozel bir cozum bulmak ve ozellik (d)'yi kullanarak tum diger
cozumleri insa etmek. Aradigimiz ozel fonksiyon ise $Q(x,t)$ adini
verecegimiz  bir fonksiyon ve bizim tanimladigimiz ozel baslangic
sartlari soyle:

\[ Q(x,0) = 1, \ x>0 \textit{ icin } \]

\[ Q(x,0)=0, \ x<0 \textit{ icin }  \]

Bu baslangic sartlarini secmemizin sebebi genisletme (dilation)
operasyonunun bu sartlara etki etmemesi. 

$Q$'yu uc adimda bulacagiz. 

1. Adim

Cozumun su ozel formda olacagini soyluyoruz. 

\[ Q(x,t) = g(p), \ p = \frac{x}{\sqrt{4kt}} \]

$g$ tek degiskenli bir fonksiyon, fonksiyonun ne oldugunu sonra
tanimlayacagiz, $\sqrt{4kt}$ ise daha ilerideki baska bir formulu
basitlestirmek icin kullaniliyor. 

$Q$'nin niye bu ozel formda olmasini istiyoruz? Cunku cozumun genisletmeye
``dayanikli'' olmasi gerekiyorsa, boyle bir cozum sadece $Q$ $x,t$'ye
$x/\sqrt{t}$ kombinasyonu uzerinden bagliysa mumkundur. Kontrol edelim,
genisleme $x/\sqrt{t}$'yi alip $\sqrt{a}x/\sqrt{at} = x/\sqrt{t}$ yapar,
yani baslangica dondurur. 

2. Adim









\end{document}

\documentclass[12pt,fleqn]{article}
\setlength{\parindent}{0pt}
\usepackage{graphicx}
\usepackage{cancel}
\usepackage{listings}
\usepackage[latin5]{inputenc}
\usepackage{color}
\setlength{\parskip}{8pt}
\setlength{\parsep}{0pt}
\setlength{\headsep}{0pt}
\setlength{\topskip}{0pt}
\setlength{\topmargin}{0pt}
\setlength{\topsep}{0pt}
\setlength{\partopsep}{0pt}
\setlength{\mathindent}{0cm}

\begin{document}
Ders 3

Yayilim (Diffusion) Denklemi

Bu bolumde cozmeye ugrasacagimiz problemi

\[  u_t = k u_{xx}, \ \ (-\infty < x < \infty, 0 < t < \infty) \]

\[ u(x,0) = \phi(x) \]


Bu denklemi ozel (particular) bir $\phi$ icin cozecegiz, ve sonra genel
cozumu bu ozel cozumden cikartacagiz. Bunu yaparken yayilim denkleminin 5
degismezlik (invariance) ozelligini kullanacagiz. 

a) Her sabit $y$ icin $u(x,y)$ cozumunu $u(x-y,t)$ olarak tercume edilir /
tasinirsa (translate), ortaya cikan denklem yeni bir cozumdur.

b) Cozumun her turevi, mesela $u_x$, $u_t$, $u_{xx}$ gibi, yeni bir
cozumdur. 

c) Cozumlerin lineer kombinasyonlari yeni bir cozumdur. 

d) Iki cozumun entegrali bir cozumdur, mesela $S(x,t)$ bir cozum ise, o
zaman $S(x-y,t)$ de bir cozumdur (bunu (a) soyluyor zaten), o zaman 

\[ v(x,t) = \int_{-\infty}^{\infty}S(x-y,t)g(y) dy   \]

bir cozumdur.














\end{document}
